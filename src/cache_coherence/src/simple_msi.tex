\subsection{Introduction to the MSI Protocol}
\label{sec:intro_to_msi}
The archetypal cache coherence protocols are those belonging to the MSI
protocol family. In their basic version, these protocols feature three stable
states from which the MSI name is derived:
\begin{itemize}
\item \textbf{Modified:}
When a cache assigns the Modified state to a memory element, it indicates that
this cache performed at least one write to that memory element. While in this
state, it can freely read and write to that memory element. This corresponds
to being the \textit{Single Writer} from Property~\ref{prop:swmr}.

\item \textbf{Shared:}
When a cache assigns the Shared state to a memory element, it indicates that
this cache is able to freely read that memory element but not write to it. This
corresponds to being one of the \textit{Multiple Readers} (and potentially the
only one) of Property~\ref{prop:swmr}.

\item \textbf{Invalid:}
When a cache assigns the Invalid state to a memory element, it is no longer
considered to be holding a copy of that memory element, thus absolving this
cache from verifying Property~\ref{prop:system_wide_value}. Not having a copy,
the cache can neither read nor write to the memory element.
\end{itemize}

\stopallthesefloats
\begin{figure}
   \begin{center}
   \begin{subfigure}[b]{0.7\linewidth}
      \begin{tikzpicture}[->,>=stealth',shorten >=1pt,auto,node distance=5cm,
                    semithick]
   \node[state] (I)              {\texttt{I}};
   \node[state] (S) [right of=I] {\texttt{S}};
   \node[state] (M) [below of=I] {\texttt{M}};

   \path[every node/.style={}]
      (I) edge [bend left] node {\textcolor{blue}{\loadinstr{}}?\textcolor{red}{\textcolor{red}{\getsquery{}}}!\textcolor{OliveGreen}{\simpledata{}}?} (S)
      (I) edge [bend right, sloped, anchor=center] node[yshift=-1em] {\textcolor{blue}{\storeinstr{}}?\textcolor{red}{\textcolor{red}{\getmquery{}}}!\textcolor{OliveGreen}{\simpledata{}}?} (M)

      (S) edge node {\textcolor{blue}{\evictinstr{}}?} (I)
      (S) edge [bend right=90, above] node {\textcolor{red}{\textcolor{red}{\getmquery{}}}?} (I)
      (S) edge [bend left, sloped, anchor=center] node[yshift=1.5em] {\textcolor{blue}{\storeinstr{}}?\textcolor{red}{\textcolor{red}{\getmquery{}}}!\textcolor{OliveGreen}{\simpledata{}}?} (M)
      (S) edge [out=430,in=400,looseness=8] node {\textcolor{blue}{\loadinstr{}}?} (S)

      (M) edge [bend left=90, above, sloped] node {\textcolor{red}{\textcolor{red}{\getmquery{}}}?\textcolor{OliveGreen}{\simpledata{}}!} (I)
      (M) edge [bend right=90, above, sloped, anchor=center] node[yshift=-1em] {\textcolor{red}{\textcolor{red}{\getsquery{}}}?\textcolor{OliveGreen}{\simpledata{}}!\textcolor{OliveGreen}{\simpledata{}}!} (S)
      (M) edge [sloped, anchor=center] node[yshift=-1em]
         {\textcolor{blue}{\evictinstr{}}?\textcolor{red}{\putmquery{}}!\textcolor{OliveGreen}{\simpledata{}}!} (I)
      (M) edge [out=230,in=200,looseness=8] node {\textcolor{blue}{\loadinstr{}}?} (M)
      (M) edge [out=280,in=250,looseness=8] node {\textcolor{blue}{\storeinstr{}}?} (M)
   ;
\end{tikzpicture}

      \caption{Cache}
      \label{fig:general_msi_cc}
   \end{subfigure}
   \\
   \begin{subfigure}[b]{0.5\linewidth}
      \begin{tikzpicture}[->,>=stealth',shorten >=1pt,auto,node distance=5cm,
                    semithick]
   \node[state] (U)              {\texttt{I}};
   \node[state] (M) [left of=U] {\texttt{M}};

   \path[every node/.style={sloped, anchor=center, yshift=1em}]
      (U) edge [bend left] node {\textcolor{red}{\textcolor{red}{\getmquery{}}}?\textcolor{OliveGreen}{\simpledata{}}!} (M)
      (U) edge [loop below] node [below,yshift=-1em] {\textcolor{red}{\textcolor{red}{\getsquery{}}}?\textcolor{OliveGreen}{\simpledata{}}!} (U)

      (M) edge [bend left] node {\textcolor{red}{\putmquery{}}?\textcolor{OliveGreen}{\simpledata{}}?} (U)
      (M) edge [bend left=90] node {\textcolor{red}{\textcolor{red}{\getsquery{}}}?\textcolor{OliveGreen}{\simpledata{}}?} (U)
   ;
\end{tikzpicture}

      \caption{Coherence Manager}
      \label{fig:general_msi_cmgr}
   \end{subfigure}
   \begin{subfigure}[b]{0.3\linewidth}
      \begin{tikzpicture}[->,>=stealth',shorten >=1pt,auto,node distance=5cm,
                    semithick]
   \node[state] (U)              {};

   \path[every node/.style={sloped, anchor=center, yshift=1em}]
      (U) edge [loop left] node [below,yshift=0.5em] {\textcolor{blue}{\loadinstr{}}!} (U)
      (U) edge [loop right] node [below,yshift=-1em] {\textcolor{blue}{\storeinstr{}}!} (U)
      (U) edge [loop below] node [below,yshift=-1em] {\textcolor{blue}{\evictinstr{}}!} (U)
   ;
\end{tikzpicture}

      \caption{Core}
      \label{fig:general_msi_cpu}
   \end{subfigure}
   \end{center}
   \caption{Overview of the MSI Protocol}
   \label{fig:general_msi}
\end{figure}

To keep it simple, the protocol presented in this section does not use any FIFO
queue. Instead, all exchanges are done synchronously. The only elements taken
into account are the state of the memory elements and the points of
synchronization (i.e. instruction and message exchanges). Thus, the notations
on the automata of this section are not the notations presented in the previous
section, but are instead based on CCS (Calculus of Communicating Systems,
\cite{10.5555/539036}): a \texttt{?} marks the reception of a signal, and
\texttt{!} marks its sending.  Queries are still sent to all components, and
data only to a single one.  Furthermore, only a single transaction (that is,
the resolution of an instruction) can occur at any
moment. This is reflected by transitions featuring multiple synchronizations,
as no other actions would be permitted to be interleaved.

Figure~\ref{fig:general_msi} shows the two automata corresponding to this
simplified protocol. Each of these automata pertains to a single memory element,
with states corresponding to what the component attributes to that memory
element.

In the case of the coherence manager (Figure~\ref{fig:general_msi_cmgr}), the
two states simply correspond to whether or not the coherence manager is in
charge of providing the current value for that memory element: \texttt{M}
indicates that one of the caches is currently in the \texttt{M} state, meaning
that the value stored in the system's main memory may be out of date; \texttt{I}
indicates that the main memory's value for that element is up-to-date, leaving
(in the MSI protocol) the coherence manager in charge of propagating copies
as the caches demand them.

\begin{example}[Simple MSI Transition]
\label{ex:general_msi_single_store}
Consider a system with two caches, $C_A$ and $C_B$, and a single memory
element $E$, such that, initially, $C_A$ holds no copy of $E$ (\texttt{I}
state) and $C_B$ holds one in the \texttt{S} state, the coherence manager
has the \texttt{I} state assigned to $E$. Were $C_A$ to receive a \storeinstr{}
request, it would broadcast a \getmquery{} to itself, $C_B$, and the coherence
manager. Upon receiving the \getmquery{}, $C_B$ would have to abandon the
\texttt{S} state for \texttt{I} in order to maintain Property~\ref{prop:swmr}.
The coherence manager would react to receiving the \getmquery{} by sending a
copy of $E$ from the system's main memory to $C_A$ (\simpledata!)
before moving to \texttt{M} so as to memorize that the most up-to-date value
for $E$ is now held by $C_A$. This \simpledata{} reply allows $C_A$ to move to
the \texttt{M} state, thus completing its transition.
\end{example}


