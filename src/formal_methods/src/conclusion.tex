\section{Conclusion}
Timed automata can be used to model complex systems featuring real-time
constraints. Through the use of queries verified using formal methods such as
model checking, these models are then used to validate properties for the
systems, such as ascerting its correct behavior or computing running time for
some of its components.

In the next chapter, cache coherence mechanics are described. The behavior of
cache coherence protocols is defined using (classical) automata. Because of the
large size of such automata, a matrix representation is used instead. In this
representation, lines correspond to locations, and columns correspond to labels
and guards. For example, the client automaton described in
Figure~\ref{fig:classical_automata} would have the matrix
representation shown in Figure~\ref{fig:matrix_example}.

\begin{figure}[hbt!]
   \centering
   \begin{tabular}{|l||c|c|c|c|}
 \hline

 \textbf{Location}
 & \textbf{request\_files!}
 & \textbf{done?}
 & \textbf{err}
 & \textbf{new\_file?}
 \\

%%%%%%%%%%%%%%%%%%%%%%%%%%%%%%%%%%%%%%%%%%%%%%%%%%%%%%%%%%%%%%%%%%%%%%%%%%%%%%%%
 \hline

%%%%%%%%%%%%%%%%%%%%%%%%%%%%%%%%%%%%%%%%%%%%%%%%%%%%%%%%%%%%%%%%%%%%%%%%%%%%%%%%
 $S_0$

 & $fetched := 0$, $S_1$
 & \disablecell{}
 & $S_E$
 & \disablecell{}
 \\
 \hline

 $S_1$

 & \disablecell{}
 & $S_0$
 & $S_E$
 & $fetched := fetched + 1$
 \\
 \hline
 $S_E$

 & \disablecell{}
 & \disablecell{}
 & \disablecell{}
 & \disablecell{}
 \\
 \hline
\end{tabular}%

   \caption{Example of matrix automaton representation}
   \label{fig:matrix_example}
\end{figure}
