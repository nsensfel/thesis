\subsection{Strategy Application for a MESI Protocol}
\label{sec:identification:application}
This section presents the application of the naive exploration of the strategy
being applied to the NXP QorIQ T4240 architecture. The steps presented here have
a slight deviation from those indicated in the strategy description. Indeed, the
strategy saw an improvement between its application as described here and its
addition to the thesis: at the time, the absence of observable coherence manager
led to its state being ignored rather than it being assumed to match the
hypothetical protocol. The updated strategy detects mismatches faster. In this
application of the strategy, a mismatch in coherence manager behavior is likely
to only be detected during the guided exploration.

\subsection{Coherence State Matching}
\begin{figure}[hbt]
\input{\chapterdirectory/figure/mesi_obs_trans_states.tex}
\caption{T4240 L2 Cache state transitions}
\label{fig:identification:state_transitions}.
\end{figure}

Applying Step~\ref{step:reachability} of
Figure~\ref{fig:identification:state_exploration} on the T4240 yields the
stable state transitions summarized in the \textit{Origin} and \textit{Observ}
columns of Figure~\ref{fig:identification:state_transitions}. The
\textit{Origin} column corresponds to the observed system state in
\obssystemstate{} prior to the operation being performed, and each
\textit{Observ} corresponds to the element of \obssystemstate{} observed upon
application of the given instruction (see top row) on the first cache. The
figure covers all the possible sets of stable states for the coherence of a
single memory element on the system's clusters, since the permutation of two
clusters does not impact the cache coherence's mechanisms. The transitions,
however, are limited to those relevant when only a single operation is applied
across the whole system. Furthermore, this does not account for any state of
the coherence manager, as these cannot be observed, as explained in
Limitation~\ref{lim:no_cmgr}.

Once these observations are completed, Step~\ref{step:state_matching} can start:
the observed states can be matched with the hypothetical ones: the observed
\benchm{}, \benche{}, and \benchi{} states perfectly match their \texttt{M},
\texttt{E}, and \texttt{I} counterparts from the MESI protocol. The \texttt{S}
state, however, seems to match our observations of both the \benchs{} and
\benchi{} states. Indeed, starting from
\texttt{$\langle$\benchi{},\benchi{},\benchm{}$\rangle$} and performing a
\texttt{load} operation on the first cluster, leads to two different states,
\benchs{} and \benchx{}, where the \texttt{S} state equivalent would have been
expected. The same occurs when starting from
\texttt{$\langle$\benchi{},\benchi{},\benche{}$\rangle$}. By itself, this
observation is not sufficient to conclude that there is a discrepancy between
hypothetical protocol and the observed one. Indeed, the two observed states are
marked as separate, their difference may very well not be related to cache
coherence (e.g.~hint for cache eviction) and the two states may actually simply
correspond to our \texttt{S} state. It is only by comparing how those now
exposed two states differ, if at all, in their behavior that we may conclude
whether they represent an inconsistency.

As we go through the different transitions from one stable state to another, we
observe that performing an evict on either \benchs{} or \benchx{} does not
affect the other caches' state, which means that reaching either
\texttt{$\langle$\benchx{},\benchi{},\benchi{}$\rangle$} or \texttt{$\langle$\benchs{},\benchi{},\benchi{}$\rangle$}
(or any permutation of these clusters) is possible.
In addition, the previous step showed that there is no
way to have a system in which two clusters hold the same memory element in the
\benchs{} state: the first cluster to reach the \benchs{} moves to the \benchx{}
state upon seeing the other's query. Neither is it possible to have all three
clusters in the \benchx{} state: the last cluster to load from \benchi{} always
enters \benchs{}, and there is no way to reach \benchs{} other than doing
exactly that.

\begin{figure}[hbt]
\begin{center}
\begin{tabular}{|c|c|c|c|c|c|}
\hline
State & \textit{Dirty} & \textit{Valid} & \textit{Share} & \textit{Exclusive}
& \textit{LastReader}
\\
\hline
\benchm{} & \checkmark & \checkmark &   &   &  \\
\hline
\benche{} &   & \checkmark &   & \checkmark &  \\
\hline
\benchi{} &   &   &   &   &  \\
\hline
\benchs{} &   & \checkmark &   &   & \checkmark\\
\hline
\benchx{} &   & \checkmark & \checkmark &   &  \\
\hline
\end{tabular}
\end{center}
\caption{Observable Coherency Cache States of the T4240 L2 Caches Protocol}
\label{fig:benchmark_states}
\end{figure}

A benefit of using CodeWarrior is that the fields defining the state of a cache
line are labelled. Figure~\ref{fig:benchmark_states} indicates the fields
corresponding to each observed cache state.

\subsection{Coherence Activity Matching}
While having both \benchx{} and \benchs{} mapped to the \texttt{S} state does
not contradict the hypothetical protocol, it does raise suspicions of a
protocol mismatch. This suspicion is confirmed by the activity analysis
(Step~\ref{step:behavior_matching} of
Figure~\ref{fig:identification:state_exploration}).

\begin{figure}[hbt!]
\begin{center}
\begin{tabular}{|c|c|c|}
\cline{2-3}
\multicolumn{1}{c}{}
& \multicolumn{2}{|c|}{$\langle$\texttt{load, -, -}$\rangle$}
\\

\hline
   \multirow{2}{*}{\textbf{Origin}}
 & \multicolumn{2}{c|}{\textbf{Behavior}}
\\
 & \textbf{Expected}
 & \textbf{Observed}
\\
\hline

%%%% I/\benchi{}/\benchi{} %%%%%%%%%%%%%%%%%%%%%%%%%%%%%%%%%%%%%%%%%%%%%%%%%%%%%%%%%%%%%%%%%%%%%
% Origin
\texttt{\textbf{$\langle$\benchi{}},\benchi{},\benchi{}$\rangle$}

% Behavior (Expected, Observed)
&
   \begin{tabular}{@{}l@{}}
      8000 L2D Accesses,\\
      8000 Reloads From CoreNet
   \end{tabular}
&
   \begin{tabular}{@{}l@{}}
      16000 L2D Accesses,\\
      8000 Reloads From CoreNet,\\
      1166700 CPU Cycles
   \end{tabular}
\\
\hline

%%%% I/\benchi{}/\benchs{} %%%%%%%%%%%%%%%%%%%%%%%%%%%%%%%%%%%%%%%%%%%%%%%%%%%%%%%%%%%%%%%%%%%%%
% Origin
\texttt{\textbf{$\langle$\benchi{}},\benchi{},\benchs{}$\rangle$}

% Behavior (Expected, Observed)
&
   \begin{tabular}{@{}l@{}}
      8000 L2D Accesses,\\
      8000 Reloads From CoreNet
   \end{tabular}
&
   \begin{tabular}{@{}l@{}}
      16000 L2D Accesses,\\
      8000 Reloads From CoreNet,\\
      850600 CPU Cycles
   \end{tabular}
\\
\hline

%%%% I/\benchi{}/X %%%%%%%%%%%%%%%%%%%%%%%%%%%%%%%%%%%%%%%%%%%%%%%%%%%%%%%%%%%%%%%%%%%%%
% Origin
\texttt{$\langle$\textbf{\benchi{}},\benchi{},\benchx{}$\rangle$}

% Behavior (Expected, Observed)
&
   \begin{tabular}{@{}l@{}}
      8000 L2D Accesses,\\
      8000 Reloads From CoreNet
   \end{tabular}
&
   \begin{tabular}{@{}l@{}}
      16000 L2D Accesses,\\
      8000 Reloads From CoreNet,\\
      1172600 CPU Cycles
   \end{tabular}
\\
\hline
\end{tabular}

\vspace{1em}

\begin{tabular}{|c|c|c|}
\cline{2-3}
\multicolumn{1}{c}{}
& \multicolumn{2}{|c|}{$\langle$\texttt{-, -, load}$\rangle$}
\\

\hline
   \multirow{2}{*}{\textbf{Origin}}
 & \multicolumn{2}{c|}{\textbf{Behavior}}
\\
 & \textbf{Expected}
 & \textbf{Observed}
\\
\hline

%%%% I/\benchi{}/\benchi{} %%%%%%%%%%%%%%%%%%%%%%%%%%%%%%%%%%%%%%%%%%%%%%%%%%%%%%%%%%%%%%%%%%%%%
% Origin
\texttt{$\langle$\textbf{\benchi{}},\benchi{},\benchi{}$\rangle$}

% Behavior (Expected, Observed)
&
   \begin{tabular}{@{}l@{}}
      8000 External Snoop Requests
   \end{tabular}
&
   \begin{tabular}{@{}l@{}}
      8000 External Snoop Requests
   \end{tabular}
\\
\hline

%%%% F/\benchi{}/\benchi{} %%%%%%%%%%%%%%%%%%%%%%%%%%%%%%%%%%%%%%%%%%%%%%%%%%%%%%%%%%%%%%%%%%%%%
% Origin
\texttt{$\langle$\textbf{\benchs{}},\benchi{},\benchi{}$\rangle$}

% Behavior (Expected, Observed)
&
   \begin{tabular}{@{}l@{}}
      8000 L2 Snoop Hits,\\
      8000 External Snoop Requests\\
   \end{tabular}
&
   \begin{tabular}{@{}l@{}}
      8000 L2 Snoop Hits,\\
      \colorbox{pink}{\textbf{8000 L2 Snoop Pushes}},\\
      8000 External Snoop Requests,\\
      \colorbox{pink}{\textbf{8000 SINTs}}
   \end{tabular}
\\
\hline

%%%% S/\benchi{}/I %%%%%%%%%%%%%%%%%%%%%%%%%%%%%%%%%%%%%%%%%%%%%%%%%%%%%%%%%%%%%%%%%%%%%
% Origin
\texttt{$\langle$\textbf{\benchx{}},\benchi{},\benchi{}$\rangle$}

% Behavior (Expected, Observed)
&
   \begin{tabular}{@{}l@{}}
      8000 L2 Snoop Hits,\\
      8000 External Snoop Requests
   \end{tabular}
&
   \begin{tabular}{@{}l@{}}
      8000 L2 Snoop Hits,\\
      8000 External Snoop Requests
   \end{tabular}
\\
\hline
\end{tabular}

\end{center}
\caption{Focusing on the Activities of \benchs{} and \benchx{}}
\label{fig:unexpected_behaviors}
\end{figure}

Figure~\ref{fig:unexpected_behaviors} shows the non-null values returned by the
performance monitors when loading a dataset of 8000 unique memory elements from
the \benchi{} state on a cluster, depending on the coherence state for these
memory elements on another cluster. The upper table indicates what is recorded
on the cluster performing the \texttt{load} operations and the bottom table
corresponds to what is recorded on the farthest cluster, hence the symmetry
between the two tables of the \textit{Origin} state column and that of operation
performed. The activity analysis results for
\texttt{$\langle$\benchi{},\benchi{},\benchi{}$\rangle$} are given as a reference
point. Indeed, as it was so far assumed that \benchx{} and \benchs{} are
equivalent to an \texttt{S} state, the results ought to have been the same in
all the lines of this first table.

The first surprising observation is that the amount of L2D accesses is
consistently twice the expected number. While it is odd and I have not found the
reason behind this discrepancy, I chose to not consider it to be a sufficient
contradiction of hypothetical protocol, as this factor holds true for every
single one of the benchmarks.

Much more interesting is the hint of truly unexpected activity found in the
upper table, where the \texttt{$\langle$\benchi{},\benchi{},\benchs{}$\rangle$}
benchmark is performed using less CPU cycles than the others. Looking at what
happens on the bottom table for the symmetrical line, it can be seen that the
cache holding the memory elements in the \benchs{} is actually providing them to
the demanding cluster. This is in clear contradiction with the hypothetical
protocol. Furthermore, this is not simply a case of having different activity
for what should be the \texttt{S} state: the
\texttt{$\langle$\benchx{},\benchi{},\benchi{}$\rangle$} line of the bottom
table indicates that no such thing is happening for memory elements in the
\benchx{} state. This proves that \benchs{} and \benchx{} are, in fact, two
completely separate stable states. In other words, the NXP QorIQ T4240
architecture does not actually use MESI as its coherence protocol.
