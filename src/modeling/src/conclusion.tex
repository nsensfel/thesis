\section{Conclusion}
This chapter presented a way to model a multi-core architecture with a focus on
cache coherence mechanisms.

The use of a network of timed automata proved to be effective in allowing a
readable and modular description of the system. Indeed, each component can be
modeled independently, with only synchronizations with other components needing
to be accounted for.

The model presented in this chapter can be configured to match the user's
architecture through easily configurable parameters. The addition of an
arbitrary number of cores with their caches and FIFOs is also supported and is
made easy by UPPAAL's automaton template features.

The coherence protocol is defined outside of the model, solely using the
notions from Chapter~\ref{cha:cache_coherence}. This makes it possible for the
user to change the resulting model's cache coherence protocol without having to
understand the way it is modeled. Furthermore, switching from
one protocol to another can be done automatically using the provided tool,
making the whole process very approachable.

Once the platform has been modeled, formal analyses can be performed. The next
chapter not only explores the results that can be obtained with the model as
described here, but also introduces definitions for the interference generated
by cache coherence and explains how the model can be used to expose them.
