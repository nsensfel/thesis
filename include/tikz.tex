\usepackage{tikz}

\usetikzlibrary{arrows}
\usetikzlibrary{automata}
\usetikzlibrary{backgrounds}
\usetikzlibrary{calc}
\usetikzlibrary{decorations.pathreplacing}
\usetikzlibrary{matrix}
\usetikzlibrary{patterns}
\usetikzlibrary{positioning}
\usetikzlibrary{shapes,shapes.arrows,shapes.callouts,shapes.symbols}

\usetikzlibrary{shapes,arrows,backgrounds,positioning,calc, automata, backgrounds,fit, arrows, decorations.pathreplacing}
\usetikzlibrary{shadows, backgrounds,graphs, trees, patterns}

\definecolor{dblue}{rgb}{.11,.4,.7}

\tikzset{block/.style={draw,rectangle,text centered, text width=#1}}
\tikzset{block/.default={draw,rectangle}}
\tikzset{arc/.style={-latex}}
\tikzset{back/.style={latex-}}
\tikzset{connect/.style={draw, circle, fill=black, scale=.3}}
\tikzset{br/.style n args={2}{below=#1 of #2, anchor= south west}}
\tikzset{bl/.style n args={2}{below=#1 of #2, anchor= south east}}
\tikzset{ar/.style n args={2}{above=#1 of #2.south east}}
\tikzset{al/.style n args={2}{above=#1 of #2.south west}}
\tikzset{bolded/.style={line width= 2 pt}}

\tikzset{com/.style={draw, ellipse, thick, text centered, text width=3cm}}


\tikzset{
    max width/.style args={#1}{
        execute at begin node={\begin{varwidth}{#1}},
        execute at end node={\end{varwidth}}
    }
}
\tikzset{%
    set width/.style args={#1}{%
      text width=#1,
        execute at begin node={\begin{varwidth}{#1}},
        execute at end node={\end{varwidth}}
    }
}


% Trying out some way to model comm. in UPPAAL
\usetikzlibrary{matrix}
\usetikzlibrary{arrows}
\pgfdeclareshape{datastore}{
  \inheritsavedanchors[from=rectangle]
  \inheritanchorborder[from=rectangle]
  \inheritanchor[from=rectangle]{center}
  \inheritanchor[from=rectangle]{base}
  \inheritanchor[from=rectangle]{north}
  \inheritanchor[from=rectangle]{north east}
  \inheritanchor[from=rectangle]{east}
  \inheritanchor[from=rectangle]{south east}
  \inheritanchor[from=rectangle]{south}
  \inheritanchor[from=rectangle]{south west}
  \inheritanchor[from=rectangle]{west}
  \inheritanchor[from=rectangle]{north west}
% Stuff below breaks compilation, because of course it does...
%  \backgroundpath{
%    %  store lower right in xa/ya and upper right in xb/yb
%    \southwest \pgf@xa=\pgf@x \pgf@ya=\pgf@y
%    \northeast \pgf@xb=\pgf@x \pgf@yb=\pgf@y
%    \pgfpathmoveto{\pgfpoint{\pgf@xa}{\pgf@ya}}
%    \pgfpathlineto{\pgfpoint{\pgf@xb}{\pgf@ya}}
%    \pgfpathmoveto{\pgfpoint{\pgf@xa}{\pgf@yb}}
%    \pgfpathlineto{\pgfpoint{\pgf@xb}{\pgf@yb}}
% }
}
