The approaches presented in Section~\ref{sec:rel_work:handling_it:through_restrictions}
are improvements on the predictability of the system, but at the cost of its
performance. Those from Section~\ref{sec:rel_work:handling_it:through_hardware} do not
generally influence the system's performance in order to make it more
predictable, but, by their very nature, are not compatible with the need to use
existing, commercial off-the-shelf architectures. Lastly,
Section~\ref{sec:rel_work:handling_it:accepting_it} presented existing strategies to
determine when the interference inherent to the use of a multi-core processor
does not lead to unacceptable execution times. This last section is what this
thesis contributes to. Indeed, none of the existing approaches tackle cache
coherence, and are instead focused the sharing of caches. The strategy chosen in
this thesis for determining the impact of interference caused by cache coherence
on the system's performance is the use of formal methods.
