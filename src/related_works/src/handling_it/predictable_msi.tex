\subsection{Predictable MSI}
\label{sec:related_works:pmsi}
While not exactly a hardware modification per se, the use of a coherence
protocol designed for its predictability is not likely to be available on COTS
hardware, which is why the work presented in \cite{conf/rtas/HassanKP17} is
included in this section.

\cite{conf/rtas/HassanKP17} exposes the sources of unpredictability in the MSI
protocol, and proposes a solution for each one of them. These solutions require
some parts of the hardware connected to cache coherence mechanisms to be
predictable. Combined with, PMSI (Predictable MSI), the coherence protocol they
introduce, these solutions are indicated as sufficient to allow calculation of
the worst case delays that may be incurred by each instruction.

Translated to the formalism from Chapter~\ref{cha:cache_coherence}, the
restrictions imposed on the hardware are as follow:
\begin{enumerate}
\item
   Access to the interconnect is regulated by time slots. Caches may only send
   queries to the interconnect during their assigned time slot.
   \label{enum:pmsi_time_slot}
\item
   For a same memory element, the coherence manager sends replies to queries in
   the order of their arrival.
\item
   Caches reply to queries in the order of their arrival. This is equivalent
   to imposing $\cachedataoutfun{}$ to be a FIFO queue.
\item
   A write request to a cache for a memory element that it does not currently
   hold with read-and-write permissions can only occur during that cache's
   time slot.
   \label{enum:pmsi_write_time}
\item
   Writing to a memory element not currently held with read-and-write
   permissions requires all other caches' pending queries for that memory
   element to be processed first. Basically, the write is stalled if there
   are any queries in $\cachequeryinfun{}$ for that memory element.
   \label{enum:pmsi_ordering}
\item
   Caches are predictible in the ordering of their processing of instructions
   and external queries.
\end{enumerate}

\begin{figure}
\centering
\begin{tabular}{|l||c|c|c|c|c||c|c|c|}
 \hline

 \textbf{State}
 & \multicolumn{3}{c|}{\textbf{Core Request}}
 & \begin{tabular}{@{}c@{}}
      \textbf{Own}\\
      \textbf{Query}
   \end{tabular}
 & \begin{tabular}{@{}c@{}}
      \textbf{Data}\\
      \textbf{Reply}
   \end{tabular}
 & \multicolumn{3}{c|}{\textbf{Received Queries}}
 \\

%%%%%%%%%%%%%%%%%%%%%%%%%%%%%%%%%%%%%%%%%%%%%%%%%%%%%%%%%%%%%%%%%%%%%%%%%%%%%%%%
 & \loadinstr{} & \storeinstr{} & \evictinstr{}
 &
 &
 & \getsquery{} & \getmquery{} & \putmquery{}
 \\
 \hline

%%%%%%%%%%%%%%%%%%%%%%%%%%%%%%%%%%%%%%%%%%%%%%%%%%%%%%%%%%%%%%%%%%%%%%%%%%%%%%%%
 \texttt{I}

 % Load/Store/Evict
 &
   \begin{tabular}{@{}c@{}}
      \sendqueryact{\getsquery{}}\\
      \setstateact{\texttt{IS\textsuperscript{D}}}
   \end{tabular}
 &
   \begin{tabular}{@{}c@{}}
      \sendqueryact{\getmquery{}}\\
      \setstateact{\texttt{IM\textsuperscript{D}}}
   \end{tabular}
 & \disablecell{}

 % Seeing Own Query
 & \disablecell{}

 % Data Reply
 & \disablecell{}

 % GetS/GetM/Upgr/PutM
 & \noaction{}
 & \noaction{}
 & \noaction{}
 \\
 \hline

%%%%%%%%%%%%%%%%%%%%%%%%%%%%%%%%%%%%%%%%%%%%%%%%%%%%%%%%%%%%%%%%%%%%%%%%%%%%%%%%
 \texttt{S}

 % Load/Store/Evict
 & \hitact{}
 &
   \begin{tabular}{@{}c@{}}
      \sendqueryact{\getmquery{}}\\
      \setstateact{\texttt{SM\textsuperscript{W}}}
   \end{tabular}
 & \setstateact{\texttt{I}}

 % Seeing Own Query
 & \disablecell{}

 % Data Reply
 & \disablecell{}

 % GetS/GetM/PutM
 & \noaction{}
 & \setstateact{\texttt{I}}
 & \disablecell{}
 \\
 \hline


%%%%%%%%%%%%%%%%%%%%%%%%%%%%%%%%%%%%%%%%%%%%%%%%%%%%%%%%%%%%%%%%%%%%%%%%%%%%%%%%
 \texttt{M}

 % Load/Store/Evict
 & \hitact{}
 & \hitact{}
 & \begin{tabular}{@{}c@{}}
      \sendqueryact{\putmquery{}}\\
      \setstateact{\texttt{MI\textsuperscript{WB}}}
   \end{tabular}

 % Seeing Own Query
 & \disablecell{}

 % Data Reply
 & \disablecell{}

 % GetS/GetM/PutM
 &
   \begin{tabular}{@{}c@{}}
      \sendqueryact{\putmquery{}}\\
      \setstateact{\texttt{MS\textsuperscript{WB}}}
   \end{tabular}
 &
   \begin{tabular}{@{}c@{}}
      \sendqueryact{\putmquery{}}\\
      \setstateact{\texttt{MI\textsuperscript{WB}}}
   \end{tabular}
 & \disablecell{}

 \\
 \hline

%%%%%%%%%%%%%%%%%%%%%%%%%%%%%%%%%%%%%%%%%%%%%%%%%%%%%%%%%%%%%%%%%%%%%%%%%%%%%%%%
 \texttt{IS\textsuperscript{D}}

 % Load/Store/Evict
 & \stallact{}
 & \stallact{}
 & \stallact{}

 % Seeing Own Query
 & \disablecell{}

 % Data Reply
 & \setstateact{\texttt{S}}

 % GetS/GetM/PutM
 & \noaction{}
 & \setstateact{\texttt{IS\textsuperscript{D}I}}
 & \disablecell{}
 \\
 \hline

%%%%%%%%%%%%%%%%%%%%%%%%%%%%%%%%%%%%%%%%%%%%%%%%%%%%%%%%%%%%%%%%%%%%%%%%%%%%%%%%
 \texttt{IM\textsuperscript{D}}

 % Load/Store/Evict
 & \stallact{}
 & \stallact{}
 & \stallact{}

 % Seeing Own Query
 & \disablecell{}

 % Data Reply
 & \setstateact{\texttt{M}}

 % GetS/GetM/PutM
 & \noaction{}
 & \setstateact{\texttt{IM\textsuperscript{D}I}}
 & \disablecell{}
 \\
 \hline

%%%%%%%%%%%%%%%%%%%%%%%%%%%%%%%%%%%%%%%%%%%%%%%%%%%%%%%%%%%%%%%%%%%%%%%%%%%%%%%%
 \texttt{SM\textsuperscript{W}}

 % Load/Store/Evict
 & \stallact{}
 & \stallact{}
 & \stallact{}

 % Seeing Own Query
 &
   \begin{tabular}{@{}c@{}}
      \senddataact{\memorytarget{}}{\simpledata{}}\\
      \setstateact{\texttt{I}}
   \end{tabular}

 % Data Reply
 & \disablecell{}

 % GetS/GetM/PutM
 & \setstateact{\texttt{IM\textsuperscript{D}S}}
 & \setstateact{\texttt{IM\textsuperscript{D}I}}
 & \disablecell{}
 \\
 \hline

%%%%%%%%%%%%%%%%%%%%%%%%%%%%%%%%%%%%%%%%%%%%%%%%%%%%%%%%%%%%%%%%%%%%%%%%%%%%%%%%
 \texttt{MI\textsuperscript{WB}}

 % Load/Store/Evict
 & \hitact{}
 & \hitact{}
 & \stallact{}

 % Seeing Own Query
 &
   \begin{tabular}{@{}c@{}}
      \senddataact{\memorytarget{}}{\simpledata{}}\\
      \setstateact{\texttt{I}}
   \end{tabular}

 % Data Reply
 & \disablecell{}

 % GetS/GetM/PutM
 & \noaction{}
 & \noaction{}
 & \disablecell{}
 \\
 \hline

%%%%%%%%%%%%%%%%%%%%%%%%%%%%%%%%%%%%%%%%%%%%%%%%%%%%%%%%%%%%%%%%%%%%%%%%%%%%%%%%
 \texttt{MS\textsuperscript{WB}}

 % Load/Store/Evict
 & \hitact{}
 & \hitact{}
 & \setstateact{\texttt{MI\textsuperscript{WB}}}

 % Seeing Own Query
 &
   \begin{tabular}{@{}c@{}}
      \senddataact{\memorytarget{}}{\simpledata{}}\\
      \setstateact{\texttt{S}}
   \end{tabular}

 % Data Reply
 & \disablecell{}

 % GetS/GetM/PutM
 & \noaction{}
 & \setstateact{\texttt{MI\textsuperscript{WB}}}
 & \disablecell{}
 \\
 \hline

%%%%%%%%%%%%%%%%%%%%%%%%%%%%%%%%%%%%%%%%%%%%%%%%%%%%%%%%%%%%%%%%%%%%%%%%%%%%%%%%
 \texttt{IM\textsuperscript{D}I}

 % Load/Store/Evict
 & \stallact{}
 & \stallact{}
 & \stallact{}

 % Seeing Own Query
 & \disablecell{}

 % Data Reply
 &
   \begin{tabular}{@{}c@{}}
      \sendqueryact{\putmquery{}}\\
      \storeinstr{} \hitact{}\\
      \setstateact{\texttt{MI}\textsuperscript{WB}}
   \end{tabular}

 % GetS/GetM/PutM
 & \noaction{}
 & \noaction{}
 & \disablecell{}
 \\
 \hline

%%%%%%%%%%%%%%%%%%%%%%%%%%%%%%%%%%%%%%%%%%%%%%%%%%%%%%%%%%%%%%%%%%%%%%%%%%%%%%%%
 \texttt{IM\textsuperscript{D}S}

 % Load/Store/Evict
 & \stallact{}
 & \stallact{}
 & \stallact{}

 % Seeing Own Query
 & \disablecell{}

 % Data Reply
 &
   \begin{tabular}{@{}c@{}}
      \sendqueryact{\putmquery{}}\\
      \storeinstr{} \hitact{}\\
      \setstateact{\texttt{MS\textsuperscript{WB}}}
   \end{tabular}

 % GetS/GetM/PutM
 & \noaction{}
 & \setstateact{\texttt{IM\textsuperscript{D}I}}
 & \disablecell{}
 \\
 \hline

 %%%%%%%%%%%%%%%%%%%%%%%%%%%%%%%%%%%%%%%%%%%%%%%%%%%%%%%%%%%%%%%%%%%%%%%%%%%%%%%%
  \texttt{IS\textsuperscript{D}I}

  % Load/Store/Evict
  & \stallact{}
  & \stallact{}
  & \stallact{}

  % Seeing Own Query
  & \disablecell{}

  % Data Reply
  &
    \begin{tabular}{@{}c@{}}
       \loadinstr{} \hitact{}\\
       \setstateact{\texttt{I}}
    \end{tabular}

  % GetS/GetM/PutM
  & \noaction{}
  & \noaction{}
  & \disablecell{}
  \\
 \hline
\end{tabular}

\caption{Split-Transaction PMSI Automaton for Cache Controllers}
\label{fig:pmsi_cc_table}
\end{figure}

\begin{figure}
\centering
\begin{tabular}{|l||c|c|c|c|c|}
 \hline

 \textbf{State}
 & \multicolumn{4}{c|}{\textbf{Received Queries}}
 & \textbf{Data Reply}
 \\

%%%%%%%%%%%%%%%%%%%%%%%%%%%%%%%%%%%%%%%%%%%%%%%%%%%%%%%%%%%%%%%%%%%%%%%%%%%%%%%%
 & \getsquery{}
 & \getmquery{}
 & \putmquery{} (Owner)
%   \begin{tabular}{@{}c@{}}
%      \putmquery{}\\
%      (Owner)
%   \end{tabular}
 & \putmquery{} (Other)
%   \begin{tabular}{@{}c@{}}
%      \putmquery{}\\
%   \end{tabular}
 & \simpledata{}
 \\
 \hline

%%%%%%%%%%%%%%%%%%%%%%%%%%%%%%%%%%%%%%%%%%%%%%%%%%%%%%%%%%%%%%%%%%%%%%%%%%%%%%%%
 \texttt{I}

 % GetS
 & \senddataact{\sendertarget}{\simpledata{}}

 % GetM
 &
   \begin{tabular}{@{}c@{}}
      \senddataact{\sendertarget}{\simpledata{}}\\
      \storeowneract{}\\
      \setstateact{\texttt{M}}
   \end{tabular}

 % PutM (owner)
 & \noaction{}

 % PutM (other)
 & \noaction{}

 % Data
 & \disablecell{}
 \\
 \hline

%%%%%%%%%%%%%%%%%%%%%%%%%%%%%%%%%%%%%%%%%%%%%%%%%%%%%%%%%%%%%%%%%%%%%%%%%%%%%%%%
 \texttt{I\textsuperscript{D}}

 % GetS
 & \stallact{}

 % GetM
 & \stallact{}

 % PutM (owner)
 & \disablecell{}

 % PutM (other)
 & \noaction{}

 % Data
 & \setstateact{\texttt{I}}
 \\
 \hline

%%%%%%%%%%%%%%%%%%%%%%%%%%%%%%%%%%%%%%%%%%%%%%%%%%%%%%%%%%%%%%%%%%%%%%%%%%%%%%%%
 \texttt{I\textsuperscript{B}}

 % GetS
 &
   \begin{tabular}{@{}c@{}}
      \resetowneract{}\\
      \senddataact{\sendertarget}{\simpledata{}}\\
      \setstateact{\texttt{I}}
   \end{tabular}

 % GetM
 &
   \begin{tabular}{@{}c@{}}
      \storeowneract{}\\
      \senddataact{\sendertarget}{\simpledata{}}\\
      \setstateact{\texttt{M}}
   \end{tabular}

 % PutM (owner)
 & \begin{tabular}{@{}c@{}}
      \resetowneract{}\\
      \setstateact{\texttt{I}}
   \end{tabular}

 % PutM (other)
 & \noaction{}

 % Data
 & \disablecell{}
 \\
 \hline

%%%%%%%%%%%%%%%%%%%%%%%%%%%%%%%%%%%%%%%%%%%%%%%%%%%%%%%%%%%%%%%%%%%%%%%%%%%%%%%%
 \texttt{M}

 % GetS
 & \stallact{}

 % GetM
 & \stallact{}

 % PutM (owner)
 & \begin{tabular}{@{}c@{}}
      \resetowneract{}\\
      \setstateact{\texttt{I\textsuperscript{D}}}
   \end{tabular}

 % PutM (other)
 & \noaction{}

 % Data
 & \setstateact{\texttt{I\textsuperscript{B}}}
 \\
 \hline

\end{tabular}%

\caption{Possible Split-Transaction PMSI Automaton for the Coherence Manager}
\label{fig:pmsi_mgr_table}
\end{figure}

Adapting \texttt{PMSI} to fit Chapter~\ref{cha:cache_coherence} is made easy by
the fact that they also based their description on the notations introduced in
\cite{Sorin:2011:PMC:2028905}. Figure~\ref{fig:pmsi_cc_table} shows how the
cache controller behaves for each memory element.  This figure is almost
identical to the one found in the paper presenting \texttt{PMSI}, with the
following modifications:
\begin{itemize}
\item
   A single \textit{Own Query} event is used instead of having
   one per type of query. This makes the table more compact, as there is never
   any state in which multiple categories of query have a different behavior
   when it comes to the handling observing their own queries on the
   interconnect.
\item
   The \texttt{Upg} query type, used to move from \texttt{S} to \texttt{M}, has
   been merged into \getmquery{}, as the differentiation did not appear to have
   any relevance to the protocol's behavior.
\item
   Emission of a \putmquery{} has been added as an action when receiving data
   in either \texttt{IM\textsuperscript{D}I} and
   \texttt{IM\textsuperscript{D}S}, as it would otherwise be impossible to exit
   the state they lead to. The need for this has been confirmed during exchanges
   with the authors.
\end{itemize}
\cite{conf/rtas/HassanKP17} does
not include a table for the coherence manager, which is instead simply described
as servicing the oldest pending query for a memory element every time it
receives data. Figure~\ref{fig:pmsi_mgr_table} shows a coherence manager that
would implement this behavior.

By comparison to the \texttt{MSI} protocol described in
Section~\ref{sec:split_msi}, \texttt{PMSI} features much fewer transient states.
This is because the restrictions imposed on the system have removed:
\begin{itemize}
\item The possibility of receiving a reply to a query not yet handled:
\texttt{IS\textsuperscript{BD}}, \texttt{IS\textsuperscript{B}},
\texttt{IM\textsuperscript{BD}}, \texttt{IM\textsuperscript{B}},
\texttt{SM\textsuperscript{BD}}, and \texttt{SM\textsuperscript{B}}.
\item Cache to cache data messages, which merge cases that were separated
because of whether data was sent only to another cache or also to the main
memory:
\texttt{IM\textsuperscript{D}SI}, \texttt{SM\textsuperscript{D}SI}.
\item Sending data as a reaction to seeing another cache's query:
\texttt{II\textsuperscript{B}} (now undistinct from
\texttt{MI\textsuperscript{B}}, which is now \texttt{MI\textsuperscript{WB}}).
\end{itemize}
In \texttt{PMSI}, receiving data for a memory element currently held in a cache
that may have modified it (e.g.~\texttt{M} state) always requires waiting for
that cache's time slot in the TDMA, as it will not perform a write back without
first broadcasting its own query indicating that it is about to do so. The cache
then sends a data message to the system's main memory, which only then is able
to reply to the original demand. While this is clearly penalizing in terms of
performance, it does lessen the variability in time required for acquisition of
data.

In conclusion, \cite{conf/rtas/HassanKP17} addresses the issue of the
predictability of memory access latency by replacing the cache coherence
protocol and performing some other hardware modifications in order to have
a generally slower but more easily computed and less fluctuating memory
access time. No modification of the software is required.

\stopallthesefloats
