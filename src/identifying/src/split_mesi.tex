\section{Hypothetical Split-Transaction MESI Protocol}
\label{sec:identification:mesi}
\begin{figure}[htb!]
\begin{center}
\resizebox{\textwidth}{!}{%
\begin{tabular}{|l||c|c|c||c||c|c||c|c|c|}
 \hline
 \multicolumn{10}{|c|}{\textbf{Cache Controller}}\\

 \hline

 \multirow{2}{*}{\textbf{State}}
 & \multicolumn{3}{c||}{\textbf{Core Request}}
 & \begin{tabular}{@{}c@{}}
      \textbf{Interconnect}\\
      \textbf{Access}
   \end{tabular}
 & \multicolumn{2}{c||}{\textbf{Data Reply}}
 & \multicolumn{3}{c|}{\textbf{Received Queries}}
 \\

%%%%%%%%%%%%%%%%%%%%%%%%%%%%%%%%%%%%%%%%%%%%%%%%%%%%%%%%%%%%%%%%%%%%%%%%%%%%%%%%
 & \loadinstr{} & \storeinstr{} & \evictinstr{}
 &
 & \simpledata{} & \exclusivedata{}
 & \getsquery{} & \getmquery{} & \putmquery{}
 \\
 \hline

%%%%%%%%%%%%%%%%%%%%%%%%%%%%%%%%%%%%%%%%%%%%%%%%%%%%%%%%%%%%%%%%%%%%%%%%%%%%%%%%
 \texttt{I}

 % Load/Store/Evict
 & \sendqueryact{\getsquery{}}, \setstateact{\texttt{IS\textsuperscript{BD}}}
 & \sendqueryact{\getmquery{}}, \setstateact{\texttt{IM\textsuperscript{BD}}}
 & \hitact{}

 % Seeing Own Query
 & \disablecell{}

 % Data Reply
 & \disablecell{}
 & \disablecell{}

 % GetS/GetM/PutM
 & \noaction{}
 & \noaction{}
 & \noaction{}
 \\
 \hline

%%%%%%%%%%%%%%%%%%%%%%%%%%%%%%%%%%%%%%%%%%%%%%%%%%%%%%%%%%%%%%%%%%%%%%%%%%%%%%%%
 \texttt{IS\textsuperscript{BD}}

 % Load/Store/Evict
 & \stallact{}
 & \stallact{}
 & \stallact{}

 % Seeing Own Query
 & \setstateact{\texttt{IEoS\textsuperscript{D}}}

 % Data Reply
 & \setstateact{\texttt{IS\textsuperscript{B}}}
 & \setstateact{\texttt{IE\textsuperscript{B}}}

 % GetS/GetM/PutM
 & \noaction{}
 & \noaction{}
 & \noaction{}
 \\
 \hline

%%%%%%%%%%%%%%%%%%%%%%%%%%%%%%%%%%%%%%%%%%%%%%%%%%%%%%%%%%%%%%%%%%%%%%%%%%%%%%%%
 \texttt{IS\textsuperscript{B}}

 % Load/Store/Evict
 & \stallact{}
 & \stallact{}
 & \stallact{}

 % Seeing Own Query
 & \setstateact{\texttt{S}}

 % Data Reply
 & \disablecell{}
 & \disablecell{}

 % GetS/GetM/PutM
 & \noaction{}
 & \noaction{}
 & \disablecell{}
 \\
 \hline

%%%%%%%%%%%%%%%%%%%%%%%%%%%%%%%%%%%%%%%%%%%%%%%%%%%%%%%%%%%%%%%%%%%%%%%%%%%%%%%%
 \texttt{IS\textsuperscript{D}}

 % Load/Store/Evict
 & \stallact{}
 & \stallact{}
 & \stallact{}

 % Seeing Own Query
 & \disablecell{}

 % Data Reply
 & \begin{tabular}{@{}c@{}}
      \resetreplytoact{},\\
      \setstateact{\texttt{S}}
   \end{tabular}

 & \begin{tabular}{@{}c@{}}
      \senddataact{\replytotarget{}}{\simpledata{}},\\
      \senddataact{\memorytarget{}}{\texttt{no-data}},\\
      \resetreplytoact{}, \setstateact{\texttt{S}}
   \end{tabular}

 % GetS/GetM/PutM
 & \noaction{}
 & \setstateact{\texttt{IS\textsuperscript{D}I}}
 & \disablecell{}
 \\
 \hline

%%%%%%%%%%%%%%%%%%%%%%%%%%%%%%%%%%%%%%%%%%%%%%%%%%%%%%%%%%%%%%%%%%%%%%%%%%%%%%%%
 \texttt{IEoS\textsuperscript{D}}

 % Load/Store/Evict
 & \stallact{}
 & \stallact{}
 & \stallact{}

 % Seeing Own Query
 & \disablecell{}

 % Data Reply
 & \setstateact{\texttt{S}}
 & \setstateact{\texttt{E}}

 % GetS/GetM/PutM
 & \storereplytoact{},
   \setstateact{\texttt{IS\textsuperscript{D}}}
 &
   \storereplytoact{},
   \setstateact{\texttt{IS\textsuperscript{D}I}}
 & \disablecell{}
 \\
 \hline

%%%%%%%%%%%%%%%%%%%%%%%%%%%%%%%%%%%%%%%%%%%%%%%%%%%%%%%%%%%%%%%%%%%%%%%%%%%%%%%%
 \texttt{IS\textsuperscript{D}I}

 % Load/Store/Evict
 & \stallact{}
 & \stallact{}
 & \stallact{}

 % Seeing Own Query
 & \disablecell{}

 % Data Reply
 &
   \begin{tabular}{@{}c@{}}
      load hit,\\
      \resetreplytoact{},\\
      \setstateact{\texttt{I}}
   \end{tabular}
 &
   \begin{tabular}{@{}c@{}}
      load hit,\\
      \resetreplytoact{},\\
      \senddataact{\replytotarget{}}{\simpledata{}},\\
      \senddataact{\memorytarget{}}{\texttt{no-data}},\\
      \setstateact{\texttt{I}}
   \end{tabular}

 % GetS/GetM/PutM
 & \noaction{}
 & \noaction{}
 & \disablecell{}
 \\
 \hline

%%%%%%%%%%%%%%%%%%%%%%%%%%%%%%%%%%%%%%%%%%%%%%%%%%%%%%%%%%%%%%%%%%%%%%%%%%%%%%%%
 \texttt{IM\textsuperscript{BD}}

 % Load/Store/Evict
 & \stallact{}
 & \stallact{}
 & \stallact{}

 % Seeing Own Query
 & \setstateact{\texttt{IM\textsuperscript{D}}}

 % Data Reply
 & \setstateact{\texttt{IM\textsuperscript{B}}}
 & \disablecell{}

 % GetS/GetM/PutM
 & \noaction{}
 & \noaction{}
 & \noaction{}
 \\
 \hline

%%%%%%%%%%%%%%%%%%%%%%%%%%%%%%%%%%%%%%%%%%%%%%%%%%%%%%%%%%%%%%%%%%%%%%%%%%%%%%%%
 \texttt{IM\textsuperscript{B}}

 % Load/Store/Evict
 & \stallact{}
 & \stallact{}
 & \stallact{}

 % Seeing Own Query
 & \setstateact{\texttt{M}}

 % Data Reply
 & \disablecell{}
 & \disablecell{}

 % GetS/GetM/PutM
 & \noaction{}
 & \noaction{}
 & \noaction{}
 \\
 \hline

%%%%%%%%%%%%%%%%%%%%%%%%%%%%%%%%%%%%%%%%%%%%%%%%%%%%%%%%%%%%%%%%%%%%%%%%%%%%%%%%
 \texttt{IM\textsuperscript{D}}

 % Load/Store/Evict
 & \stallact{}
 & \stallact{}
 & \stallact{}

 % Seeing Own Query
 & \disablecell{}

 % Data Reply
 & \setstateact{\texttt{M}}
 & \disablecell{}

 % GetS/GetM/PutM
 & \storereplytoact{}, \setstateact{\texttt{IM\textsuperscript{D}S}}
 & \storereplytoact{}, \setstateact{\texttt{IM\textsuperscript{D}I}}
 & \disablecell{}
 \\
 \hline

%%%%%%%%%%%%%%%%%%%%%%%%%%%%%%%%%%%%%%%%%%%%%%%%%%%%%%%%%%%%%%%%%%%%%%%%%%%%%%%%
 \texttt{IM\textsuperscript{D}I}

 % Load/Store/Evict
 & \stallact{}
 & \stallact{}
 & \stallact{}

 % Seeing Own Query
 & \disablecell{}

 % Data Reply
 &
   \begin{tabular}{@{}c@{}}
      store hit,\\
      \senddataact{\replytotarget{}}{\simpledata{}},\\
      \resetreplytoact{}, \setstateact{\texttt{I}}
   \end{tabular}
 & \disablecell{}

 % GetS/GetM/PutM
 & \noaction{}
 & \noaction{}
 & \disablecell{}
 \\
 \hline

%%%%%%%%%%%%%%%%%%%%%%%%%%%%%%%%%%%%%%%%%%%%%%%%%%%%%%%%%%%%%%%%%%%%%%%%%%%%%%%%
 \texttt{IM\textsuperscript{D}S}

 % Load/Store/Evict
 & \stallact{}
 & \stallact{}
 & \stallact{}

 % Seeing Own Query
 & \disablecell{}

 % Data Reply
 &
   \begin{tabular}{@{}c@{}}
      store hit,\\
      \senddataact{\replytotarget{}}{\simpledata{}},\\
      \senddataact{\memorytarget{}}{\simpledata{}},\\
      \resetreplytoact{}, \setstateact{\texttt{S}}
   \end{tabular}
 & \disablecell{}

 % GetS/GetM/PutM
 & \noaction{}
 & \setstateact{\texttt{IM\textsuperscript{D}SI}}
 & \disablecell{}
 \\
 \hline

%%%%%%%%%%%%%%%%%%%%%%%%%%%%%%%%%%%%%%%%%%%%%%%%%%%%%%%%%%%%%%%%%%%%%%%%%%%%%%%%
 \texttt{IM\textsuperscript{D}SI}

 % Load/Store/Evict
 & \stallact{}
 & \stallact{}
 & \stallact{}

 % Seeing Own Query
 & \disablecell{}

 % Data Reply
 & \begin{tabular}{@{}c@{}}
      store hit,\\
      \senddataact{\replytotarget{}}{\simpledata{}},\\
      \senddataact{\memorytarget{}}{\simpledata{}},\\
      \resetreplytoact{}, \setstateact{\texttt{I}}
   \end{tabular}
 & \disablecell{}

 % GetS/GetM/PutM
 & \noaction{}
 & \noaction{}
 & \disablecell{}
 \\
 \hline

%%%%%%%%%%%%%%%%%%%%%%%%%%%%%%%%%%%%%%%%%%%%%%%%%%%%%%%%%%%%%%%%%%%%%%%%%%%%%%%%
 \texttt{S}

 % Load/Store/Evict
 & \hitact{}
 & \sendqueryact{\getmquery{}}, \setstateact{\texttt{SM\textsuperscript{BD}}}
 & \hitact{}, \setstateact{\texttt{I}}

 % Seeing Own Query
 & \disablecell{}

 % Data Reply
 & \disablecell{}
 & \disablecell{}

 % GetS/GetM/PutM
 & \noaction{}
 & \setstateact{\texttt{I}}
 & \disablecell{}
 \\
 \hline

%%%%%%%%%%%%%%%%%%%%%%%%%%%%%%%%%%%%%%%%%%%%%%%%%%%%%%%%%%%%%%%%%%%%%%%%%%%%%%%%
 \texttt{SM\textsuperscript{BD}}

 % Load/Store/Evict
 & \hitact{}
 & \stallact{}
 & \stallact{}

 % Seeing Own Query
 & \setstateact{\texttt{SM\textsuperscript{D}}}

 % Data Reply
 & \setstateact{\texttt{SM\textsuperscript{B}}}
 & \disablecell{}

 % GetS/GetM/PutM
 & \noaction{}
 & \setstateact{\texttt{IM\textsuperscript{BD}}}
 & \disablecell{}
 \\
 \hline

%%%%%%%%%%%%%%%%%%%%%%%%%%%%%%%%%%%%%%%%%%%%%%%%%%%%%%%%%%%%%%%%%%%%%%%%%%%%%%%%
 \texttt{SM\textsuperscript{B}}

 % Load/Store/Evict
 & \hitact{}
 & \stallact{}
 & \stallact{}

 % Seeing Own Query
 & \setstateact{\texttt{M}}

 % Data Reply
 & \disablecell{}
 & \disablecell{}

 % GetS/GetM/PutM
 & \noaction{}
 & \setstateact{\texttt{IM\textsuperscript{B}}}
 & \disablecell{}
 \\
 \hline


%%%%%%%%%%%%%%%%%%%%%%%%%%%%%%%%%%%%%%%%%%%%%%%%%%%%%%%%%%%%%%%%%%%%%%%%%%%%%%%%
 \texttt{SM\textsuperscript{D}}

 % Load/Store/Evict
 & \hitact{}
 & \stallact{}
 & \stallact{}

 % Seeing Own Query
 & \disablecell{}

 % Data Reply
 & store hit, \setstateact{\texttt{M}}
 & \disablecell{}

 % GetS/GetM/PutM
 & \storereplytoact{}, \setstateact{\texttt{SM\textsuperscript{D}S}}
 & \storereplytoact{}, \setstateact{\texttt{SM\textsuperscript{D}I}}
 & \disablecell{}
 \\
 \hline

%%%%%%%%%%%%%%%%%%%%%%%%%%%%%%%%%%%%%%%%%%%%%%%%%%%%%%%%%%%%%%%%%%%%%%%%%%%%%%%%
 \texttt{SM\textsuperscript{D}I}

 % Load/Store/Evict
 & \hitact{}
 & \stallact{}
 & \stallact{}

 % Seeing Own Query
 & \disablecell{}

 % Data Reply
 & \begin{tabular}{@{}c@{}}
      store hit,\\
      \senddataact{\replytotarget{}}{\simpledata{}},\\
      \resetreplytoact{}, \setstateact{\texttt{I}}
   \end{tabular}
 & \disablecell{}

 % GetS/GetM/PutM
 & \noaction{}
 & \noaction{}
 & \disablecell{}
 \\
 \hline

%%%%%%%%%%%%%%%%%%%%%%%%%%%%%%%%%%%%%%%%%%%%%%%%%%%%%%%%%%%%%%%%%%%%%%%%%%%%%%%%
 \texttt{SM\textsuperscript{D}S}

 % Load/Store/Evict
 & \hitact{}
 & \stallact{}
 & \stallact{}

 % Seeing Own Query
 & \disablecell{}

 % Data Reply
 & \begin{tabular}{@{}c@{}}
      store hit,\\
      \senddataact{\replytotarget{}}{\simpledata{}},\\
      \senddataact{\memorytarget{}}{\simpledata{}},\\
      \resetreplytoact{}, \setstateact{\texttt{S}}
   \end{tabular}
 & \disablecell{}

 % GetS/GetM/PutM
 & \noaction{}
 & \setstateact{\texttt{SM\textsuperscript{D}SI}}
 & \disablecell{}
 \\
 \hline

%%%%%%%%%%%%%%%%%%%%%%%%%%%%%%%%%%%%%%%%%%%%%%%%%%%%%%%%%%%%%%%%%%%%%%%%%%%%%%%%
 \texttt{SM\textsuperscript{D}SI}

 % Load/Store/Evict
 & \hitact{}
 & \stallact{}
 & \stallact{}

 % Seeing Own Query
 & \disablecell{}

 % Data Reply
 & \begin{tabular}{@{}c@{}}
      store hit,\\
      \senddataact{\replytotarget{}}{\simpledata{}},\\
      \senddataact{\memorytarget{}}{\simpledata{}},\\
      \resetreplytoact{}, \setstateact{\texttt{I}}
   \end{tabular}
 & \disablecell{}

 % GetS/GetM/PutM
 & \noaction{}
 & \noaction{}
 & \disablecell{}
 \\
 \hline

%%%%%%%%%%%%%%%%%%%%%%%%%%%%%%%%%%%%%%%%%%%%%%%%%%%%%%%%%%%%%%%%%%%%%%%%%%%%%%%%
 \texttt{M}

 % Load/Store/Evict
 & \hitact{}
 & \hitact{}
 & \sendqueryact{\putmquery{}}, \setstateact{\texttt{MI\textsuperscript{B}}}

 % Seeing Own Query
 & \disablecell{}

 % Data Reply
 & \disablecell{}
 & \disablecell{}

 % GetS/GetM/PutM
 &
   \begin{tabular}{@{}c@{}}
      \senddataact{\memorytarget{}}{\simpledata{}},\\
      \senddataact{\sendertarget{}}{\simpledata{}}, \setstateact{\texttt{S}}
   \end{tabular}
 & \senddataact{\sendertarget{}}{\simpledata{}}, \setstateact{\texttt{I}}
 & \disablecell{}
 \\
 \hline

%%%%%%%%%%%%%%%%%%%%%%%%%%%%%%%%%%%%%%%%%%%%%%%%%%%%%%%%%%%%%%%%%%%%%%%%%%%%%%%%
 \texttt{MI\textsuperscript{B}}

 % Load/Store/Evict
 & \hitact{}
 & \hitact{}
 & \stallact{}

 % Seeing Own Query
 & \senddataact{\memorytarget{}}{\simpledata{}}, \setstateact{\texttt{I}}

 % Data Reply
 & \disablecell{}
 & \disablecell{}

 % GetS/GetM/PutM
 &
   \begin{tabular}{@{}c@{}}
      \senddataact{\memorytarget{}}{\simpledata{}},\\
      \senddataact{\sendertarget{}}{\simpledata{}},
      \setstateact{\texttt{II\textsuperscript{B}}}
   \end{tabular}
 & \senddataact{\sendertarget{}}{\simpledata{}},
   \setstateact{\texttt{II\textsuperscript{B}}}
 & \disablecell{}
 \\
 \hline

%%%%%%%%%%%%%%%%%%%%%%%%%%%%%%%%%%%%%%%%%%%%%%%%%%%%%%%%%%%%%%%%%%%%%%%%%%%%%%%%
 \texttt{II\textsuperscript{B}}

 % Load/Store/Evict
 & \stallact{}
 & \stallact{}
 & \stallact{}

 % Seeing Own Query
 & \setstateact{\texttt{I}}

 % Data Reply
 & \disablecell{}
 & \disablecell{}

 % GetS/GetM/PutM
 & \noaction{}
 & \noaction{}
 & \noaction{}
 \\
 \hline

%%%%%%%%%%%%%%%%%%%%%%%%%%%%%%%%%%%%%%%%%%%%%%%%%%%%%%%%%%%%%%%%%%%%%%%%%%%%%%%%
 \texttt{E}

 % Load/Store/Evict
 & \hitact{}
 & \hitact{}, \setstateact{\texttt{M}}
 &
   \begin{tabular}{@{}c@{}}
      \sendqueryact{\putmquery{}}\\
      \setstateact{\texttt{EI\textsuperscript{B}}}
   \end{tabular}

 % Seeing Own Query
 & \disablecell{}

 % Data Reply
 & \disablecell{}
 & \disablecell{}

 % GetS/GetM/PutM
 &
   \begin{tabular}{@{}c@{}}
      \senddataact{\memorytarget{}}{\texttt{no-data}},\\
      \senddataact{\sendertarget{}}{\simpledata{}}, \setstateact{\texttt{S}}
   \end{tabular}
 & \senddataact{\sendertarget{}}{\simpledata{}}, \setstateact{\texttt{I}}
 & \disablecell{}
 \\
 \hline

%%%%%%%%%%%%%%%%%%%%%%%%%%%%%%%%%%%%%%%%%%%%%%%%%%%%%%%%%%%%%%%%%%%%%%%%%%%%%%%%
 \texttt{IE\textsuperscript{B}}

 % Load/Store/Evict
 & \stallact{}
 & \stallact{}
 & \stallact{}

 % Seeing Own Query
 & \setstateact{\texttt{E}}

 % Data Reply
 & \disablecell{}
 & \disablecell{}

 % GetS/GetM/PutM
 & \noaction{}
 & \noaction{}
 & \noaction{}
 \\
 \hline

%%%%%%%%%%%%%%%%%%%%%%%%%%%%%%%%%%%%%%%%%%%%%%%%%%%%%%%%%%%%%%%%%%%%%%%%%%%%%%%%
 \texttt{EI\textsuperscript{B}}

 % Load/Store/Evict
 & \hitact{}
 & \hitact{}, \setstateact{\texttt{MI\textsuperscript{B}}}
 & \stallact{}

 % Seeing Own Query
 & \senddataact{\memorytarget{}}{\texttt{no-data}}, \setstateact{\texttt{I}}

 % Data Reply
 & \disablecell{}
 & \disablecell{}

 % GetS/GetM/PutM
 &
   \begin{tabular}{@{}c@{}}
      \senddataact{\memorytarget{}}{\texttt{no-data}},\\
      \senddataact{\sendertarget{}}{\simpledata{}},
      \setstateact{\texttt{II\textsuperscript{B}}}
   \end{tabular}
 & \senddataact{\sendertarget{}}{\simpledata{}},
   \setstateact{\texttt{II\textsuperscript{B}}}
 & \disablecell{}
 \\
 \hline

% \multicolumn{1}{l|}{}
% & \multicolumn{6}{c||}{\textbf{Handling Requests}}
% & \multicolumn{3}{c|}{\textbf{Handling Queries}}
% \\
% \cline{2-10}
\end{tabular}

}
\end{center}
\caption{Description of the cache controller for the MESI protocol}
\label{fig:mesi_cc_table}
\end{figure}

\begin{figure}[htb!]
\begin{center}
\resizebox{\textwidth}{!}{%
\begin{tabular}{|l||c|c|c|c||c|c|}
 \hline

 \multicolumn{7}{|c|}{\textbf{Coherence Manager}}\\

 \hline

 \multirow{2}{*}{\textbf{State}}
 & \multicolumn{4}{c||}{\textbf{Received Queries}}
 & \multicolumn{2}{c|}{\textbf{Data Reply}}
 \\

%%%%%%%%%%%%%%%%%%%%%%%%%%%%%%%%%%%%%%%%%%%%%%%%%%%%%%%%%%%%%%%%%%%%%%%%%%%%%%%%
 & \texttt{GetS}
 & \texttt{GetM}
 &
   \begin{tabular}{@{}c@{}}
      \texttt{PutM}\\
      (Owner)
   \end{tabular}
 &
   \begin{tabular}{@{}c@{}}
      \texttt{PutM}\\
      (Other)
   \end{tabular}
 & \texttt{data}
 & \texttt{no-data}
 \\
 \hline

%%%%%%%%%%%%%%%%%%%%%%%%%%%%%%%%%%%%%%%%%%%%%%%%%%%%%%%%%%%%%%%%%%%%%%%%%%%%%%%%
 \texttt{I}

 % GetS
 & \texttt{read},
   \senddataact{\texttt{s}}{\texttt{data-e}},
   \storeowneract{},
   \setstateact{\texttt{M}}

 % GetM
 & \senddataact{\texttt{s}}{\texttt{data}},
   \storeowneract{},
   \setstateact{\texttt{M}}

 % PutM (owner)
 & \disablecell{}

 % PutM (other)
 & -

 % Data & Data-Exclusive
 & \disablecell{}
 & \disablecell{}
 \\
 \hline

%%%%%%%%%%%%%%%%%%%%%%%%%%%%%%%%%%%%%%%%%%%%%%%%%%%%%%%%%%%%%%%%%%%%%%%%%%%%%%%%
 \texttt{M}

 % GetS
 & \resetowneract{}, \setstateact{\texttt{S\textsuperscript{D}}}

 % GetM
 & \storeowneract{}

 % PutM (owner)
 & \resetowneract{}, \setstateact{\texttt{I\textsuperscript{D}}}

 % PutM (other)
 & -

 % Data & Data-Exclusive
 & write, \setstateact{\texttt{IoS\textsuperscript{B}}}
 & \setstateact{\texttt{IoS\textsuperscript{B}}}
 \\
 \hline

%%%%%%%%%%%%%%%%%%%%%%%%%%%%%%%%%%%%%%%%%%%%%%%%%%%%%%%%%%%%%%%%%%%%%%%%%%%%%%%%
 \texttt{I\textsuperscript{D}}

 % GetS
 & \texttt{stall}

 % GetM
 & \texttt{stall}

 % PutM (owner)
 & \texttt{stall}

 % PutM (other)
 & -

 % Data & Data-Exclusive
 & write, resume, \setstateact{\texttt{I}}
 & resume, \setstateact{\texttt{I}}
 \\
 \hline

%%%%%%%%%%%%%%%%%%%%%%%%%%%%%%%%%%%%%%%%%%%%%%%%%%%%%%%%%%%%%%%%%%%%%%%%%%%%%%%%
 \texttt{S\textsuperscript{D}}

 % GetS
 & \texttt{stall}

 % GetM
 & \texttt{stall}

 % PutM (owner)
 & \texttt{stall}

 % PutM (other)
 & -

 % Data & Data-Exclusive
 & write, resume, \setstateact{\texttt{S}}
 & resume, \setstateact{\texttt{S}}
 \\
 \hline

%%%%%%%%%%%%%%%%%%%%%%%%%%%%%%%%%%%%%%%%%%%%%%%%%%%%%%%%%%%%%%%%%%%%%%%%%%%%%%%%
 \texttt{IoS\textsuperscript{B}}

 % GetS
 & \resetowneract{}, \setstateact{\texttt{S}}

 % GetM
 & \storeowneract{}

 % PutM (owner)
 & \resetowneract{}, \setstateact{\texttt{I}}

 % PutM (other)
 & -

 % Data & Data-Exclusive
 & \disablecell{}
 & \disablecell{}
 \\
 \hline

%%%%%%%%%%%%%%%%%%%%%%%%%%%%%%%%%%%%%%%%%%%%%%%%%%%%%%%%%%%%%%%%%%%%%%%%%%%%%%%%
 \texttt{S}

 % GetS
 &
   \texttt{read},
   \senddataact{\texttt{s}}{\texttt{data}}

 % GetM
 & \senddataact{\texttt{s}}{\texttt{data}}, \storeowneract{},
   \setstateact{\texttt{M}}

 % PutM (owner)
 & \disablecell{}

 % PutM (other)
 & -

 % Data & Data-Exclusive
 & \disablecell{}
 & \disablecell{}
 \\
 \hline
\end{tabular}


}
\end{center}
\caption{Description of the coherence manager for the MESI protocol}
\label{fig:mesi_cmgr_table}
\end{figure}

Following the information available in documentation of the architecture
(\cite{T4240} and \cite{e6500}), the first step is to define an ambiguity-free
description of the MESI protocol (Figures~\ref{fig:mesi_cc_table} and
\ref{fig:mesi_cmgr_table}).  Introduced in
\cite{Papamarcos:1984:LCS:773453.808204}, the MESI protocol adds a fourth stable
state, \textit{Exclusive}, which indicates that not only does the cache
controller have read-only permissions, but also that no other cache currently
holds any permission to access the memory element. This allows the cache
controller to upgrade to read-and-write permissions without having to perform a
costly communication. Just as it is used to keep track of whether a cache holds
a read-and-write copy of a memory element in the MSI protocol, this definition
of the MESI protocol uses the coherence manager to detect when a cache can be
said to be the sole owner of a memory element.

This version of the MESI protocol uses three types of data replies:
\texttt{data}, \texttt{data-e}, and \texttt{no-data}. \texttt{data} indicates
that the value associated with the memory element is sent.  By sending a
\texttt{no-data} reply, cache controllers can indicate to the coherence manager
that the memory element has been discarded (its value is not part of the
reply). The coherence manager can send \texttt{data-e} replies, which are
equivalent to \texttt{data}, with the added information that the recipient is
its sole owner.

Here are some examples of remarkable behaviors exhibited by this definition of
the MESI protocol.

\begin{example}[\textbf{Reaching \texttt{E}}]
\label{ex:reaching_e}
To hold a memory element in the \texttt{E} state, a cache must be the only one
to have a copy of that memory element. The caches rely on the coherence manager
to know when it is the case. The coherence manager uses its \texttt{I} state to
mark memory elements that are sure not to be in any caches.  Thus, if no cache
controllers hold the memory element and the coherence manager is in \texttt{I},
whenever a core \texttt{loads} the data it becomes \texttt{E} in its cache,
receiving the \texttt{data} message leads it to the \texttt{E} state instead of
the \texttt{S} state.

It is important to notice that it is not easy for the coherence manager to detect whether a cache controller is the sole owner.
Indeed, the
coherence manager is not always able to know that all caches have evicted their
copy of a memory element: in Figure~\ref{fig:mesi_cc_table}, the cache
controller's table indicates that an eviction from \texttt{S} does not lead to
any message.
The only way for the coherence
manager to return to the \texttt{I} state is for a cache to evict its copy
of a memory element in either the \texttt{E} or \texttt{M} state without another
cache asking for a copy.
\end{example}

\begin{example}[\textbf{Sharing from \texttt{E}}]
\label{ex:sharing_from_e}
From the coherence manager's point of view, there is no difference between a
cache controller owning a memory element in the \texttt{E} state and one in the
\texttt{M} state. Thus, if there is a cache owning a copy of a memory element in
the \texttt{E} state, the coherence manager will assume that this cache may have
modified the value and that the main memory no longer holds the correct value.
As a result, the cache holding the \texttt{E}xclusive copy of the memory element
will transfer it to any other cache that asks for it. If this is caused by
another cache demanding a read-only copy (\texttt{GetS}), the coherence manager
will expect an update on the value of the memory element. This update can come
in two forms: either the cache that exclusively held the memory element made a
modification (in which case it would have moved to the \texttt{M}odified state)
and sends a \texttt{data} message, or it has not and it sends a \texttt{no-data} message.
\end{example}
