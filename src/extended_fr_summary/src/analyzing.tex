\section{Analyser la coh\'erence de cache}
\label{fr:sec:analyze}
Cette section présente les analyses permettant d'utiliser le modèle de
la section précédente pour mettre en évidence les interférences liées
à la cohérence de cache. Une partie de ces travaux ont été publiés
dans \cite{ecrts19}.

\begin{figure}[hbt]
  \centering
  \resizebox{.7\linewidth}{!}{

\begin{tikzpicture}[->,>=stealth',shorten >=1pt,auto,  semithick, node distance=0.5cm]
\tikzstyle{section} = [state, rectangle, fill=gray!20]
\tikzstyle{result} = [state, rectangle, rounded corners]
\tikzstyle{auxresult} = [state,rectangle,draw=white]
\tikzstyle{input} = [state,rectangle, draw=blue, thick, fill=blue!20, align=center, rounded corners, minimum height=2em]
\tikzstyle{interresult} = [state, rectangle,dashed, rounded corners]

\node[section] (WCETANA)
   {
      \begin{tabular}{@{}c@{}}
         WCET\\
         Analysis\\
         (Section~\ref{sec:analysis:wcet})
      \end{tabular}
   };


\node[auxresult] (SLOWFACTOR) [above right=of WCETANA,yshift=-20pt]
   {
      \begin{tabular}{@{}c@{}}
         Slowdown\\
         Factors
      \end{tabular}
   };

\node[result] (PROGWCET) [below right=of WCETANA,yshift=20pt]
   {
      \begin{tabular}{@{}c@{}}
         WCET of\\
         Programs
      \end{tabular}
   };


\node[interresult,node distance=2cm] (NOSHAREDWCET) [below =of WCETANA]
   {
      \begin{tabular}{@{}c@{}}
         Without Shared\\
         Variables
      \end{tabular}
   };

\node[result] (IMPACTWCET) [xshift=-1cm,yshift=-0.5cm, below right =of NOSHAREDWCET]
   {
      \begin{tabular}{@{}c@{}}
         Impact of\\
         Interference\\
         on WCET
      \end{tabular}
   };

\node[section,node distance=4cm] (HITMISSANA) [right =of WCETANA]
   {
      \begin{tabular}{@{}c@{}}
         Hit \& Miss\\
         Analysis\\
         (Section~\ref{sec:analysis:hit_and_miss})
      \end{tabular}
   };

\node[input] (PARAMEDMODEL) [above =of HITMISSANA]
   {
      \begin{tabular}{@{}c@{}}
         Instantiated\\
         Model
      \end{tabular}
   };

\node[result] (INSTRACCU) [below left =of HITMISSANA,xshift=1cm]
   {
      \begin{tabular}{@{}c@{}}
         Instruction\\
         Accuracy
      \end{tabular}
   };

\node[auxresult] (MEMEACCU) [below right =of HITMISSANA,xshift=-1cm]
   {
      \begin{tabular}{@{}c@{}}
         Mem. Element\\
         Accuracy
      \end{tabular}
   };


\node[section] (INTERCAT) [right =of HITMISSANA,xshift=2.5cm]
   {
      \begin{tabular}{@{}c@{}}
         Interference\\
         Categorization\\
         (Section~\ref{sec:analysis:exposing_interference})
      \end{tabular}
   };

\node[input] (PROTOCOL) [above =of INTERCAT]
   {
      \begin{tabular}{@{}c@{}}
         Cache Coherence\\
         Protocol
      \end{tabular}
   };

\node[interresult] (ANOTPROTO) [below =of INTERCAT]
   {
      \begin{tabular}{@{}c@{}}
         Annotated\\
         Protocol
      \end{tabular}
   };

\node[section] (INSTRIMPA) [below =of ANOTPROTO,xshift=-0.5cm]
   {
      \begin{tabular}{@{}c@{}}
         Instruction Impact\\
         Analysis\\
         (Section~\ref{sec:analysis:missing_link})
      \end{tabular}
   };

\node[result] (RELINSTRINTER) [below =of INSTRIMPA]
   {
      \begin{tabular}{@{}c@{}}
         Relation Between\\
         Instruction \&\\
         Interference
      \end{tabular}
   };

\path[draw,->] (PARAMEDMODEL) -| (WCETANA);
\path[draw,->] ([yshift=-30pt]WCETANA) -- (PROGWCET);
\path[draw,->] ([yshift=30pt]WCETANA) -- (SLOWFACTOR);

\path[draw,->] (PARAMEDMODEL) -- (HITMISSANA);
\path[draw,->] (HITMISSANA) -- (INSTRACCU);
\path[draw,->] (HITMISSANA) -- (MEMEACCU);

\path[draw,->] (PROTOCOL) -- (INTERCAT);
\path[draw,->] (INTERCAT) -- (ANOTPROTO);
\path[draw,->] (ANOTPROTO) -- (INSTRIMPA);
\path[draw,->] (PARAMEDMODEL) -| ([xshift=10pt]INSTRIMPA.north west);
\path[draw,->] (INSTRIMPA) -- (RELINSTRINTER);

\path[draw,->] ([xshift=-20pt]WCETANA) -- ([xshift=-20pt]NOSHAREDWCET);
\path[draw,->] (PROGWCET) -- (IMPACTWCET);
\path[draw,->] (NOSHAREDWCET) -- (IMPACTWCET);
\end{tikzpicture}
}
\caption{Vue d'ensemble des analyses proposées}
\label{fr:fig:analysis:summary}
\end{figure}

Le modèle présenté dans la section précédente comporte des paramètres (par exemple la durée d'un accès à la mémoire) qui doivent être instanciés via une campagne de benchmarks afin de mener les analyses qui sont décrites ici. Ce processus d'instanciation est en dehors du périmètre de la thèse.
La Figure~\ref{fr:fig:analysis:summary} fournit une vue d'ensemble des analyses
proposées. Les rectangles avec fond gris correspondent aux analyses et ceux sans
arrière plan sont les résultats principaux. Les éléments sans bordure sont des
résultats accessoires. Les bordures en pointillés indiquent les résultats
intermédiaires, qui ne sont pas censés être utiles en eux-mêmes.

% Pour que les analyses présentées dans cette section fournissent des informations
% intéressantes à l'applicant, le modèle doit être instancié afin de correspondre
% à l'architecture cible. Cela signifie que les paramètres listés dans
% l'Annexe~\ref{app:model_parameters} ont été définis suite
% à une campagne de benchmarks mesurant la performance.

\begin{figure}[hbt]
\begin{minipage}[c]{0.55\linewidth}
\includegraphics[width=\linewidth]{\chapterdirectory/figure/analysis/model_overview.pdf}
\caption{Aperçu de l'exemple de modèle instancié}
\label{fr:fig:analysis:model}
\end{minipage}
\begin{minipage}[t]{0.2\linewidth}
\begin{enumerate}
  \setlength{\itemsep}{0pt}%
   \setlength{\parskip}{0pt}%
\item \texttt{store 1}
\item \texttt{store 2}
\item \texttt{load 1}
\item \texttt{store 1}
\item \texttt{load 3}
\item \texttt{store 2}
\item \texttt{load 1}
\item \texttt{store 1}
\item \texttt{load 2}
\item \texttt{store 2}
\item \texttt{end}
\end{enumerate}
\caption{Modèle de programme pour Cœur 1}
\label{fr:fig:analysis:demo_prog1}
\end{minipage}
\begin{minipage}[t]{0.2\linewidth}
\begin{enumerate}
  \setlength{\itemsep}{0pt}%
   \setlength{\parskip}{0pt}%
\item \texttt{store 1}
\item \texttt{store 3}
\item \texttt{load 3}
\item \texttt{store 2}
\item \texttt{load 1}
\item \texttt{store 2}
\item \texttt{load 3}
\item \texttt{store 1}
\item \texttt{load 2}
\item \texttt{store 3}
\item \texttt{end}
\end{enumerate}
\caption{Modèle de programme pour Cœur 2}
\label{fr:fig:analysis:demo_prog2}
\end{minipage}
\end{figure}

Le modèle instancié utilisé pour l'illustration des analyses dans cette section
est représenté dans la Figure~\ref{fr:fig:analysis:model}. Celui-ci utilise un
protocole MESI dont les détails sont fournis dans la version complète de la
thèse (voir Figure~\ref{fig:mesi_cc_table}). Le programme
utilisé par chaque cœur est indiqué par les
Figures~\ref{fr:fig:analysis:demo_prog1} et \ref{fr:fig:analysis:demo_prog2}.
Ces modèles instanciés comportent toujours de l'indéterminisme, qui
représente ce que l'applicant ne connaît pas ou ne peut pas contrôler
sur l'architecture.  Par conséquent, le modèle instancié admet
toujours plusieurs traces d'exécution possibles. Les analyses étant
basées sur l'algorithme de model checking d'UPPAAL, toutes ces traces
sont explorées.


\subsection{Analyse de l'impact sur le temps d'exécution}
\label{fr:sec:analysis:wcet}
Cette première analyse regarde les effets de l'interférence sur le temps
d'exécution total des programmes. Notons cependant qu'il s'agit du temps d'exécution calculé sur le modèle
qui peut s'éloigner du temps d'exécution sur l'architecture réelle, notamment en raison des abstractions faites dans la modélisation ou d'une instanciation du modèle insuffisamment précise. 
Les valeurs obtenues sont tout de même intéressantes à comparer avec
d'autres analyses de temps sur un modèle similaire, par exemple pour
le calcul de facteurs de ralentissement (voir
Définition~\ref{fr:def:slowdownfactor}). Par exemple, on peut obtenir
la proportion du temps d'exécution causée par la cohérence de cache en
comparant le modèle avec une version adaptée du même modèle dans laquelle
aucune variable n'est partagée. Bien que cette version adaptée ne soit
probablement pas réaliste, son temps d'exécution peut être utilisé
comme point de référence.

\begin{definition}[Impact de la cohérence de cache sur le temps d'exécution]
Soit $W_{s}$ le temps d'exécution d'un programme sur le modèle instancié, et
$W_{p}$ son temps d'exécution sur une instance du même modèle dans laquelle
toutes les variables ont été rendues privées. La part de $W_{s}$ correspondant
aux mécanismes de cohérence de cache peut être obtenue avec l'équation suivante:
$T_{cc} = W_{s} - W_{p}$
\end{definition}

Pour obtenir les valeurs de $W_{s}$ et $W_{p}$ en utilisant UPPAAL, on emploie
la vérification de modèle. En effet, ces valeurs correspondent au maximum d'une
horloge mesurant le temps d'exécution du cœur étudié. Il est donc possible de la
récupérer avec une formule semblable à\\
\lstinline!sup{not Core1.Terminated}: Core1.runtime!, qui retourne le maximum
pour l'horloge \lstinline!Core1.runtime! dans l'ensemble des états du système
pour lesquels l'automate \lstinline!Core1! n'est pas dans la localité
\lstinline!Terminated!.

\begin{example}[Exemple de mesures de temps d'exécution]
\begin{figure}[hbt]
\centering
\begin{tabular}{|c|c|c|c|c|}
\cline{2-5}
\multicolumn{1}{c|}{}
      & $W_s$ & $W_p$ & $T_{cc}$ & Isolation \\
\hline
Cœur 1 & 2652 & 1102 & 1550 & 702\\
\hline
Cœur 2 & 2452 & 1452 & 1000 & 904\\
\hline
\end{tabular}
\caption{Exemple d'analyse de temps d'exécution}
\label{fr:fig:analysis:wcet_calc}
\end{figure}
La Figure~\ref{fr:fig:analysis:wcet_calc} indique la valeur maximale du temps
d'exécution pour chaque cœur, avec différentes versions de l'exemple du modèle
instancié.
$W_s$ correspond à celle du modèle instancié original (et donc avec les
variables partagées). $W_p$ correspond à celle d'un modèle dans lequel toutes les
variables ont été rendues privées. Ici, on incrémente les adresses d'éléments
mémoires du programme sur le Cœur 2 par $3$ pour éviter tout partage. $T_{cc}$
correspond à la part du temps d'exécution de $W_s$ prise par les mécanismes de
cohérence cache. Enfin, pour montrer un exemple de facteur de ralentissement, on
étudie aussi le cas où chaque programme est exécuté en isolation.
Les résultats de l'analyse sont donc:
\begin{itemize}
    \setlength{\itemsep}{0pt}%
   \setlength{\parskip}{0pt}%
\item
   Cœur 1 souffre d'un facteur de ralentissement de $2652/702 = 3,77$ quand son
   programme tourne en même temps que celui de Cœur 2, comparé à son exécution
   en isolation.
\item
   Cœur 2 souffre d'un facteur de ralentissement de $2452/904 = 2,71$ quand son
   programme tourne en même temps que celui de Cœur 1, comparé à son exécution
   en isolation.
\item
   Exécuter les deux programmes en isolation l'un après l'autre aurait un temps
   d'exécution maximum de $702 + 904 = 1606$ unités de temps.
\item
   Exécuter les deux programmes en parallèle a un temps d'exécution maximal de
   $\max(2652, 2452) = 2652$.
\item
   Exécuter les deux programmes en parallèle mais sans variables partagées a un
   temps d'exécution maximal de $\max(1102, 1452) = 1452$.
\item
   Approximativement $(1550/2652)*100=58,44 \%$  du temps d'exécution de
   Cœur 1 est causé par l'interférence liée à la cohérence de caches.
\item
   Approximativement $(1000/2452)*100=40,78\%$ du temps d'exécution de
   Cœur 2 est causé par l'interférence liée à la cohérence de caches.
\end{itemize}
\end{example}


\subsection{Catégorisation des accès au cache}
\label{fr:sec:cat_cache_access}
Une interférence peut empêcher une instruction de récupérer une valeur dans le cache (parce que son contenu a été modifié récemment à cause d'une autre instruction). C'est ce qu'on appelle un \emph{cache-miss}.
Les analyses de cette
section s'intéressent à la catégorisation des instructions en fonction des \emph{cache-miss} observés. C'est une technique utilisée dans la littérature (voir
Section~\ref{fr:sec:handling_it}).
L'objectif de cette analyse est donc de classer chaque instruction en
tant que \textit{always-hit} (cache toujours prêt),
\textit{always-miss} (cache jamais prêt) ou \textit{uncategorized} (le
cache peut être prêt ou ne pas l'être selon l'exécution). Pour cela,
on utilise l'opérateur de logique temporelle \agop{} afin de s'assurer
que pour toute trace d'exécution du modèle, l'instruction donnée:
\begin{itemize}
    \setlength{\itemsep}{0pt}%
   \setlength{\parskip}{0pt}%
\item Est résolue par le cache immédiatement, dans quel cas l'instruction est
classée \textit{always-hit}.
\item N'est pas résolue par le cache immédiatement et donc
classée \textit{always-miss}.
\item Sinon, on la classe comme \textit{uncategorized}.
\end{itemize}

\begin{example}[Catégorisation des accès au cache]
\label{fr:ex:analysis:instr_chara}
\begin{figure}[hbt]
\begin{center}
\begin{subfigure}[t]{0.45\textwidth}
\centering
\begin{enumerate}
    \setlength{\itemsep}{0pt}%
   \setlength{\parskip}{0pt}%
\item \texttt{store 1} est classé \texttt{AM}.
\item \texttt{store 2} est classé \texttt{AM}.
\item \texttt{load 1} est classé \texttt{UN}.
\item \texttt{store 1} est classé \texttt{UN}.
\item \texttt{load 3} est classé \texttt{AM}.
\item \texttt{store 2} est classé \texttt{AM}.
\item \texttt{load 1} est classé \texttt{AH}.
\item \texttt{store 1} est classé \texttt{AM}.
\item \texttt{load 2} est classé \texttt{UN}.
\item \texttt{store 2} est classé \texttt{UN}.
\end{enumerate}
\caption{Catégorisation pour le Cœur 1}
\label{fr:fig:analysis:demo_chara_prog1}
\end{subfigure}
\begin{subfigure}[t]{0.45\textwidth}
\centering
\begin{enumerate}
    \setlength{\itemsep}{0pt}%
   \setlength{\parskip}{0pt}%
\item \texttt{store 1} est classé \texttt{AM}.
\item \texttt{store 3} est classé \texttt{AM}.
\item \texttt{load 3} est classé \texttt{AH}.
\item \texttt{store 2} est classé \texttt{AM}.
\item \texttt{load 1} est classé \texttt{AM}.
\item \texttt{store 2} est classé \texttt{UN}.
\item \texttt{load 3} est classé \texttt{AH}.
\item \texttt{store 1} est classé \texttt{AM}.
\item \texttt{load 2} est classé \texttt{UN}.
\item \texttt{store 3} est classé \texttt{AM}.
\end{enumerate}
\caption{Catégorisation pour le Cœur 2}
\label{fr:fig:analysis:demo_chara_prog2}
\end{subfigure}
\end{center}
\caption{Exemple de catégorisation des accès cache}
\label{fr:fig:analysis:demo_chara_progs}
\end{figure}
La Figure~\ref{fr:fig:analysis:demo_chara_progs} montre le résultat de la
catégorisation de chaque instruction de notre exemple.
\texttt{AH} correspond à \textit{always-hit}, \texttt{AM} correspond
à \textit{always-miss} et \texttt{UN} correspond
à \textit{uncategorized}.
\end{example}

Cette catégorisation des instructions permet de déterminer quelles
instructions font varier le temps d'exécution. Cependant, certaines de ces
variations ne sont pas liées à la cohérence de cache. Pour
pouvoir répondre au besoin de la certification, il va donc être
nécessaire d'analyser l'interférence elle-même.

\subsection{Catégorisation de l'interférence}
\label{fr:sec:analysis:exposing_interference}
Pour pouvoir comprendre la cause et les effets des interférences générées par
la cohérence de cache, nous proposons de les classifier en fonction de leurs effets
sur le cache affecté. Cette section présente les trois catégories
d'interférences proposées dans cette thèse.

\begin{definition}[Interférence mineure]
On considère qu'un cache subit une interférence mineure lorsqu'il reçoit une
demande en provenance d'un autre cache sans que cette demande ne nécessite
d'actions de sa part. En effet, le cache a alors pris le temps de traiter une
demande sans que cela n'ait eu d'utilité.
\end{definition}

\begin{example}[Interférence mineure]
Le protocole MSI simplifié de la Section~\ref{fr:sec:cache_coherence_msi}
présente des interférences mineures lorsqu'un cache tenant un élément mémoire
dans l'état \texttt{I} reçoit des demandes, ou qu'il reçoit un \texttt{GetS}
(demande d'accès en lecture) alors qu'il a l'élément en \texttt{S}.
\end{example}

\begin{definition}[Interférence de rétrogradation]
On considère qu'un cache subit une interférence de rétrogradation lorsqu'il
perd les permissions d'écriture sur un élément mémoire suite à la demande d'un
autre cache.
\end{definition}

\begin{example}[Interférence de rétrogradation]
Le protocole MSI simplifié de la Section~\ref{fr:sec:cache_coherence_msi}
présente une interférence de rétrogradation lorsqu'un cache tenant un élément
mémoire dans l'état \texttt{M} reçoit une demande \texttt{GetS} de la part d'un
autre cache. En effet, il passe alors à l'état \texttt{S} et perd ses
permissions d'écriture.
\end{example}

\begin{definition}[Interférence d'expulsion]
On considère qu'un cache subit une interférence d'expulsion lorsqu'il perd
toutes ses permissions sur un élément mémoire suite à la demande d'un autre
cache.
\end{definition}

\begin{example}[Interférence d'expulsion]
Le protocole MSI simplifié de la Section~\ref{fr:sec:cache_coherence_msi}
présente des interférences d'expulsion losrqu''un cache tenant un élément mémoire
dans l'état \texttt{S} ou \texttt{M} reçoit une demande \texttt{GetM} (demande
d'accès en lecture et écriture) de la part d'un autre cache. En effet, il passe
alors à l'état \texttt{I} et perd toutes ses permissions.
\end{example}

\subsection{Révéler les interférences liées à la cohérence de cache}
\label{fr:sec:analysis:missing_link}
Pour révéler toutes les interférences causées par la cohérence de
cache en tenant compte des effets catégorisés, il suffit de
faire en sorte que le modèle détecte les cas où l'interférence a
affecté une instruction. Cela permet alors d'utiliser les outils de
model checking d'UPPAAL pour déterminer quelle instruction cause
une interférence sur quelle autre instruction.
Ainsi, cette section identifie les interférences liées à la cohérence de cache en
définissant deux ensembles finis $S_A$ et $S_E$ composés de triplets
$\langle I_o, E, I_t \rangle$, tels que $I_o$ correspond à l'instruction
causant une interférence de type $E$ sur l'instruction $I_T$. L'ensemble
$S_A$ contient les triplets pour lesquels l'interférence est certaine de se
produire alors que $S_E$ correspond à ceux pour lesquels au moins une exécution
présente l'interférence en question.
Combiné avec les résultats de l'analyse de la
Section~\ref{fr:sec:cat_cache_access}, ceci fournit à l'applicant à la fois les
causes et les effets des interférences liées à la cohérence de cache dans le
système modélisé.

\begin{figure}[hbt]
\centering
\begin{tikzpicture}[->,>=stealth',shorten >=1pt,auto,  semithick]
\tikzstyle{instr} = [state,rectangle,  node distance = 1cm and 10cm]
\tikzstyle{cacheline} = [state,rectangle, node distance = 0.5cm and 3cm]

\node[instr] (C1L0)                    {1. \storeinstr{} 1 (AM)};
\node[instr] (C1L1) [below = of C1L0]  {2. \storeinstr{} 2 (AM)};
\node[instr] (C1L2) [below = of C1L1]  {3. \loadinstr{}  1 (UN)};
\node[instr] (C1L3) [below = of C1L2]  {4. \storeinstr{} 1 (UN)};
\node[instr] (C1L4) [below = of C1L3]  {5. \loadinstr{}  3 (AM)};
\node[instr] (C1L5) [below = of C1L4]  {6. \storeinstr{} 2 (AM)};
\node[instr] (C1L6) [below = of C1L5]  {7. \loadinstr{}  1 (AH)};
\node[instr] (C1L7) [below = of C1L6]  {8. \storeinstr{} 1 (AM)};
\node[instr] (C1L8) [below = of C1L7]  {9. \loadinstr{}  2 (UN)};
\node[instr] (C1L9) [below = of C1L8]  {10. \storeinstr{} 2 (UN)};

\node[instr] (C2L0) [right = of C1L0]  {1. \storeinstr{} 1 (AM)};
\node[instr] (C2L1) [below = of C2L0]  {2. \storeinstr{} 3 (AM)};
\node[instr] (C2L2) [below = of C2L1]  {3. \loadinstr{}  3 (AH)};
\node[instr] (C2L3) [below = of C2L2]  {4. \storeinstr{} 2 (AM)};
\node[instr] (C2L4) [below = of C2L3]  {5. \loadinstr{}  1 (AM)};
\node[instr] (C2L5) [below = of C2L4]  {6. \storeinstr{} 2 (UN)};
\node[instr] (C2L6) [below = of C2L5]  {7. \loadinstr{}  3 (AH)};
\node[instr] (C2L7) [below = of C2L6]  {8. \storeinstr{} 1 (AM)};
\node[instr] (C2L8) [below = of C2L7]  {9. \loadinstr{}  2 (UN)};
\node[instr] (C2L9) [below = of C2L8]  {10. \storeinstr{} 3 (AM)};


\path[dashed,every node/.style={sloped, anchor=east}]
   (C2L0) edge [very near end] node [above]{EX} (C1L2)
   (C2L5) edge [very near end] node [above]{EX} (C1L8)
   (C2L8) edge [very near end] node [above]{DE} (C1L9)

   (C1L0) edge [very near end] node [above]{EX} (C2L4)
   (C1L3) edge [very near end] node [above]{EX} (C2L4)
   (C1L5) edge [very near end] node [above]{EX} (C2L5)
   (C1L7) edge [very near end] node [above]{EX} (C2L7)
   (C1L5) edge [very near end] node [above]{EX} (C2L8)
;

\path[every node/.style={sloped, anchor=east}]
   (C2L3) edge [very near end] node [above]{EX} (C1L5)
   (C2L4) edge [very near end] node [above]{DE} (C1L7)



   (C1L4) edge [very near end] node [above]{DE} (C2L9)
;


\end{tikzpicture}

\caption{Interférence dans l'exemple de modèle instancié}
\label{fr:fig:analysis:interference_between_instructions}
\end{figure}

\begin{example}[Interférences dans notre exemple]
  La Figure~\ref{fr:fig:analysis:interference_between_instructions}
  montre les interférences entre les instructions des programmes de
  l'exemple. Les flèches partent de l'instruction générant
  l'interférence. Celles qui sont en pointillés indiquent une
  interférence ne se produisant pas dans certaines des traces
  d'exécution du modèle.  \lstinline!EX! correspond à une interférence
  d'expulsion et \lstinline!DE!  à une interférence de rétrogradation.
  La colonne de gauche correspond au programme de Cœur 1 et celle de
  droite à celui de Cœur 2.  
\end{example}
