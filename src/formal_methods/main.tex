\chapter{(Timed) Automata}
\label{cha:formal_methods}
\label{cha:timed_automata}
\definechapterdirectory{src/formal_methods}
This chapter presents the concept of timed automata, which is the formal model
used in this thesis, and an associated formal method: reachability analysis
through model checking. In the first section, the main concepts behind
classical automata are introduced, with some fairly common additions: the use
of variables, having conditions and actions in transitions, and
synchronization. A definition of the temporal logic operators used in this
thesis is also given in the classical automata section. Timed automata are
introduced in a second section. As this thesis does not dwell into the theory
of automata, but simply uses them as a modeling tool, many details (such as
their precise semantics, the language theory or the details of how model
checking is achieved) are omitted.  Indeed, the objective of the chapter is for
the reader to have an understanding of the models of cache coherence presented
in later chapters, as well as the operators being used to query on them. Since
the aforementioned details are not directly related to the work presented in
this thesis, they are considered to be outside of the scope of this chapter.

\section{Classical Automata}
This section is meant as a reminder on classical automata. Readers looking for
in-depth information on the subject are encouraged to read \cite{arnold},
\cite{hopcroft}, or \cite{Schnoebelen:10.5555/1965314}.

\subsection{System Definition}

\begin{definition}[Syntax of Constraints and Actions]
\label{def:formal_methods:transition_grammar}
Given a set of variables \automatavariables{}, the grammar used when writing
constraints and actions in transitions is as follows, with \textbf{ident}
standing for a variable in \automatavariables{}:\\
$\textbf{lop} ::= \land~|~\lor $\\
$\textbf{cop} ::=\!\!\! ~< | \le | = | \ge | > $\\
$\textbf{mop} ::= + | - | * |~/$\\
$\textbf{mexpr} ::=
   \textbf{mexpr}~\textbf{mop}~\textbf{mexpr}
   ~|~ \textbf{ident}
   ~|~ \mathbb{Z}
$\\
$
\textbf{abexpr} ::=
   \textbf{mexpr}~\textbf{cop}~\textbf{mexpr}
   ~|~ \texttt{true}
   ~|~ \texttt{false}
$\\
$
\textbf{bexpr} ::=
   \neg \textbf{bexpr}
   ~|~ \textbf{bexpr}~\textbf{lop}~\textbf{bexpr}
   ~|~ \textbf{abexpr}
$\\
$
\textbf{assign} ::=
   \textbf{assign};~\textbf{assign}
   ~|~ \textbf{ident}~:=~\textbf{mexpr}
   ~|~ if~(\textbf{bexpr})~\{\textbf{assign}\}
   ~|~ \texttt{nop}
$
\end{definition}

\begin{definition}[Classical Automata System]
A classical automata system \automatasystem{} is a tuple
$\langle \automatastates{}, \allowbreak{}
\automatainit{}, \allowbreak{}
\automatacondlabels{}, \allowbreak{}
\automatachanlabels{}, \allowbreak{}
\automatavariables{}, \allowbreak{}
\automataactionlabels{}, \allowbreak{}
\automatarelations{}\rangle$ where:
\begin{itemize}
\item \automatastates{} is a finite set of states.
\item \automatainit{} is the initial state ($\automatainit{} \in
\automatastates{}$).
\item \automatavariables{} is a finite set of integer variables, taking their value on a finite subset \automatavardomain{} of integers.
\item
   $\automatachanlabels{} = \automatachanlabels{}^{\alpha} \cup
   \automatachanlabels{}^{\texttt{sync}}$ is a finite set of labels, with
   $\automatachanlabels{}^{\texttt{sync}}$ corresponding to labels meant for
   synchronization and $\automatachanlabels{}^{\alpha}$ being regular labels.
   The labels in $\automatachanlabels{}^{\texttt{sync}}$ affixed by either `?'
   or `!', with `?' denoting a reception on a ``channel'', and `!' an emission.
\item
   \automatacondlabels{} = \textbf{bexpr}(\automatavariables{}), as defined in
   Definition~\ref{def:formal_methods:transition_grammar}.
\item
   \automataactionlabels{} = \textbf{assign}(\automatavariables{}), as defined
   in Definition~\ref{def:formal_methods:transition_grammar}.
\item
   $\automatarelations{} \subseteq
      \automatastates{}
      \times \automatacondlabels{}
      \times \automatachanlabels{}
      \times \automataactionlabels{}
      \times \automatastates{}
   $ is the transition relation.
 \end{itemize}
 Note that we use both a set \automatastates{} of
 states and a set \automatavariables{} of integer variables, in order
 to be close to the framework of UPPAAL, which we are using in the
 rest of the thesis.
 
The semantics of \automatasystem{} is given via its set of valid
transitions (see Definition~\ref{sec:system-definition}) and its
execution traces (see Definition~\ref{def:formal_methods:trace}).
\end{definition}

\begin{definition}[Valuation]
Valuations map variables to their value:
$\automataenvironment{}:\automatavariables{} \to \automatavardomain$.
Given a valuation $\automataenvironment{}$, and a guard
$c \in \automatacondlabels{}$, we note $\automataenvironment{} \models_{PL} c$
to indicate that $c$ is true under the valuation $v$.

Similarly, given $a \in \automataactionlabels{}$, $v[a]$ denotes the valuation
obtained from $v$ by the application of the action $a$, were all variables
updated by $a$ have their new value and all other variables keep their previous
value.
\end{definition}


\begin{definition}[Transition]
  \label{sec:system-definition}
Given an automaton
$\automatasystem{} = \allowbreak{}
\langle \automatastates{}, \allowbreak{}
\automatainit{}, \allowbreak{}
\automatacondlabels{}, \allowbreak{}
\automatachanlabels{}, \allowbreak{}
\automatavariables{}, \allowbreak{}
\automataactionlabels{}, \allowbreak{}
\automatarelations{}\rangle$, we define \automatanext{}, which
indicates all valid transitions that can be performed from
$\langle \automatastate{}, \automataenvironment{}\rangle$, with
$s \in \automatastates{}$ and $\automataenvironment{}$ a valuation:
$\automatanext{}(\langle \automatastate{}, \automataenvironment{}\rangle)
\triangleq \{\langle \automatastate{}', \automataenvironment{}'\rangle
|
   \existsin{\langle \automatastate{}, c, l, a, \automatastate{}' \rangle}{\automatarelations{}}{%
%      (o = \automatastate{})
%      \land (d = \automatastate{}')
      (\automataenvironment{} \models_{PL} c)
      \land
      \automataenvironment{}' = \automataenvironment{}[a]
   }\}
$
\end{definition}

\begin{definition}[Path \& Trace]
\label{def:formal_methods:trace}
We consider a path to be a \emph{maximal} sequence of
states/transitions
$\langle \automatastate{1}, \automataenvironment{1}\rangle
\automatatransition{} \langle \automatastate{2},
\automataenvironment{2}\rangle \automatatransition{} \cdots$ such that
for each $i$,
$\langle \automatastate{i+1}, \automataenvironment{i+1}\rangle \in
\automatanext{}(\langle \automatastate{i},
\automataenvironment{i}\rangle) $. The sequence is maximal in the
sense that it is either infinite or of length $N$ and such that
$\automatanext{}(\langle \automatastate{N},
\automataenvironment{N}\rangle)$ is empty. A path
starting from
$ \langle \automatainit{}, \automataenvironment{0}\rangle$ (where
\automataenvironment{0} is the initial valuation) is called a trace.
\end{definition}

\begin{figure}[hbt!]
   \centering
   \begin{tabular}{cc}
   \begin{tikzpicture}[->,>=stealth',shorten >=1pt,auto,node distance=3cm,
                    semithick]
   \node[initial,state] (S0)              {$S_0$};
   \node[state] (S1) [right of=S0] {$S_1$};
   \node[state] (SE) [below of=S1] {$S_E$};

   \path[every node/.style={sloped, anchor=center, yshift=1em}]
      (S0) edge [bend left] node [above=-1em]{
      \begin{tabular}{l}
         $\textbf{request\_files}!$\\
         $\textit{fetched} := 0$
      \end{tabular}
      } (S1)

      (S0) edge [bend right] node [above=-1em]{ $\textbf{err}$ } (SE)

      (S1) edge [bend left] node [below=1em] {
      \begin{tabular}{l}
         $\textbf{done}?$\\
      \end{tabular}
      } (S0)

      (S1) edge [bend left] node [above=-1em]{ $\textbf{err}$ } (SE)

      (S1) edge [loop right] node [below=1em] {
      \begin{tabular}{l}
         $\textbf{new\_file}?$\\
         $\textit{fetched} := \textit{fetched} + 1$\\
      \end{tabular}
      } (S1)
   ;
\end{tikzpicture}
 &
   \input{\chapterdirectory/figure/classical_automata_new_b}
   \end{tabular}
   \caption{Example of two classical automata}
   \label{fig:classical_automata}
\end{figure}

\begin{example}[Classical Automata]
\label{ex:classical_automata}
Figure~\ref{fig:classical_automata} shows two automata modeling a client (on
the left) that fetches a number of files from a server (on the right).  In this
scenario, the system loops infinitely, with the client initiating a request for
files (\textbf{request\_files}), and counting (\textit{fetched}) their arrival
(\textbf{new\_file}) until the server indicates that all were transfered
(\textbf{done}). On each request, the server sends exactly 386 files (as
counted by \textit{sent}).
\end{example}

\begin{example}[Traces]
Here are some examples of traces for the client automaton from
Example~\ref{ex:classical_automata}:
\begin{itemize}
\item
   $
   \langle S_0, \{\langle \textit{fetched}, 0 \rangle\}\rangle \allowbreak{}
   \automatatransitiontrace{\textbf{err}}{} \allowbreak{}
   \langle S_E, \{\langle \textit{fetched}, 0 \rangle\}\rangle
   $
\item
   $
   \langle S_0, \{\langle \textit{fetched}, 0 \rangle\}\rangle \allowbreak{}
   \automatatransitiontrace{\textbf{request\_files!}}{\textit{fetched} := 0} \allowbreak{}
   \langle S_1, \{\langle \textit{fetched}, 0 \rangle\}\rangle \allowbreak{}
   \automatatransitiontrace{\textbf{err}}{} \allowbreak{}
   \langle S_E, \{\langle \textit{fetched}, 0 \rangle\}\rangle
   $
\end{itemize}
And for the server automaton:
\begin{itemize}
\item
   $
   \langle S_0, \{\langle \textit{sent}, 0 \rangle\}\rangle \allowbreak{}
   \automatatransitiontrace{\textbf{err}}{} \allowbreak{}
   \langle S_E, \{\langle \textit{sent}, 0 \rangle\}\rangle
   $
\item
   $
   \langle S_0, \{\langle \textit{sent}, 0 \rangle\}\rangle \allowbreak{}
   \automatatransitiontrace{\textbf{request\_files?}}{\textit{sent} := 0} \allowbreak{}
   \langle S_1, \{\langle \textit{sent}, 0 \rangle\}\rangle \allowbreak{}
   \automatatransitiontrace{\textbf{new\_file!}\\\textit{sent} < 386}{\textit{sent} := \textit{sent} + 1} \allowbreak{}
   \langle S_1, \{\langle \textit{sent}, 1 \rangle\}\rangle \allowbreak{}
   \automatatransitiontrace{\textbf{new\_file!}\\\textit{sent} < 386}{\textit{sent} := \textit{sent} + 1} \allowbreak{} \allowbreak{}
   \cdots
   \automatatransitiontrace{\textbf{new\_file!}\\\textit{sent} < 386}{\textit{sent} := \textit{sent} + 1} \allowbreak{} \allowbreak{}
   \langle S_1, \{\langle \textit{sent}, 386 \rangle\}\rangle \allowbreak{}
   \automatatransitiontrace{\textbf{done!}\\\textit{sent} = 386}{} \allowbreak{} \allowbreak{}
   \langle S_0, \{\langle \textit{sent}, 386 \rangle\}\rangle \allowbreak{}
   \automatatransitiontrace{\textbf{err}}{} \allowbreak{}
   \langle S_E, \{\langle \textit{sent}, 386 \rangle\}\rangle
   $
\end{itemize}
\end{example}

\begin{definition}[Synchronized automata]
Given \emph{n} automata
$\automatasystem{}_i = \allowbreak{}
\langle \automatastates{}_i, \allowbreak{}
\automatainit{}_i, \allowbreak{}
\automatacondlabels{}_i, \allowbreak{}
\automatachanlabels{}_i, \allowbreak{}
\automatavariables{}_i, \allowbreak{}
\automataactionlabels{}_i, \allowbreak{}
\automatarelations{}_i\rangle$, and a synchronization constraint
$\automatasyncconstraint{} \subseteq (\automatachanlabels{}_1 \cup \{-\})
\times \cdots \times (\automatachanlabels{}_n \cup \{-\})$, we can define a new automaton
$\automatasystem{}_s = \allowbreak{}
\langle \automatastates{}_s, \allowbreak{}
\automatainit{}_s, \allowbreak{}
\automatacondlabels{}_s, \allowbreak{}
\automatachanlabels{}_s, \allowbreak{}
\automatavariables{}_s, \allowbreak{}
\automataactionlabels{}_s, \allowbreak{}
\automatarelations{}\rangle$,
corresponding to the synchronized product of the $\automatasystem{}_i$ automata
according to \automatasyncconstraint{}, with the following rules:
\begin{itemize}
\item $\automatastates{}_s =
   \automatastates{}_1 \times \cdots \times \automatastates{}_n$
\item $\automatainit{}_s =
   \langle{} \automatainit{}_1, \cdots, \automatainit{}_n\rangle{}$
\item $\automatacondlabels{}_s =
   \automatacondlabels{}_1 \times \cdots \times \automatacondlabels{}_n$
\item $\automatachanlabels{}_s =
(\automatachanlabels{}_1 \cup \{-\})
\times \cdots \times (\automatachanlabels{}_n \cup \{-\})$. We extend the labels
   with $-$ to mark that a sub-automaton does not perform any transition.
\item $\automatavariables{}_s =
   \automatavariables{}_1 \cup \cdots \cup \automatavariables{}_n$,
   with $\forallin{i,j}{1..n}{(i \neq j) \implies (\automatavariables{}_i \cap \automatavariables{}_j = \emptyset)}$.
\item $\automataactionlabels{}_s =
   \automataactionlabels{}_1 \times \cdots \times \automataactionlabels{}_n$
\item
   $\automatarelations{}_s \subseteq
      \automatastates{}_s
      \times \automatacondlabels{}_s
      \times \automatachanlabels{}_s
      \times \automataactionlabels{}_s
      \times \automatastates{}_s
   $, with
   $$
   \langle
      \langle o_1, \cdots, o_n \rangle,
      \langle c_1, \cdots, c_n \rangle,
      \langle l_1, \cdots, l_n \rangle,
      \langle a_1, \cdots, a_n \rangle,
      \langle d_1, \cdots, d_n \rangle
   \rangle \in \automatarelations{}_s
   $$
   $$
   \iff
   \begin{cases}
      \langle l_1, \cdots, l_n \rangle \in \automatasyncconstraint{}\\
      \langle
         \forall i \in 1..n :
         o_i, c_i, l_i, a_i, d_i \rangle \in \automatarelations{}_i
         \lor (o_i = d_i \land l_i = - \land c_i = \texttt{true} \land a_i =
         \textit{nop})
   \end{cases}
   $$
\end{itemize}
\automatasyncconstraint{} is implicitely defined by the labels in
$\automatachanlabels_{1..n}$: for any transition with a label in
$\automatachanlabels^{\alpha}$, there is an entry in \automatasyncconstraint{}
with no other simultaneous transition allowed (indicated by $-$, which means
the particular sub-automaton does not perform a transition). For any transition
$chan$ in $\automatachanlabels^{\texttt{sync}}$, \automatasyncconstraint{} has
an entry for each possible $chan!$, $chan?$, $-$ label combination such that
there is a single $chan!$ label and a single $chan?$ label. This convention was
introduced in CCS (\cite{10.5555/539036}).
\end{definition}

\begin{definition}[Synchronized Automata Semantics]
As the product of synchronized automata is itself a classical automaton, its
semantics is the same as those from Definition~\ref{def:formal_methods:trace}.
\end{definition}

\iffalse
\begin{definition}[Synchronized Automata Semantics]
A valid transition for $\automatasystem{}_s$ is then defined as:
$\automatanext{}(\langle \automatastate{}, \automataenvironment{}\rangle)
\triangleq \{\langle \automatastate{}', \automataenvironment{}'\rangle
|
   \existsin{\langle \automatastate{}, c, l, a, \automatastate{}' \rangle}{%
      \automatarelations{}_s}{%
      (\llbracket{}c\rrbracket_{\automataenvironment{}} = true)
      \land (\automataenvironment{}' = \llbracket{}a\rrbracket_{\automataenvironment{}})
      \land l \in \automatasyncconstraint{}
   }\}
$
\end{definition}
\fi
\begin{figure}[hbt!]
   \centering
   \begin{tikzpicture}[->,>=stealth',shorten >=1pt,auto,node distance=3.5cm,
                    semithick]
   \node[state] (S00)        {$\langle S_0, S_0 \rangle$};
   \node[]  (Shadow0) [above of=S00,yshift=-1.5cm]       {start};
   \node[state] (S11) [below of=S00] {$\langle S_1, S_1 \rangle$};
   \node[state] (S0E) [right of=S00] {$\langle S_0, S_E \rangle$};
   \node[state] (SE0) [right of=S0E] {$\langle S_E, S_0 \rangle$};
   \node[state] (S1E) [right of=S11] {$\langle S_1, S_E \rangle$};
   \node[state] (SE1) [right of=S1E] {$\langle S_E, S_1 \rangle$};
   \node[state] (SEE) [below right of=SE0,yshift=0.6445cm] {$\langle S_E, S_E \rangle$};

   \path[every node/.style={sloped, anchor=center, yshift=1em}]
      (Shadow0) edge node {} (S00)

      (S00) edge [bend right] node [below=2em]{
      \begin{tabular}{l}
         $\langle \textbf{request\_files}!, \textbf{request\_files}? \rangle$\\
         $\langle \textit{fetched} := 0, \textit{sent} := 0 \rangle$
      \end{tabular}
      } (S11)

      (S11) edge [bend right] node [below=2em]{
      \begin{tabular}{l}
         $\langle \texttt{true}, \textit{sent} = 386 \rangle$\\
         $\langle \textbf{done}?, \textbf{done}! \rangle$
      \end{tabular}
      } (S00)

      (S00) edge [bend left] node [below=1.5em]{ $\langle -, \textbf{err} \rangle$ } (S0E)
      (S00) edge [bend left] node { $\langle \textbf{err}, - \rangle$ } (SE0)
      (S0E) edge node [below=1em] { $\langle \textbf{err}, - \rangle$ } (SEE)
      (SE0) edge [bend left] node  { $\langle -, \textbf{err} \rangle$ } (SEE)


      (S11) edge [bend right] node [above=-0.5em]{ $\langle -, \textbf{err} \rangle$ } (S1E)
      (S11) edge [bend right] node [below=1em] { $\langle \textbf{err}, - \rangle$ } (SE1)
      (S1E) edge node { $\langle \textbf{err}, - \rangle$ } (SEE)
      (SE1) edge [bend right] node [below=1.5em]  { $\langle -, \textbf{err} \rangle$ } (SEE)

      (S11) edge [loop below] node [below=0.8em,xshift=2cm]{%
      \begin{tabular}{l}
         $\langle \textbf{new\_file}?, \textbf{new\_file}! \rangle$\\
         $\langle \textit{fetched} := \textit{fetched} + 1, \textit{sent} := \textit{sent} + 1\rangle$
      \end{tabular}
      } (S1E)
   ;
\end{tikzpicture}

   \caption{Example of synchronized automaton}
   \label{fig:classical_synchronized_automaton}
\end{figure}

\begin{example}
Figure~\ref{fig:classical_automata} is in fact a network of automata, which is
one way of representing synchronization between automata.
Figure~\ref{fig:classical_synchronized_automaton} shows another representation,
with a single automaton resulting from the synchronized product of the automata
from Figure~\ref{fig:classical_automata}. In this case,
\automatasyncconstraint{} is defined as:
$\{
   \langle \textbf{request\_files}!, \textbf{request\_files}? \rangle, \allowbreak{}
   \langle \textbf{done}?, \textbf{done}! \rangle, \allowbreak{}
   \langle \textbf{new\_file}?, \textbf{new\_file}! \rangle, \allowbreak{}
   \langle \textbf{err}, - \rangle, \allowbreak{}
   \langle -, \textbf{err} \rangle \allowbreak{}
\}$.
\end{example}

\stopallthesefloats{}
\subsection{Query Logic Operators and Semantics}
\label{sec:automata_query_logic}
Given an automaton \automatasystem{} and an initial valuation
$\automataenvironment{0}$, we can define the satisfaction relation for a
property $\phi$. We assume $\phi$ to be a formula in
$\textbf{bexpr}(\automatavariables{})$
written in the subset of CTL
(\cite{10.1145/567067.567080}) temporal operators described below. Readers
interested in the details of how these are actually verified are encouraged to
read on model-checking (for example, \cite{clarke}).
The satisfaction of
$\langle \automatastate{}, \automataenvironment{} \rangle \models \phi$ is
defined using the following decomposition:
\iffalse
\begin{description}
\item[%
   $
      \langle
         \automatainit{},
         \automataenvironment{0}
      \rangle{}
      \models
      \afop{}~\phi \triangleq
   $]~~\\
      For all paths starting from
         $\langle
            \automatainit{},
            \automataenvironment{0}
         \rangle$,
         there is, within the path, a
         $\langle
            \automatastate{}',
            \automataenvironment{}'
         \rangle$,
         such that
         $
         \automataenvironment{}'
         \models_{PL} \phi
         $
\item[%
   $
      \langle
         \automatainit{},
         \automataenvironment{0}
      \rangle{}
      \models
      \efop{}~\phi \triangleq
   $]~~\\
      There is a path starting from
         $\langle
            \automatainit{},
            \automataenvironment{0}
         \rangle$ in which is found a
         $\langle
            \automatastate{}',
            \automataenvironment{}'
         \rangle$,
         such that
         $
         \automataenvironment{}'
         \models_{PL} \phi
         $
\item[%
   $
      \langle
         \automatainit{},
         \automataenvironment{0}
      \rangle{}
      \models
      \agop{}~\phi \triangleq
   $]~~\\
      For all paths starting from
         $\langle
            \automatainit{},
            \automataenvironment{0}
         \rangle$,
         all
         $\langle
            \automatastate{}',
            \automataenvironment{}'
         \rangle$
         of the path
         verify
         $
         \automataenvironment{}'
         \models_{PL} \phi
         $
\item[%
   $
      \langle
         \automatainit{},
         \automataenvironment{0}
      \rangle{}
      \models
      \egop{}~\phi \triangleq
   $]~~\\
      There is a path starting from
         $\langle
            \automatainit{},
            \automataenvironment{0}
         \rangle$, such that all 
         $\langle
            \automatastate{}',
            \automataenvironment{}'
         \rangle$
         verify
         $
         \automataenvironment{}'
         \models_{PL} \phi
         $
\item[%
   $
      \langle
         \automatainit{},
         \automataenvironment{0}
      \rangle{}
      \models
      \leadstoop{\phi}{\psi} \triangleq
   $]~~\\
      For any paths \pi starting from
         $\langle
            \automatainit{},
            \automataenvironment{0}
         \rangle$,
         any suffix of \pi starting from a
         $\langle
            \automatastate{}',
            \automataenvironment{}'
         \rangle$
         such that
         $
         \automataenvironment{}'
         \models_{PL} \phi
         $
         also contains some
         $\langle
            \automatastate{}'',
            \automataenvironment{}''
         \rangle$
         such that
         $
         \automataenvironment{}''
         \models_{PL} \psi
         $.\footnote{In CTL, this can also be defined as $\agop{}(\phi \implies
         \afop{}~\psi)$}
\end{description}
\fi
\begin{description}
\item[%
   $
      \langle
         \automatastate{},
         \automataenvironment{}
      \rangle{}
      \models
      \psi
      \triangleq
   $]
   $\automataenvironment{} \models_{PL} \psi$, where $\psi$ is an
   expression in $\textbf{abexpr}(\automatavariables{})$.
\item[%
   $
      \langle
         \automatastate{},
         \automataenvironment{}
      \rangle{}
      \models
      \neg \phi
      \triangleq
   $]
   $
      \langle
         \automatastate{},
         \automataenvironment{}
      \rangle{}
      \not\models
      \phi
   $
\item[%
   $
      \langle
         \automatastate{},
         \automataenvironment{}
      \rangle{}
      \models
      \phi \land \psi
      \triangleq
   $]
   $
      (\langle
         \automatastate{},
         \automataenvironment{}
      \rangle{}
      \models
      \phi)
   $
   and
   $(\langle
         \automatastate{},
         \automataenvironment{}
      \rangle{}
      \models
      \psi)
   $
\item[%
   $
      \langle
         \automatastate{},
         \automataenvironment{}
      \rangle{}
      \models
      \afop{}~\phi \triangleq
   $]~~\\
      For all paths starting from
         $\langle
            \automatastate{},
            \automataenvironment{}
         \rangle$,
         there is, within the path, a
         $\langle
            \automatastate{}',
            \automataenvironment{}'
         \rangle$,
         such that
         $\langle
            \automatastate{}',
            \automataenvironment{}'
         \rangle{}
         \models~\phi
         $
\item[%
   $
      \langle
         \automatastate{},
         \automataenvironment{}
      \rangle{}
      \models
      \efop{}~\phi \triangleq
   $]~~\\
      There is a path starting from
         $\langle
            \automatastate{},
            \automataenvironment{}
         \rangle$ in which is found a
         $\langle
            \automatastate{}',
            \automataenvironment{}'
         \rangle$,
         such that
         $\langle
            \automatastate{}',
            \automataenvironment{}'
         \rangle{}
         \models~\phi
         $
\item[%
   $
      \langle
         \automatastate{},
         \automataenvironment{}
      \rangle{}
      \models
      \agop{}~\phi \triangleq
   $]~~\\
      For all paths starting from
         $\langle
            \automatastate{},
            \automataenvironment{}
         \rangle$,
         all
         $\langle
            \automatastate{}',
            \automataenvironment{}'
         \rangle$
         of the path
         verify
         $\langle
            \automatastate{}',
            \automataenvironment{}'
         \rangle{}
         \models~\phi
         $
\item[%
   $
      \langle
         \automatastate{},
         \automataenvironment{}
      \rangle{}
      \models
      \egop{}~\phi \triangleq
   $]~~\\
      There is a path starting from
         $\langle
            \automatastate{},
            \automataenvironment{}
         \rangle$, such that all
         $\langle
            \automatastate{}',
            \automataenvironment{}'
         \rangle$
         verify
         $\langle
            \automatastate{}',
            \automataenvironment{}'
         \rangle{}
         \models~\phi
         $
\item[%
   $
      \langle
         \automatastate{},
         \automataenvironment{}
      \rangle{}
      \models
      \leadstoop{\phi}{\psi} \triangleq
   $]~~\\
      For all paths starting from
         $\langle
            \automatastate{},
            \automataenvironment{}
         \rangle$,
         any sub-path starting from a
         $\langle
            \automatastate{}',
            \automataenvironment{}'
         \rangle$
         such that
         $
         \automataenvironment{}'
         \models_{PL} \phi
         $
         also contains at least one
         $\langle
            \automatastate{}'',
            \automataenvironment{}''
         \rangle$
         such that
         $
         \automataenvironment{}''
         \models_{PL} \psi
         $.
\end{description}
\iffalse
\subsection{Reachability}
\begin{description}
\item[%
   $
      \langle
         \automatasystem{},
         \automatastate{},
         \automataenvironment{}
      \rangle{}
      \models
      \phi
      \triangleq
   $]
   $\llbracket\phi\rrbracket_{\automataenvironment{}}$
\item[%
   $
      \langle
         \automatasystem{},
         \automatastate{},
         \automataenvironment{}
      \rangle{}
      \models
      \axop{}~\phi \triangleq
   $]~~\\
   $
      \forallin{%
         \langle{}
            \automatastate{}',
            \automataenvironment{}'
         \rangle{}
      }{\automatanext{}(\langle{} \automatastate{}, \automataenvironment{} \rangle{})}{%
         \langle
            \automatasystem{},
            \automatastate{}',
            \automataenvironment{}'
         \rangle{}
         \models \phi
      }
   $
\item[%
   $
      \langle
         \automatasystem{},
         \automatastate{},
         \automataenvironment{}
      \rangle{}
      \models
      \exop{}~\phi \triangleq
   $]~~\\
   $
      \existsin{%
         \langle{}
            \automatastate{}',
            \automataenvironment{}'
         \rangle{}
      }{\automatanext{}(\langle{} \automatastate{}, \automataenvironment{} \rangle{})}{%
         \langle
            \automatasystem{},
            \automatastate{}',
            \automataenvironment{}'
         \rangle{}
         \models \phi
      }
   $
\item[%
   $
      \langle
         \automatasystem{},
         \automatastate{},
         \automataenvironment{}
      \rangle{}
      \models
      \afop{}~\phi \triangleq
   $]~~\\
   $
      (
         \langle
            \automatasystem{},
            \automatastate{},
            \automataenvironment{}
         \rangle{}
         \models \phi
      )
      \lor
      \forallin{%
         \langle{}
            \automatastate{}',
            \automataenvironment{}'
         \rangle{}
      }{\automatanext{}(\langle{} \automatastate{}, \automataenvironment{} \rangle{})}{%
         \langle
            \automatasystem{},
            \automatastate{}',
            \automataenvironment{}'
         \rangle{}
         \models \afop{}~\phi
      }
   $
\item[%
   $
      \langle
         \automatasystem{},
         \automatastate{},
         \automataenvironment{}
      \rangle{}
      \models
      \efop{}~\phi \triangleq
   $]~~\\
   $
      (
         \langle
            \automatasystem{},
            \automatastate{},
            \automataenvironment{}
         \rangle{}
         \models \phi
      )
      \lor
      \existsin{%
         \langle{}
            \automatastate{}',
            \automataenvironment{}'
         \rangle{}
      }{\automatanext{}(\langle{} \automatastate{}, \automataenvironment{} \rangle{})}{%
         \langle
            \automatasystem{},
            \automatastate{}',
            \automataenvironment{}'
         \rangle{}
         \models \efop{}~\phi
      }
   $
\item[%
   $
      \langle
         \automatasystem{},
         \automatastate{},
         \automataenvironment{}
      \rangle{}
      \models
      \agop{}~\phi \triangleq
   $]~~\\
   $
      (
         \langle
            \automatasystem{},
            \automatastate{},
            \automataenvironment{}
         \rangle{}
         \models \phi
      )
      \land
      \forallin{%
         \langle{}
            \automatastate{}',
            \automataenvironment{}'
         \rangle{}
      }{\automatanext{}(\langle{} \automatastate{}, \automataenvironment{} \rangle{})}{%
         \langle
            \automatasystem{},
            \automatastate{}',
            \automataenvironment{}'
         \rangle{}
         \models \agop{}~\phi
      }
   $
\item[%
   $
      \langle
         \automatasystem{},
         \automatastate{},
         \automataenvironment{}
      \rangle{}
      \models
      \egop{}~\phi \triangleq
   $]~~\\
   $
      (
         \langle
            \automatasystem{},
            \automatastate{},
            \automataenvironment{}
         \rangle{}
         \models \phi
      )
      \land
      \ifexistsin{%
         \langle{}
            \automatastate{}',
            \automataenvironment{}'
         \rangle{}
      }{\automatanext{}(\langle{} \automatastate{}, \automataenvironment{} \rangle{})}{%
         \langle
            \automatasystem{},
            \automatastate{}',
            \automataenvironment{}'
         \rangle{}
         \models \egop{}~\phi
      }
   $
\item[%
   $
      \langle
         \automatasystem{},
         \automatastate{},
         \automataenvironment{}
      \rangle{}
      \models
      \auop{\phi}{\psi} \triangleq
   $]~~\\
   $
      (
         \langle
            \automatasystem{},
            \automatastate{},
            \automataenvironment{}
         \rangle{}
         \models \psi
      )
      %\verifiesfun{}(\psi, \automataenvironment{})
      \lor
      (
         %\verifiesfun{}(\phi, \automataenvironment{})
         (
            \langle
               \automatasystem{},
               \automatastate{},
               \automataenvironment{}
            \rangle{}
            \models \phi
         )
         \land
      \forallin{%
         \langle{}
            \automatastate{}',
            \automataenvironment{}'
         \rangle{}
      }{\automatanext{}(\langle{} \automatastate{}, \automataenvironment{} \rangle{})}{%
         \langle
            \automatasystem{},
            \automatastate{}',
            \automataenvironment{}'
         \rangle{}
         \models
         \auop{\phi}{\psi}
      }
   $
\item[%
   $
      \langle
         \automatasystem{},
         \automatastate{},
         \automataenvironment{}
      \rangle{}
      \models
      \euop{\phi}{\psi} \triangleq
   $]~~\\
   $
      (
         \langle
            \automatasystem{},
            \automatastate{},
            \automataenvironment{}
         \rangle{}
         \models \psi
      )
      %\verifiesfun{}(\psi, \automataenvironment{})
      \lor
         %\verifiesfun{}(\phi, \automataenvironment{})
         (
            \langle
               \automatasystem{},
               \automatastate{},
               \automataenvironment{}
            \rangle{}
            \models \phi
         )
         \land
      \existsin{%
         \langle{}
            \automatastate{}',
            \automataenvironment{}'
         \rangle{}
      }{\automatanext{}(\langle{} \automatastate{}, \automataenvironment{} \rangle{})}{%
         \langle
            \automatasystem{},
            \automatastate{}',
            \automataenvironment{}'
         \rangle{}
         \models
         \euop{\phi}{\psi}
      }
   $
\end{description}
\fi
\iffalse
\paragraph{Linear Temporal Logic}
Given $t\in\automatatraces{}$,
\begin{description}
\item[%
   $
      \langle{}
         \automatasystem{},
         t,
         i
      \rangle{}
      \models
      \xop{}~\phi \triangleq
   $]
   $
      \langle{}
         \automatasystem{},
         t,
         i+1
      \rangle{}
      \models
      \phi
   $
\item[%
   $
      \langle{}
         \automatasystem{},
         t,
         i
      \rangle{}
      \models
      \fop{}~\phi \triangleq
   $]
   $
      (
         \langle{}
            \automatasystem{},
            t,
            i
         \rangle{}
         \models
         \phi
      )
      \lor
      (
         \langle{}
            \automatasystem{},
            t,
            i+1
         \rangle{}
         \models
         \fop{}~\phi
      )
   $
\item[%
   $
      \langle{}
         \automatasystem{},
         t,
         i
      \rangle{}
      \models
      \gop{}~\phi \triangleq
   $]
   $
      (
         \langle{}
            \automatasystem{},
            t,
            i
         \rangle{}
         \models
         \phi
      )
      \land
      (
         (
            |t| \leq i
         )
         \lor
         \langle{}
            \automatasystem{},
            t,
            i+1
         \rangle{}
         \models
         \gop{}~\phi
      )
   $
\item[%
   $
      \langle{}
         \automatasystem{},
         t,
         i
      \rangle{}
      \models
      \uop{\phi}{\psi} \triangleq
   $]
   $
      (
         \langle{}
            \automatasystem{},
            t,
            i
         \rangle{}
         \models
         \phi
      )
      \lor
      (
         (
            \langle{}
               \automatasystem{},
               t,
               i
            \rangle{}
            \models
            \psi
         )
         \land (|t| > i)
         \land
         \langle{}
            \automatasystem{},
            t,
            i+1
         \rangle{}
         \models
         \uop{\phi}{\psi}
      )
   $
\item[%
   $
      \langle{}
         \automatasystem{},
         t,
         i
      \rangle{}
      \models
      \rop{\phi}{\psi} \triangleq
   $]
   $
      (
         \langle{}
            \automatasystem{},
            t,
            i
         \rangle{}
         \models
         \psi
      )
      \land
      (
         (
            \langle{}
               \automatasystem{},
               t,
               i
            \rangle{}
            \models
            \phi
         )
         \lor (|t| \leq i)
         \land
         (
            \langle{}
               \automatasystem{},
               t,
               i+1
            \rangle{}
            \models
            \rop{\phi}{\psi}
         )
      )
   $
\item[%
   $
      \langle{}
         \automatasystem{},
         t,
         i
      \rangle{}
      \models
      \wop{\phi}{\psi} \triangleq
   $]
   $
      (
         \langle{}
            \automatasystem{},
            t,
            i
         \rangle{}
         \models
         \phi
      )
      \lor
      (
         \langle{}
            \automatasystem{},
            t,
            i
         \rangle{}
         \models
         \psi
      )
      \lor (|t| \leq i)
      \lor
      \langle{}
         \automatasystem{},
         t,
         i+1
      \rangle{}
      \models
      \uop{\phi}{\psi}
   $
\item[%
   $
      \langle{}
         \automatasystem{},
         t,
         i
      \rangle{}
      \models
      \mop{\phi}{\psi} \triangleq
   $]
   $
      (
         \langle{}
            \automatasystem{},
            t,
            i
         \rangle{}
         \models
         \rop{\phi}{\psi}
      )
      \land
      (
         \langle{}
            \automatasystem{},
            t,
            i
         \rangle{}
         \models
         \fop{}~\phi
      )
   $
\end{description}
\fi

\begin{example}
Examples of reachability analysis that would be relevant for the system used in
Example~\ref{ex:classical_automata} include:
\begin{itemize}
\item
   Ensuring the count of transferred files is always within what we expect:
$\agop{}~\textit{fetched} \ge 0 \land \textit{fetched} \le 386 \land
   \textit{sent} \ge 0 \land \textit{sent} \le 386$
\item
   Checking consistency between the number of sent and fetched files:
   $\agop{}~\textit{fetched} = \textit{sent}$
\item
   Verifying that all files always end up being received:
   $\afop{}~\textit{fetched} = 386$
\end{itemize}
\end{example}

\section{UPPAAL and Networks of Timed Automata}
This section summarizes the differences brought by timed automata and networks
of timed automata to the classical automata described previously. The features
presented here are those found within UPPAAL, a modeling tool for networks of
timed automata introduced in \cite{Bengtsson:1996:UTS:239587.239611}. The
addition of time leads to states in which a new type of variables, clocks,
evolve even when no transition is activated. To account for this,
\textit{states} are now referred to as \textit{locations} instead. Transitions
are all instantaneous. Readers looking for further information on timed
automata are encouraged to check out the papers in which they were first
described (\cite{143902} and \cite{tcs126(2)-AD}), or a more in-depth
introductory course on the subject (\cite{bouyer-cours} and
\cite{cours-raskin}).

\subsection{System Definition}
\begin{definition}[Clocks]
Timed automata feature a special type of variable, called \textit{clocks},
which model the passing of time. Transitions are instantaneous, can reset
clocks (but not set them to a specific value), and have guards referring to the
clock's current value. Within a location, however, time passes at the same rate
for all clocks in the system.
\end{definition}

\begin{definition}[Syntax of Constraints and Actions]
\label{def:formal_methods:transition_grammar2}
Given a set of variables \automatavariables{}, and a set of clocks
\automataclocks{}, the grammar used when writing
constraints and actions in transitions is as follows, with \textbf{ident}
standing for a variable in \automatavariables{}, and \textbf{clk} standing
for a clock in \automataclocks{}:\\
$\textbf{lop} ::= \land~|~\lor $\\
$\textbf{cop} ::=\!\!\! ~< | \le | = | \ge | > $\\
$\textbf{mop} ::= +~| - | * |~/$\\
$
\textbf{val} ::=
   \textbf{ident}
   ~|~ \mathbb{Z}
$\\
$\textbf{mexpr} ::=
   \textbf{mexpr}~\textbf{mop}~\textbf{mexpr}
   ~|~ \textbf{val}
$\\
$
\textbf{bexpr} ::=
   \neg \textbf{bexpr}
   ~|~ \textbf{bexpr}~\textbf{lop}~\textbf{bexpr}
   ~|~ \textbf{mexpr}~\textbf{cop}~\textbf{mexpr}
   ~|~ \textbf{clk}~\textbf{cop}~\textbf{val}
   ~|~ \textbf{clk} - \textbf{clk}~\textbf{cop}~\textbf{val}
   ~|~ \texttt{true}
   ~|~ \texttt{false}
$\\
$\textbf{iexpr} ::=
   \textbf{iexpr} \land \textbf{iexpr}
   ~|~ \textbf{clk}~\textbf{cop}~\textbf{val}
   ~|~ \textbf{clk} - \textbf{clk}~\textbf{cop}~\textbf{val}
   ~|~ \texttt{true}
$\\
$
\textbf{assign} ::=
   \textbf{assign};~\textbf{assign}
   ~|~ \textbf{ident}~:=~\textbf{mexpr}
   ~|~ if~(\textbf{bexpr})~\{\textbf{assign}\}
   ~|~ \textbf{clk} := 0
   ~|~ \texttt{nop}
$
\end{definition}

\begin{definition}[Locations]
Unlike in classic automata, states are referred to as \textit{locations}, since a state is now defined by a location \textit{and} a value for each clock (and a value for each integer variable in our framework).
% This can also be used to force a location to be left
% before a certain amount of time passes.
% \iffalse
% Indeed, without additional constraints
% such as an \texttt{urgent} channel synchronization, there is no obligation for
% activeable transitions to be taken immediately.
%\fi
Time related attributes can be applied to locations:
\begin{description}
\item[\texttt{urgent}:] This location must be left before any time passes.
\item[\texttt{committed}:] This location must be left before any time passes,
and only transition leaving a \texttt{committed} location are enabled.
\item[Invariant in \textbf{iexpr}:] Locations can feature an invariant on
clocks, which must be verified in order for the location to exist.
\end{description}
\end{definition}

The difference between an \texttt{urgent} and a \texttt{committed} location is
only meaningful if there are multiple automata.
Example~\ref{ex:urgent_vs_committed_locations} uses a network of automata to
illustrate the difference between these two attributes.

\begin{definition}[Timed Automata System]
A timed automata system \automatasystem{} is a
$\langle \automatastates{}, \allowbreak{}
\automatainvariants{}, \allowbreak{}
\automatainit{}, \allowbreak{}
\automatacondlabels{}, \allowbreak{}
\automatachanlabels{}, \allowbreak{}
\automatachanpriorities{}, \allowbreak{}
\automatavariables{}, \allowbreak{}
\automataclocks{}, \allowbreak{}
\automataactionlabels{}, \allowbreak{}
\automatarelations{}\rangle$ tuple, where:
\begin{itemize}
\item \automatastates{} is a finite set of locations.
   $\automataurgentlocations \subseteq \automatastates{}$
   denotes the \texttt{urgent} locations, and
   $\automatacommittedlocations \subseteq \automatastates{}$
   the \texttt{committed} ones.
   $\automataurgentlocations \cap \automatacommittedlocations = \emptyset$.
\item
   $\automatainvariants{} : \automatastates{} \to \textbf{iexpr}$ indicates
   the invariant of each location.
\item \automatainit{} is the initial location ($\automatainit{} \in
\automatastates{}$).
\item \automatavariables{} is a finite set of variables.
\item \automataclocks{} is a finite set of clocks.
\item
   $\automatachanlabels{} = \automatachanlabels{}^{\alpha} \cup
   \automatachanlabels{}^{\texttt{sync}}$ is a finite set of labels, with
   $\automatachanlabels{}^{\texttt{sync}}$ corresponding to labels meant for
   synchronization and $\automatachanlabels{}^{\alpha}$ being regular labels,
   with
   $\automatachanlabels{}^{\texttt{sync}} \cap \automatachanlabels{}^{\alpha}
   = \emptyset$.
   The labels in $\automatachanlabels{}^{\texttt{sync}}$ affixed by either `?'
   or `!', with `?' denoting a reception on a ``channel'', and `!' an emission.
   In addition, synchronization labels can be further categorized
   into $\automataurgentchans{} \subseteq \automatachanlabels{}^{\texttt{sync}}$
   corresponding to the \texttt{urgent} channels, and
   $\automatabroadcastchans{} \subseteq \automatachanlabels{}^{\texttt{sync}}$
   corresponding to the \texttt{broadcast} ones.
\item
   $\automatachanpriorities{}:
      \automatachanlabels{}^{\texttt{sync}}
      \to set(\automatachanlabels{}^{\texttt{sync}})$
   indicates, for any label, which labels have a strictly lower priority. It satisfies the following properties:
   $\forallin{l_1,l_2}{\automatachanlabels{}^{\texttt{sync}}}{%
      (l_1 \notin \automatachanpriorities{}(l_1)) \land (l_2 \in \automatachanpriorities{}(l_1))
      \implies
      (
         l_1 \not\in \automatachanpriorities{}(l_2)
         \land
         \automatachanpriorities{}(l_2) \subset \automatachanpriorities{}(l_1)
      )
   }
   $.
\item
   \automatacondlabels{} = \textbf{bexpr}(\automatavariables{},
   \automataclocks{}), as defined in
   Definition~\ref{def:formal_methods:transition_grammar2}.
\item
   \automataactionlabels{} = \textbf{assign}(\automatavariables{},
   \automataclocks{}), as defined in
   Definition~\ref{def:formal_methods:transition_grammar2}.
\item
   $\automatarelations{} \subseteq
      \automatastates{}
      \times \automatacondlabels{}
      \times \automatachanlabels{}
      \times \automataactionlabels{}
      \times \automatastates{}
   $ is the transition relation.
\end{itemize}
The semantics of \automatasystem{} is given via its valid transitions
(see Definition~\ref{sec:system-definition}) and its execution traces
(see Definition~\ref{def:formal_methods:trace2}).
\end{definition}

\begin{definition}[Clock Valuation]
Clocks are kept separate from standard variables, including in the definition of
the valuation. $\automataclockvals{}: \automataclocks \to \mathbb{R}^+$ is the
function mapping each clock to its valuation. As a shorthand, the increment
of the value of all clocks in \automataclockvals{} by $t$ units of time is
written $(\automataclockvals{} + t)$.
\end{definition}

\begin{definition}[Transition]
\label{sec:system-definition}Let $\automatasystem{} = \allowbreak{}
\langle \automatastates{}, \allowbreak{}
\automatainvariants{}, \allowbreak{}
\automatainit{}, \allowbreak{}
\automatacondlabels{}, \allowbreak{}
\automatachanlabels{}, \allowbreak{}
\automatachanpriorities{}, \allowbreak{}
\automatavariables{}, \allowbreak{}
\automataclocks{}, \allowbreak{}
\automataactionlabels{}, \allowbreak{}
\automatarelations{}\rangle$ be an automaton. Given a location $\automatastate{} \in \automatastates{}$, a valuation $\automataenvironment{}$, a clock valuation $\automataclockvals{}$, a duration $t$ and a transition $\langle \automatastate{}, c, l, a, \automatastate{}' \rangle \in \automatarelations{}$, we define the set of the reachable states (without considering priorities) $\automatareach{}(\langle \automatastate{}, \automataenvironment{}, \automataclockvals{}\rangle, \langle \automatastate{}, c, l, a, \automatastate{}' \rangle, t) \triangleq \{\langle \automatastate{}', \automataenvironment{}', \automataclockvals{}' \rangle |
%      (o = \automatastate{})
%      \land (d = \automatastate{}')
      \langle \automataenvironment{}, (\automataclockvals{} + t) \rangle \models_{PL} c
      \land
      \automataenvironment{}' = \automataenvironment{}[a]
      \land
      \automataclockvals{}' = (\automataclockvals{} + t)[a]
      \land
      \langle \automataenvironment{}, \automataclockvals{} \rangle \models_{PL} \automatainvariants{}(\automatastate{})
      \land
      \langle \automataenvironment{}', \automataclockvals{}' \rangle \models_{PL} \automatainvariants{}(\automatastate{}')\}$.

We now define \automatanext{}, which
indicates all valid transitions that can be performed, and takes into account that no transition with higher priority is doable.

$\automatanext{}(\langle \automatastate{}, \automataenvironment{}, \automataclockvals{}\rangle, t)
\triangleq
\{\langle \automatastate{}', \automataenvironment{}', \automataclockvals{}' \rangle |
  \existsin{\langle \automatastate{}, c, l, a, \automatastate{}' \rangle \rangle}{\automatarelations}{\langle \automatastate{}', \automataenvironment{}', \automataclockvals{}' \rangle \in \automatareach{}(\langle \{ \automatastate{}, \automataenvironment{}, \automataclockvals{}\rangle, t) \land \\
      \forallin{\langle \automatastate{b}, c_b, l_b, a_b, \automatastate{b}' \rangle}{\automatarelations{}}{\text{ if } l \in \automatachanpriorities{}(l_b) \text{ then }
\automatareach{}(\langle \{ \automatastate{}, \automataenvironment{}, \automataclockvals{}\rangle, \langle \automatastate{b}, c_b, l_b, a_b, \automatastate{b}' \rangle, t) \text{ is empty.}}
% \forallin{\langle \automatastate{}'', \automataenvironment{}'', \automataclockvals{}'' \rangle}{}{}
%          (\langle \automataenvironment{}, (\automataclockvals{} + t) \rangle \models_{PL} c_b)
%          \land
%          \automataenvironment{}'' = \automataenvironment{}[a_b]
%          \land
%          \automataclockvals{}'' = (\automataclockvals{} + t)[a_b]
%          \land
%          \langle \automataenvironment{b}'', \automataclockvals{b}'' \rangle \models_{PL} \automatainvariants{}(\automatastate{b}')
%          \land
%          \langle \automataenvironment{b}'', \automataclockvals{b}'' \rangle \models_{PL} \automatainvariants{}(\automatastate{b}')
%          \land
%          l \in \automatachanpriorities{}(l_b) \allowbreak{}
%       }
   }\}
$
\end{definition}

\begin{definition}[Path \& Trace]
\label{def:formal_methods:trace2}
We consider a path to be a \emph{maximal} sequence of states/transitions
$\langle \automatastate{1}, \automataenvironment{1}, \automataclockvals{1}\rangle
\automatatransition{}^{\!\!\!t_1} \langle \automatastate{2},
\automataenvironment{2}, \automataclockvals{2}\rangle \automatatransition{}^{\!\!\!t_2} \cdots$ such that $
   \forall i%
      \langle \automatastate{i+1}, \automataenvironment{i+1}, \automataclockvals{i+1}\rangle \in \automatanext{}(\langle \automatastate{i}, \automataenvironment{i}, \automataclockvals{i}\rangle, t)
$. The sequence is maximal in the sense that it is either infinite or of length $N$ and such that $\automatanext{}()$ is empty for any $t\in \mathbb{R}$.
A path starting from $ \langle
\automatainit{}, \automataenvironment{0},  \automataclockvals{0}\rangle$ (where
\automataenvironment{0} is the initial valuation and \automataclockvals{0}
associates each clock with 0) is called a trace.
\end{definition}

\begin{figure}[hbt!]
   \centering
   \begin{tabular}{cc}
   \begin{tikzpicture}[->,>=stealth',shorten >=1pt,auto,node distance=3cm,
                    semithick]
   \node[initial,state] (S0)              {$S_0$};
   \node[state] (S1) [right of=S0] {$S_1$};
   \node[state] (S2) [below of=S0] {$S_2$};
   \node[state] (S3) [below of=S1] {$S_3$};

   \path[every node/.style={sloped, anchor=center, yshift=1em}]
      (S0) edge [bend left] node [above=-1em]{
      \begin{tabular}{l}
         $\textbf{chan0}!$\\
      \end{tabular}
      } (S1)

      (S1) edge [bend left] node [below=1em]{
      \begin{tabular}{l}
         $\textbf{chan1}?$\\
      \end{tabular}
      } (S0)

      (S0) edge [bend right] node [below=1em]{
      \begin{tabular}{l}
         $\textbf{chan2}!$\\
      \end{tabular}
      } (S2)

      (S1) edge [bend left] node [above=-1em]{
      \begin{tabular}{l}
         $\textbf{chan3}?$\\
      \end{tabular}
      } (S3)
   ;
\end{tikzpicture}
 &
   \input{\chapterdirectory/figure/urgent_committed_example_b}
   \end{tabular}\\
   \begin{tikzpicture}[->,>=stealth',shorten >=1pt,auto,node distance=3cm,
                    semithick]
   \node[initial,state] (S0)              {$S_8$};

   \path[every node/.style={sloped, anchor=center, yshift=1em}]
      (S0) edge [loop above] node [above=-1em]{
      \begin{tabular}{l}
         $\textbf{chan3}!$\\
      \end{tabular}
      } (S0)

      (S0) edge [loop below] node [below=1em]{
      \begin{tabular}{l}
         $\textbf{chan2}?$\\
      \end{tabular}
      } (S0)
   ;
\end{tikzpicture}

   \caption{Example of Network of Timed Automata}
   \label{fig:timed_automata_urgent_comitted}
\end{figure}

\begin{example}[Urgent and Committed Locations]
\label{ex:urgent_vs_committed_locations}
Consider the network of timed automata shown in Figure~\ref{fig:timed_automata_urgent_comitted}.
Without any attributes on locations, the following path is legal:
$
   \langle \langle S_0, S_4, S_8 \rangle, \{\},  \{\langle C_0, 0\rangle\}\rangle \allowbreak
   \automatatransitiontrace{\langle \textbf{chan0!}, \textbf{chan0?}, -\rangle}{}^{75}
   \allowbreak
   \langle \langle S_1, S_5, S_8 \rangle, \{\}, \{\langle C_0, 75\rangle\}\rangle \allowbreak
   \automatatransitiontrace{\langle \textbf{chan3?}, -, \textbf{chan3!}\rangle}{}^{39}
   \allowbreak
   \langle \langle S_3, S_5, S_8 \rangle, \{\}, \{\langle C_0, 114\rangle\}\rangle \allowbreak
$
Let us now consider $S_1$ as \texttt{urgent}. The previous
$\langle \langle S_1, S_5, S_8 \rangle, \{\}, \{\langle C_0, 75\rangle\}\rangle \allowbreak
\automatatransitiontrace{\langle \textbf{chan3?}, -, \textbf{chan3!}\rangle}{}^{39}$
transition is no longer legal, as a maximum of $0$ units of time is now allowed
to be stayed at $S_1$.
If we instead considered $S_5$ to be \texttt{committed}, that transition would
still not be allowed to occur after more than $0$ units of time, but in
addition, it would also be illegal because it does not involve a transition
from a \texttt{committed} location (neither $S_1$ nor $S_8$, the two locations
involved in the transition, are \texttt{committed}). Performing
$\langle \langle S_1, S_5, S_8 \rangle, \{\}, \{\langle C_0, 75\rangle\}\rangle \allowbreak
\automatatransitiontrace{\langle -, \textbf{chan3?}, \textbf{chan3!}\rangle}{}^{0}$
instead would be legal (although the resulting state is different).
\end{example}

\begin{definition}[Channel Attributes]
In UPPAAL, all synchronizations have a single emitter transition, but the
number of receivers depends on the type of channel: standard channels have
exactly one receiver, and \texttt{broadcast} channels activate any automaton
able to perform a reception on that channel (even allowing the emitter to
transition alone if no receiver is available). It should be noted that available
receivers for the channel are forced to transition.

Communication channels can have attributes related to time. Indeed, the
\texttt{urgent} attribute indicates that, if a transition featuring this channel
is able to occur, it must do so before any time passes.

In a network of automata, the \automatachanpriorities{} function is shared by
all automata. Thus, the priority of channels is the same accross all automata.
\end{definition}

\begin{example}[Urgent and Broadcast Channels]
\label{ex:urgent_vs_committed_locations}
Consider the network of timed automata shown in Figure~\ref{fig:timed_automata_urgent_comitted}.
Without any attributes on channels, the following paths are legal:
\begin{itemize}
\item
$
   \langle \langle S_0, S_4, S_8 \rangle, \{\}, \{\}\rangle \allowbreak
   \automatatransitiontrace{\langle \textbf{chan0!}, \textbf{chan0?}, -\rangle}{}^{75} \allowbreak
   \langle \langle S_1, S_5, S_8 \rangle, \{\}, \{\}\rangle \allowbreak
   \automatatransitiontrace{\langle \textbf{chan3?}, -, \textbf{chan3!}\rangle}{}^{39} \allowbreak
   \langle \langle S_3, S_5, S_8 \rangle, \{\}, \{\}\rangle \allowbreak
   \automatatransitiontrace{\langle -, \textbf{chan3?}, \textbf{chan3!}\rangle}{}^{42} \allowbreak
   \langle \langle S_3, S_7, S_8 \rangle, \{\}, \{\}\rangle
$
\item
$
   \langle \langle S_0, S_4, S_8 \rangle, \{\}, \{\}\rangle \allowbreak
   \automatatransitiontrace{\langle \textbf{chan2!}, -, \textbf{chan2?}\rangle}{}^{23}
   \allowbreak
   \langle \langle S_2, S_4, S_8 \rangle, \{\}, \{\}\rangle \allowbreak
$
\item
$
   \langle \langle S_0, S_4, S_8 \rangle, \{\}, \{\}\rangle \allowbreak
   \automatatransitiontrace{\langle -, \textbf{chan2!}, \textbf{chan2?}\rangle}{}^{0}
   \allowbreak
   \langle \langle S_0, S_6, S_8 \rangle, \{\}, \{\}\rangle \allowbreak
   \automatatransitiontrace{\langle \textbf{chan2!}, -, \textbf{chan2?}\rangle}{}^{78}
   \allowbreak
   \langle \langle S_2, S_6, S_8 \rangle, \{\}, \{\}\rangle \allowbreak
$
\end{itemize}
Let us now consider \textbf{chan0} as \texttt{urgent}.
$
   \langle \langle S_0, S_4, S_8 \rangle, \{\}, \{\}\rangle \allowbreak
   \automatatransitiontrace{\langle \textbf{chan0!}, \textbf{chan0?}, -\rangle}{}^{75}
   \allowbreak
$
is no longer a legal transition: as the synchronization on \textbf{chan0} is
able to occur, it must do so before any time passes. The transition would be
legal if $75$ was $0$ instead. Furthermore, the transition
$
   \langle \langle S_0, S_4, S_8 \rangle, \{\}, \{\}\rangle \allowbreak
   \automatatransitiontrace{\langle \textbf{chan2!}, -, \textbf{chan2?}\rangle}{}^{23}
$ is also not legal, for the same reason: despite \textbf{chan0} not being
synchronized on, it was still an available synchronization and so no time must
pass while the \textbf{chan0} synchronization is available. Replacing $23$ by
$0$ makes this a legal transition. The third proposed path is still allowed
when \textbf{chan0} is marked as urgent: on the first transition, \textbf{chan0}
can be synchronized on, but no time passes prior to the transition occuring;
on the second transition, \textbf{chan0} can no longer be synchronized on, and
so its \texttt{urgent} attribute is not taken into account.

Let us now consider \textbf{chan3} as a broadcast channel.
$
\langle \langle S_1, S_5, S_8 \rangle, \{\}, \{\}\rangle \allowbreak
   \automatatransitiontrace{\langle \textbf{chan3?}, -, \textbf{chan3!}\rangle}{}^{39} \allowbreak$
is not longer a legal transition: $S_5$ is able to synchronize on \textbf{chan3}
as well, and thus must do so. On the other hand, a previously illegal transition
is now possible:$
\langle \langle S_1, S_5, S_8 \rangle, \{\}, \{\}\rangle \allowbreak
   \automatatransitiontrace{\langle \textbf{chan3?}, \textbf{chan3?}, \textbf{chan3!}\rangle}{}^{39} \allowbreak
$.
Perhaps more surprisingly, the following path is legal:
$
\langle \langle S_0, S_4, S_8 \rangle, \{\}, \{\}\rangle \allowbreak
   \automatatransitiontrace{\langle -, -, \textbf{chan3!}\rangle}{}^{3} \allowbreak
\langle \langle S_0, S_4, S_8 \rangle, \{\}, \{\}\rangle \allowbreak
   \automatatransitiontrace{\langle -, -, \textbf{chan3!}\rangle}{}^{89} \allowbreak
$.
Indeed, all automata able to synchronize (none) do so. Note that only the
receiving label can have any number of occurrences in the transition, there is
always a single emitting label. Thus, setting \textbf{chan2} as a broadcast
channel instead would not cause any change in legal paths.
\end{example}

\begin{figure}[hbt!]
   \centering
   \begin{tabular}{cc}
   \input{\chapterdirectory/figure/timed_automata_a} &
   \input{\chapterdirectory/figure/timed_automata_b}
   \end{tabular}
   \caption{Another example of timed automata}
   \label{fig:timed_automata}
\end{figure}

\begin{example}
\label{ex:timed_automata}
In an evolution from Example~\ref{ex:classical_automata},
Figure~\ref{fig:timed_automata} shows a network of automata modeling the exact
same scenario, but with the addition temporal behaviors. The server now
takes between 32 and 64 time units before providing each file. Transfer times
are thus able to be modeled. To avoid having both automata wait forever in
their $S_0$ location, we consider the \textbf{request\_file} channel as being
\texttt{urgent}. Similarly, the \textbf{done} channel needs to be made urgent
for the $S_1$ locations to be left as soon as all files were transfered. Instead
of having the \textbf{request\_file} channel be \texttt{urgent}, another
solution would be to mark the client's $S_0$ location as \texttt{urgent}. This
would also lead to a deadlock after all transfers have been done.
\end{example}

\iffalse
\begin{definition}[Synchronizations]
In regards to actions performed during a transition, the emitter's actions
are performed prior to that of the receivers. This makes it possible for the
emitter to store a value in a variable, and have the receiver read that new
value within a single step, much like a message being passed.
\end{definition}
\fi


\subsection{Query Logic Operators and Semantics}
\label{sec:uppaal_queries}
UPPAAL uses the temporal operators described in
Section~\ref{sec:automata_query_logic}. In this case however, $\phi$ is
a
$
\textbf{bexpr} ::= \allowbreak
   \neg \textbf{bexpr} \allowbreak
   ~|~ \textbf{bexpr}~\textbf{lop}~\textbf{bexpr} \allowbreak
   ~|~ \textbf{mexpr}~\textbf{cop}~\textbf{mexpr} \allowbreak
   ~|~ \textbf{clk}~\textbf{cop}~\textbf{val} \allowbreak
   ~|~ \textbf{clk} - \textbf{clk}~\textbf{cop}~\textbf{val} \allowbreak
   ~|~ \texttt{true} \allowbreak
   ~|~ \texttt{false} \allowbreak
   ~|~ \texttt{deadlock} \allowbreak
   ~|~ \textbf{sub-automaton}.\textbf{location} \allowbreak
$,
with \texttt{deadlock} being true if and only if no transition is able to occur
and time is not allowed to pass, and \textbf{sub-automaton}.\textbf{location}
being true if and only if said sub-automaton is currently in the given
location.

In addition to these temporal operators, UPPAAL features operators that seek
the extremum of either a clock or a variable, with the possibility of an
invariant delimiting the system locations in which the value is considered. Given
either a variable or a clock $t$,
$\texttt{sup}\{\phi\}: t \triangleq$
maximum value for $t$ across all traces, such that this value has been reached
in a state satisfying $\phi$.\\
The $\texttt{inf}$ operator is the equivalent for finding the minimum value.

\begin{example}
In Example~\ref{ex:timed_automata}, computation of the WCET for file fetching
can now be achieved through model checking:
$\texttt{sup}\{\textit{client}.S_1\}: C_0$.  This query looks for the maximum
value reached by $C_0$ when the \textit{client} automaton is in the $S_1$
location.
Considering $max$ to be the returned value, a trace leading to that value could
be obtained by using UPPAAL to query
for $\agop{}~\neg((C_0 = max) \land client.S1)$
\end{example}

\section{Conclusion}
This chapter has shown how UPPAAL's model checking capabilities can be exploited
to analyze the interference caused by cache coherence on the model from
Chapter~\ref{cha:modeling_cache_coherence}.

This analysis starts by a computation of the WCET for each program. Useful in
itself, this analysis is extended by that of the WCET for these programs with
the architecture in different configurations in order to extract more
information about how much of the execution time is caused by interference.

In order to more precisely understand what determine the WCET and to provide
the user with information about elements of the program that can directly be
manipulated, the analysis proceeds by an categorization of the accuracy of each
instruction. This indicates which instructions are unaffected by the
interference, which instructions are always time-consuming, and which
instructions take a varying amount of time depending on the execution. By
looking at the accuracy of all accesses made on each memory element, patterns
for these instructions of varying execution time can sometimes be found, which
results in a more predictable system.

The determining factor for the accuracy of instructions is then properly
defined. This corresponds to a categorization of all external queries depending
on their effects on the permissions held by a cache, and whether a loss of
permission led to an instruction taking additional time. Thus, three categories
of interference are defined: minor (no change of permission, but loss of time
due to query processing), demoting (loss of writing permission), and expelling
(loss of all permissions).

Finally, analyses are performed in order to determine how each instruction
interfere with the other instructions. This results in a graph showing, for
each instruction, which instruction can generate interference that will
directly impact it, the category of this interference, and whether this
interference occurs on all possible executions or not.

This provides the user with a clear understanding of the causes and effects of
cache coherence interference on the programs' instructions, opening the way to
finding means of mitigation for this interference.

