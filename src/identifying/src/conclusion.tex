\section{Conclusion}
This chapter presented an approach to the identification of the cache coherence
mechanisms implemented by an architecture. Such a step is necessary to expose
and resolve any ambiguities the user may have on those mechanisms. It does not
perform an exhaustive profiling of speed and bandwidth offered by the cache
coherence, but instead ensures that the user has a complete list of the
mechanisms for which to perform this profiling.

The application of this approach on the NXP QorIQ T4240 fully vindicated the
need for this additional step in the profiling of an architecture. Indeed, the
results showed that the difference between what I believed the architecture to
implement and what it actually does are far beyond the level of negligible
detail: an additional coherence state was found. Without realizing that this
state exists, the speed and bandwidth analysis may have obtained false results
by attributing performance measurements to the wrong state and thus incorrectly
profiling the architecture without realizing it.

There were in fact multiple contributions in this chapter: the approach itself,
which is the main contribution; the NXP QorIQ T4240 example, which serves as a
warning not to overlook the necessity of such a step; and the formalization of
two protocols, which are minor contributions in themselves.

Having reached the end of this chapter, the issue of detecting sources of
interference in the cache coherence mechanisms is resolved. Being able to
determine the effects of this interference on a running program will require
tools. The approach chosen in this thesis is to rely on models. The next chapter
thus introduces how a model of the system is created.
