The crudest approach to the handling of interference generated by cache
coherence is to not have any because all caches are disabled. While this
solution tremendously improves the predictability of the system, it also
tremendously decreases its execution speed, to the point where it may be
preferable to use a single-core architecture with caches instead.

One step above is a solution in which the caches are enabled, but their content
is locked, making their usefulness severely limited but without compromising the
system's predictability.

In this section are presented strategies that allow the use of caches in a
limited manner in order to achieve reasonable execution speed while keeping the
system as predictable as possible.
