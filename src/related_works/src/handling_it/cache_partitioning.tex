\subsection{Shared Cache Partitioning}
\cite{10.1145/1629335.1629369} presents two algorithms for a condition
sufficient to prove schedulability, it is meant for tasks with time and cache
space constraints in the context of a shared L2 cache in which space is
partitioned in order to avoid contentions.  Thus, this work is more about
proving that the measures taken in order to address interference are sufficient
rather than a strategy to control the interference itself. For these
schedulability tests, the tasks are assumed to be non-preemptive and to have a
fixed priority.

The paper's first algorithm is based on constraint programming. It basically
considers that there is a task missing its deadline, which implies that either
all cores are being used, or that too little cache space is available.
Intervals at which either of these conditions is true are searched for in order
to find their maximal impact of the hypothetical task that missed its deadline.
If this maximum impact is lower than the slack (longest affordable delay) of
this hypothetical task, then the tasks are schedulable. The second algorithm is
very similar, but simplifies the constraints by simply assuming the worst
possible interference from the other tasks in all criteria with much less
regard for whether the configuration leading to this interference is actually
possible. This effectively creates a more scalable schedulability test, at the
cost of being much more pessimistic.
