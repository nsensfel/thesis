\newpage{}
\section{Variants}
This section presents the most common variants of the MSI. In effect, the main
factor of a cache coherence protocol's efficiency is how frequently it needs to
fetch or write data to the main memory. As a result, it is unsurprising to see
that all of these optimizations are focused on removing that necessity in
certain common scenarios. Each of them adds a new stable state, corresponding
to a new permission or new responsibility, and each one can be combined with
the others to form a new variant (forming for example, the MESI, MOESI, and
MESIF protocols).

\paragraph{Exclusive State}
The Exclusive state (\texttt{E}) is equivalent to the \texttt{S} state, with the
added information that no other cache currently holds any copy of the memory
element. As a result, the cache modifying the value of its copy of the memory
element without broadcasting for new permission does not violate
Properties~\ref{prop:system_wide_value} and \ref{prop:swmr}. It does, however,
mean that the coherence manager must be able to detect that no caches hold any
copy of the memory element, which implies some change in how caches evict
from the \texttt{S} state, as they otherwise need not do inform any other
component.

\paragraph{Owned State}
The Owned state (\texttt{O}) allows a cache to keep its modified copy of a
memory element without having to perform a write-back when another cache asks
for a read-only copy of that memory element.

\paragraph{Forward State}
The Forward state (\texttt{F}) is equivalent to an \texttt{S} state where the
cache is also in charge of providing a copy of the memory element to any cache
that demand it.
