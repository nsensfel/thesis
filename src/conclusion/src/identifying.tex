\section{Identifying the Protocol}
The details of cache coherence mechanisms provided in an architecture's
documentation do not go into sufficient details for certification purposes. The
documentation can even be vague enough to cause the applicant to be misled
about which protocol is implemented on the architecture.  To remedy this, the
first contribution made in this thesis is the cache coherence protocol
identification process described in
Chapter~\ref{cha:identifying_cache_coherence}.

\subsection{Summary}
The proposed cache coherence protocol identification process relies on being
able to observe the binary flags defining the state of cache lines, as well as
having sufficient performance monitors to observe cache coherence related
activity (although solutions such as the one presented in
Section~\ref{sec:elusive_hardware_monitors} may provide an alternative). The
general idea being to use the documentation as a basis for the creation of a
detailed hypothetical cache coherence, then validating it against observations
made on the architecture by performing micro-benchmarks. These benchmarks first
perform a reachability analysis on cache states, then the hypothetical cache
coherence is used to generate benchmarks corresponding to behaviors not exposed
by the initial reachability analysis.

Successful application of the proposed strategy results in an ambiguity-free
description of the architecture's cache coherence protocol (meaning a
description following the notations presented in
Chapter~\ref{cha:cache_coherence}).  This ambiguity-free description of the
architecture's cache coherence protocol is an important information to have in
order to prove that the effects of cache coherence on the system are under
control. Indeed, performing benchmarks to measure access latency or bandwidth
without this information is likely to result in the attribution of
characteristics to the application of instructions on an mistaken system state.
For example considering writing speed to be a certain value for memory elements
seemingly in \texttt{Shared} in all caches of the system realizing that the
protocol actually has a \texttt{Forward} state that will improve access speed.

To illustrate the need for cache coherence identification, the case of the NXP
QorIQ T4240 is presented in Chapter~\ref{cha:identifying_cache_coherence}.
This architecture's documentation files indicate a MESI protocol (in the core
documentation \cite{e6500}) with \textit{cache intervention} (in the
motherboard family's documentation \cite{T4240}). By applying of the cache
coherence protocol identification process, the architecture is shown to
implement a MESIF protocol, which is fair from being obvious from the
aforementioned documentation.

\subsection{Limitations}
There are two main limitations to the cache coherence protocol identification
strategy: the risk of an unreasonably large search space during the naive
reachability analysis and the possibility of undetected
behaviors.

The proposed naive reachability analysis performs up to $3 * \cachecount{}$,
benchmarks per observed system state (i.e.~each instruction on each core, if no
caches have the same state). Unfortunately, the number of observed system states
can be needlessly high if the considered binary flags of cache lines contain
quickly changing information not pertinent to cache coherence. For example, if
flags meant to provide information to the cache replacement policy are
unwittingly considered, the number of observed system states is likely to be
exceedingly high.

The other limitation is that it is always possible for the architecture to have
behaviors that were not observed by the benchmarks. Indeed, provided the system
state is actually defined by components that cannot be observed, the naive
reachability analysis might not encounter them. Likewise, nothing can be done
to ensure the benchmarks guided by the hypothetical protocol would expose them
either.

\subsection{Future Works}
The identification process would greatly benefit from the addition of a step
in which the cache's binary flags are analyzed in order to remove what has no
chance of being related to cache coherence.

The \textit{Naught} library currently only performs a single benchmark per
execution. It should be possible to improve it in order to have all benchmarks
for an observable system state be performed in a single execution.

\textit{Naught} could further be improved to perform the complete
identification process in a single execution, if given the ability to look at
the binary flags of cache lines. This would render the application of this
identification process trivial and greatly improve its chances at being adopted
as a common practice.
