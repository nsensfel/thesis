\stopallthesefloats
\section{Model Checking Scalability Considerations}
The analyses presented in this chapter rely on model checking. To validate that
the model verifies a property, model checking gradually generates and explores
all possible traces from the model. The time required for this exploration is
the main limiting factor of the analyses of the chapter.

Indeed, checking the model described in
Chapter~\ref{cha:modeling_cache_coherence} can become unreasonably time
consuming. A part of the issue stems from the fact that, despite the efforts to
reduce the amount of undesired non-determinism described in the previous
chapter, some of it remains.

Instead of going over each property presented in this chapter, it uses the
property that requires the largest exploration: In all traces, there is a state
in which all cores have completed their program. This corresponds to the
following formula:
\begin{lstlisting}
A<>(Core1.terminated and Core2.terminated)
\end{lstlisting}

\begin{figure}
\begin{center}
\begin{tabular}{|l|c|c|c|}
\hline
Cores                & 2      & 3     & 4\\
\hline
Execution time (s)   & 0.593  & 12.453   & 425.109\\
\hline
\end{tabular}
\end{center}
\caption{Model Checking execution time relative to number of cores}
\label{fig:exec_time_cores}
\end{figure}

Figure~\ref{fig:exec_time_cores} shows the execution time of the model checking
process with varying number of cores in the modeled architecture. The 2 cores
column corresponds to the model described in
Section~\ref{sec:analysis:demo_model}. The cores added for the 3 and 4 cores
benchmarks are clones of the second core. The results clearly show that
increasing the number of cores will quickly reach execution times beyond usable
levels. This is not unexpected: even useful non-determinism such as the order
of arbitration for the interconnect generate by itself a number of traces
exponential to the number of cores (since all possible orders are explored).

\begin{figure}
\begin{center}
\begin{tabular}{|l|c|c|c|c|}
\hline
Program Size         & 10      & 20      & 30     & 40\\
\hline
Execution time (s)   & 0.593   & 1.297   & 1.906  & 2.266\\
\hline
\end{tabular}
\end{center}
\caption{Model Checking execution time relative to length of programs}
\label{fig:exec_time_program}
\end{figure}

Figure~\ref{fig:exec_time_program} shows how long model checking takes on the
dual core from Section~\ref{sec:analysis:demo_model} depending on the number of
instructions for the program on each core. 10 instructions corresponds to the
original number of instructions. In order to add more instructions, the programs
were extended by appending a copy of all their current instructions. The results
show that increasing the length of programs does not lead to a rapid increase of
model checking times.

\stopallthesefloats
