\section{Analyzing Impact on Program Execution Time}
\label{sec:analysis:wcet}
This first analysis looks at the effects of interference on the programs
themselves by computing their worst-case execution time. Note that when using
the limited representation of the programs and of the architecture in the model
proposed in Chapter~\ref{cha:modeling_cache_coherence}, the execution times
resulting from the analyses performed in this chapter are unlikely to be those
of the real applications and should therefore only be used in comparisons with
other similarly obtained execution times. Indeed, such results can be used to
compare the execution time of programs in different configurations, such as for
computation of slowdown factors (see Definition~\ref{def:slowdown-factor}).
Additionally, it is possible to obtain an estimate of the execution time that
is due to cache coherence interference by comparing the execution time of each
program with that of a configuration in which there are no shared variables.
This configuration might not be viable in the real system, but its analysis
provides valuable information as a reference point, since it corresponds to the
programs neither benefiting nor suffering from cache coherence while still
having to account for concurrent accesses to the system's main memory.

When looking at the execution time of programs without shared variables,
the results do not include much of the interference from cache coherence, but
still has the cost of concurrent memory accesses. This particular cost may
actually be lowered by cache coherence. Indeed, having programs running in
parallel and sharing data using cache coherence can lead to run-times
increasing because cache permissions are lost, but it can also lead to
run-times decreasing because more values are available in caches and may thus
be retrieved faster than when those are only in the main memory.

By removing the execution time obtained by the analysis of each programs without
shared variables from their execution time with those shared variables, the time
lost or gained because of cache coherence is obtained. More precisely, given
$W_{s}$ the execution time of a program running in parallel with other programs
and having shared variable and $W_{p}$ the equivalent without any shared
variables, $T_{cc} = W_{s} - W_{p}$ corresponds to the execution time added by
cache coherence interference.

To compute the $W_{s}$ and $W_{p}$ using the model described in
Chapter~\ref{cha:modeling_cache_coherence}, UPPAAL's model checking is made to
find the maximum value of the \lstinline{runtime} clock of a core automaton
outside of the \textit{Terminated} location
(e.g.~\lstinline!sup{not Core1.Terminated}: Core1.runtime! for \texttt{Core1}).

%\todomsg{Beware, lstinline has a weird font in these environments}
\begin{example}[Slowdown Factors on the Model from Section~\ref{sec:analysis:demo_model}]
Figure~\ref{fig:analysis:wcet_calc} shows different worst case execution times
for the model from Section~\ref{sec:analysis:demo_model}. The first line
corresponds to the result of \lstinline!sup{not Core1.Terminated}: Core1.runtime!
and the second is the equivalent for the other core. The \textit{Shared
Variables} results are obtained by simply making this query on the instantiated
model from Section~\ref{sec:analysis:demo_model} as-is, with both programs
running in parallel unmodified. The \textit{No Shared Variables} results are
obtained by shifting all addresses from the program running on Core2 in order
to avoid sharing any address between the two cores. In this case, all addresses
from Figure~\ref{fig:analysis:demo_prog2} are increased by $3$. $T_{cc}$
corresponds to the execution time which is due to by cache coherence
interference. Lastly, an example of alternative configuration in which each
program runs alone on the platform is also analyzed: the \textit{Isolation}
configuration is achieved by replacing the other program with one composed only
of a single \lstinline!INSTR_END! instruction.

\begin{figure}[hbt!]
\centering
\begin{tabular}{|c|c|c|c|c|}
\cline{2-5}
\multicolumn{1}{c|}{}
      & Shared Variables & No Shared Variables & $T_{cc}$ & Isolation \\
\hline
Core1 & 2652 & 1102 & 1550 & 702\\
\hline
Core2 & 2452 & 1452 & 1000 & 904\\
\hline
\end{tabular}
\caption{WCET Analysis of Model from Section~\ref{sec:analysis:demo_model}}
\label{fig:analysis:wcet_calc}
\end{figure}

The results of the analysis are as follows:
\begin{itemize}
\item
   Core1 suffers a slowdown factor of $2652/702 = 3.77$ when running in
   parallel with Core2, compared to in isolation.
\item
   Core2 suffers a slowdown factor of $2452/904 = 2.71$ when running in
   parallel with Core1, compared to in isolation.
\item
   Running the two programs in isolation one after the other has a maximum
   execution time of $702 + 904 = 1606$.
\item
   Running the two programs with their shared variables in parallel has a
   maximum execution time of $\max(2652, 2452) = 2652$.
\item
   Running the two programs without shared variables in parallel has a
   maximum execution time of $\max(1102, 1452) = 1452$.
\item
   Approximately $(1550/2652)*100=58.44$ percent of Core1's worst execution time
   is caused by cache coherence interference.
\item
   Approximately $(1000/2452)*100=40.78$ percent of Core2's worst execution time
   is caused by cache coherence interference.
\end{itemize}

The following observations can thus be made:
\begin{itemize}
\item The programs in this example greatly interfere with each other.
\item
   Running the two programs in isolation one after the other actually leads to
   a WCET lower than running both in parallel with shared variables:
   $1606 < 2652$. As a result, if the nature of the programs allow it, running
   the programs one after the other is preferable.
\item
   Running the two programs in parallel but with no shared variables is
   preferable to running them in isolation one after the other ($1452 < 1606$).
   This is also a result which is only exploitable if the nature of the programs
   allow them to be transformed into an alternative that either uses separate
   copies of the shared variables or uses less shared variables altogether.
\end{itemize}
\end{example}

Thus, comparing WCET in different configurations of the architecture provides
the user with an understanding of how much interference is affecting the
programs because of cache coherence. However, this WCET analysis shows the
accumulation of all interference. In order to better control this interference,
a more precise analysis of the causes of this interference has to be performed.
This starts with determining which instructions are impacted by interference.
