\chapter{Model Parameters}
\label{app:model_parameters}
In order to tailor the model to match the user's architecture, a number of
parameters can be modified:

\begin{itemize}
\item
   \lstinline!LAST_ADDR!, an integer, which, in effect, corresponds to the
   number of memory elements used by the system. The memory element $0$ is
   reserved as a default \lstinline!NULL! value.
\item
   \lstinline!CORE_COUNT!, an integer corresponding to the number of caches
   present in the system. The name comes from the limitation to a single core
   per cache.
\item
   \lstinline!LINES_PER_CACHE!, an integer indicating the number of memory
   elements that can be held in each cache.
\item
   \lstinline!COMPONENT_COUNT! an integer equal to the highest component ID
   value.
\item
   \lstinline!USE_LOCK_FREE_CACHES! is a Boolean controling whether cores can
   send new instructions before the previous ones have been resolved. If they
   are allowed to, the value is set to \lstinline!true!.
\item
   \lstinline!REQ_BUFFER_SIZE! is an integer indicating how many pending
   requests a cache can handle simultaneously. Because of how automated eviction
   is handled, the minimal value is $2$.
\item
   \lstinline!IN_QUERY_BUFFER_SIZE! is the number of slots available in an
   incoming query FIFO queue.
\item
   \lstinline!OUT_QUERY_BUFFER_SIZE! is the number of slots available in an
   outgoing query FIFO queue.
\item
   \lstinline!IN_DATA_BUFFER_SIZE! is the number of slots available in an
   incoming data FIFO queue.
\item
   \lstinline!OUT_DATA_BUFFER_SIZE! is the number of slots available in an
   outgoing data FIFO queue.
\end{itemize}

Furthermore, the time required for certain operations can be set:
\begin{itemize}
\item
   \lstinline!RAM_READ_TIME! is the time during which the memory controller
   is inactive before sending the queried memory element.
\item
   \lstinline!RAM_WRITE_TIME! is the time during which the memory controller
   is inactive after having updated a memory element.
\item
   \lstinline!QUERY_HANDLING_TIME! is the inactivity period of a cache receiving
   a new query.
\item
   \lstinline!DATA_HANDLING_TIME! is the inactivity period of a cache receiving
   a new data message.
\item
   \lstinline!REQUEST_HANDLING_TIME! is the inactivity period of a cache
   receiving a new request from its core.
\item
   \lstinline!DATA_TRANSFER_TIME! is the delay for transfer through the bus
   of a data message from one component to another.
\item
   \lstinline!QUERY_TRANSFER_TIME! is the time needed for a query to be
   broadcasted by the bus.
\item
   \lstinline!CLOCK_CYCLE_TIME! is how long a core stays inactive after sending
   an instruction.
\end{itemize}

By default, the model uses the values indicated in
Figure~\ref{fig:analysis:demo_params}, which do not correspond to any
architecture in particular and were arbitrarily chosen to be small and vaguely
realistic in their proportion to one another.

\begin{figure}[hbt!]
\begin{multicols}{2}
\begin{itemize}
\item \lstinline!LINES_PER_CACHE = 20!
\item \lstinline!USE_LOCK_FREE_CACHES = true!
\item \lstinline!RAM_READ_TIME = 200!
\item \lstinline!RAM_WRITE_TIME = 300!
\item \lstinline!QUERY_HANDLING_TIME = 4!
\item \lstinline!DATA_HANDLING_TIME = 5!
\item \lstinline!REQUEST_HANDLING_TIME = 6!
\item \lstinline!DATA_TRANSFER_TIME = 17!
\item \lstinline!QUERY_TRANSFER_TIME = 24!
\item \lstinline!CLOCK_CYCLE_TIME = 50!
\item \lstinline!REQ_BUFFER_SIZE = 3!
\item \lstinline!IN_QUERY_BUFFER_SIZE = 5!
\item \lstinline!OUT_QUERY_BUFFER_SIZE = 5!
\item \lstinline!IN_DATA_BUFFER_SIZE = 6!
\item \lstinline!OUT_DATA_BUFFER_SIZE = 6!
\end{itemize}
\end{multicols}
\caption{Default Model Parameters}
\label{fig:analysis:demo_params}
\end{figure}
