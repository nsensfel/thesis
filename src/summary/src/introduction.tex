\section{Introduction}
\subsection{Contexte}
The ever increasing complexity of aircraft and the market's depreciation of
single-core processors are motivating the introduction of multi-core processors
in aeronautical systems.

The operation of a safety critical system requires its certification by the
relevant authorities. This certification is obtained through a process in which
an applicant argues for the compliance of that system with regulation. This
thesis is part of the Phylog project. The objective of the Phylog project
is to provide tools that will help building a strong case for applicants
attempting to pass the certification process of an aeronautical computer system.
The requirements that the applicants must prove this computer system passes
include those listed in the CAST-32A (\cite{cast32}). This document focuses on
the particularities of multi-core processors and the way these particularities
complicate the demonstration of both safety and performance standard objectives
fulfillment. Among the requirements listed in the CAST-32A figures
\textit{Resource Usage 3}, which requires the complete identification of all
interference and its effects with the chosen configuration: \textit{The
applicant has identified the interference channels that could permit
interference to affect the software applications hosted on the MCP cores, and
has verified the applicant's chosen means of mitigation of the interference.}
To help applicants fulfill this objective, this
thesis focuses on interference generated by a prevalent feature of multi-core
processors: cache coherence.

\begin{definition}[Interference]
An interference is the unwarranted modification of the execution time of an
application because of the actions of another.
\end{definition}

\section{Vue d'ensemble du r\'esum\'e}
\todomsg{Turn this into way fewer words}
This thesis starts by introducing prerequisites: timed automata
(Chapter~\ref{cha:formal_methods}) and cache
coherence (Chapter~\ref{cha:cache_coherence}).
Once these have
been presented, the focus of this thesis can be explained in full
(Chapter~\ref{cha:second_intro}).
Indeed, The purpose of the thesis is to
develop a framework to ensure the applicant is made aware of the interference
generated by cache coherence in their chosen COTS multi-core processor.

To determine the state of the art and what specifically needs to be developed,
architecture profiling, current practices for use of
caches in critical environments, solution relying on formal methods.

After clarifying what is left to be done to achieve a full framework that will
help with the cache coherence part of the certification, this thesis proposes
three contributions: a strategy to properly identify an architecture's cache
coherence protocol, a model template for multi-core architectures with cache
coherence support, and analyses to be performed on instantiated models in order
to expose the interference.
