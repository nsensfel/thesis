This thesis proposes tools to help in the certification of multi-core
processors for use in aeronautical systems. While the parallel nature of
multi-core processors can greatly improve computation speeds, it also makes
them difficult to predict, preventing their use in critical environments.
Indeed, in such processors, the cores share access to nearly all resources and
this causes conflicts, or interference, which lead to seemingly random
variation in the execution time. Among the complex mechanisms prone to
interference is cache coherence, which ensures that cores that use a same
atomic memory block cannot blindly override the modifications made by another
core and that all cores are made aware of all modifications. To achieve cache
coherence, the processor automatically follows a predetermined protocol which
defines messages to be generated according to the actions of a core, as well as
the actions to be performed when another core’s message is received.


The focus of this thesis is to identify the interference generated by the cache
coherence mechanisms and provide a way to predict their effects on the
applications, as a first step toward their mitigation. The first contribution
made is to address the ambiguities in the understanding applicants have of the
coherence protocol implemented on their chosen architecture. Indeed,
architecture documentation does not generally offer sufficient details on their
cache coherence protocol. This thesis proposes a formalization of some standard
cache protocols and a strategy relying on micro-benchmarks in order to clarify
the implementation details of the architecture’s protocol. This approach is
applied to the NXP QorIQ T4240 architecture. Once the protocol has been
correctly identified, the second contribution consists in the making of a
low-level description of the architecture using timed automata in order to
adequately represent the micro-behaviors and understand clearly how the cache
coherence protocol acts. In effect, a generic model generation framework has
been developed, capable of handling cache coherence protocols as described by
the applicant, and to support architectures with different configurations in
order to better fit the applicant’s chosen architecture. The third contribution
explains how to make use of the timed automata low-level representation of the
architecture to expose interference. It proposes a strategy to detail the
causes and effects of cache coherence interference on the given programs.
Starting from a simple analysis of execution time, the results go down to
instruction level, indicating how each instruction generates and suffers from
interference. This is intended to provide sufficient information on cache
coherence interference to the applicant, both for the purposes of certification
and to form the base upon which a mitigation strategy can be started.

In effect, this thesis provides the applicant with the means to understand the
cache coherence mechanisms used by their chosen architecture and to expose the
interference they generate.
