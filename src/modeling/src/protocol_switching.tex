\section{Switching Coherence Protocol}
\label{sec:protocol_switching}
To make the model more easily adapted to a different architecture,
it is accompanied by a tool called CoProSwi, that makes it possible to switch
to a different cache coherence protocol automatically.

This tool takes as input the model and a description of
the target cache coherence protocol under the form of a text file describing
the tables that define the behavior of caches and of the coherence manager (as
in Figures~\ref{fig:msi_cc_table} and \ref{fig:msi_mgr_table}).

Protocols described for CoProSwi indicate:
\begin{itemize}
\item A list of data message types. For example, \lstinline!(add_data_type DATA_MSG)!.
\item
   A list of query message types. For example,
   \lstinline!(add_data_type GET_SHARED)!.
\item A definition of the cache controller's behavior.
\item A definition of the coherence manager's behavior.
\end{itemize}

A notable omission are the instructions: CoProSwi assumes all protocols are
defined around the \loadinstr{}, \storeinstr{}, and \evictinstr{} instructions.
The \ownquery{} event is also implicitely declared.

When defining either the behavior of either the cache controller or the
coherence manager:
\begin{itemize}
\item
   Each coherence state is declared. These declarations indicate whether the
   state is \lstinline!transient! or \lstinline!stable!. For example,
   \lstinline!(add_state stable MSI_INVALID)! declares the \textit{Invalid}
   stable state.
\item
   The default coherence state for memory elements is also indicated. It is
   considered to be the \textit{Invalid} state's representation. For example,
   \lstinline!(set_default_state MSI_INVALID)! sets \lstinline!MSI_INVALID!  as
   the \textit{Invalid} state. The state must have been declared beforehand.
\item
   For each state, the list of actions to performed upon observation of an
   event is then defined. In the case of the coherence's manager definition,
   a \lstinline!if_is_owner! operator is added, allowing the definition of two
   different action lists according to whether the message received came from
   the component currently set as the memory element's owner or not.
\item
   If no actions are to be taken, the \lstinline!(none)! action must be
   indicated. Indeed, the tool assumes that if a case is missing, it is because
   of an user error, and not because no actions are to be taken.
\item
   The coherence manager only handles query and data message events, not
   instructions nor reception of its own query (since it assumed to be unable
   to send any).
\end{itemize}

CoProSwi takes care of all the otherwise tedious aspects of describing the
protocol directly in UPPAAL, such as generating separate actions to be
performed during the un-stalling process in caches (in order to correctly
manage request queues). In effect, no knowledge of the model's
inner workings is needed to describe the protocol or to use CoProSwi in
general, since the protocol definition only uses notions from
Chapter~\ref{cha:cache_coherence}.

The modification of the model file is done through simple search and replace:
only the automata's actions and variables needs to be modified. Locations and
transitions are neither added nor removed. Thus, the model's file contains tags
that points to the code that needs modification by the tool upon switching of
protocol.

\begin{figure}[hbt!]
\begin{center}
\begin{tabular}{c}
\begin{lstlisting}
/*[CMGR_STATES_DECLARATION]*/
const mem_state_t MEM_I = 0;
const mem_state_t MEM_I_D = 1;
const mem_state_t MEM_S = 2;
const mem_state_t MEM_I_O_S_B = 3;
const mem_state_t MEM_S_D = 4;
const mem_state_t MEM_M = 5;
/*[/CMGR_STATES_DECLARATION]*/
\end{lstlisting}
\end{tabular}
\end{center}
\caption{Example of Tagged UPPAAL Code}
\label{fig:UPPAAL:tag_example}
\end{figure}

Figure~\ref{fig:UPPAAL:tag_example} provides an example of tagged code section.
This particular snippet corresponds to the coherence manager's states
declaration.  When changing the coherence protocol, CoProSwi replaces the
content found between the \lstinline!CMGR_STATES_DECLARATION! tags so that it
matches the new protocol.  As a result, the UPPAAL file can be easily modified
by the user without having to wonder what will break compatibility with
CoProSwi. Furthermore, this means there is no master template: the output of
CoProSwi can be re-used as-is if the protocol needs to be changed again.

In addition to being able to modify the model, CoProSwi is also able to list
all stable state change paths. This is useful when applying the coherence
protocol identification process described in
Chapter~\ref{cha:identifying_cache_coherence}.
