\section{Cache Line Organization}
Caches hold a finite amount of cache lines. Thus, at some point, the need for a
new cache line when none are available may arise. Deciding which cache line to
evict in order to store the new one has a very important impact on performance.
This is typically decided by two factors: its placement policy and its
replacement policy. These two policies interact: the placement policy determines
which group of cache lines are considered by the replacement policy.

\subsection{Replacement Policies}
The cache eviction policy is the algorithm used to determine which line to
evict when there is no more room. The optimal strategy would ensure that the
line that is evicted is the one that would not be used for the longest time.
However, being able to determine which line this is would require explicit
management of the caches.

Since it is commonly assumed that data recently accessed is likely to be
accessed again soon (a principle known as \textit{locality of reference}), the
most common replacement strategy is to choose the least recently used cache
line as the one to evict. This is referred to as the \textit{LRU} policy.
However, keeping accurate track of which line was accessed last is costly in
both space and time, and so, in practice, a less precise approach, the
Pseudo-LRU (or \textit{PLRU}) is implemented instead. The principle behind
\textit{PLRU} is similar to having a binary tree, with the leaves corresponding
to cache lines. Each node of the tree has a single bit. This bit indicates which
of the two children to follow to find a cache line that was not recently used.
When using any of the cache lines, all nodes in the path leading to it are set
so that they point to a node not also in the path.

Other replacement policies include \textit{FIFO}, in which the cache line that
was the least recently \textit{allocated} is evicted, and cache line uses have
no impact; using frequency of use instead of date, in something similar to the
LRU policy (called \textit{LFU}, for Least-Frequently Used); and random
replacement (\textit{RR}).

These policies are applied to a set of cache lines. Exactly what lines are part
of this set is determined by the placement policy.

\begin{figure}[hbt!]
\begin{center}

\begin{tikzpicture}[->,>=stealth',shorten >=1pt,auto,  semithick]
\tikzstyle{address} = [state,rectangle,  node distance = 0.5cm and 2cm]
\tikzstyle{cacheline} = [state,rectangle, node distance = 0.5cm and 2cm]

\node[cacheline] (CL0) {Cache Line 0: A [3]};
\node[cacheline] (CL1) [below = of CL0] {Cache Line 1: B [2]};
\node[cacheline] (CL2) [below = of CL1] {Cache Line 2: C [1]};
\node[cacheline] (CL3) [below = of CL2] {Cache Line 3: D [0]};

\node[] (sep0) [right = of CL1,xshift=-1em,yshift=-2em] {$ \Longrightarrow $};

\node[cacheline] (CL0b) [right = of CL0] {Cache Line 0: E [0]};
\node[cacheline] (CL1b) [below = of CL0b] {Cache Line 1: B [3]};
\node[cacheline] (CL2b) [below = of CL1b] {Cache Line 2: C [2]};
\node[cacheline] (CL3c) [below = of CL2b] {Cache Line 3: D [1]};

\node[] (sep1) [right = of CL1b,xshift=-1em,yshift=-2em] {$ \Longrightarrow $};

\node[cacheline] (CL0c) [right = of CL0b] {Cache Line 0: E [1]};
\node[cacheline] (CL1c) [below = of CL0c] {Cache Line 1: B [3]};
\node[cacheline] (CL2c) [below = of CL1c] {Cache Line 2: C [0]};
\node[cacheline] (CL3c) [below = of CL2c] {Cache Line 3: D [2]};
\end{tikzpicture}

\end{center}
\caption{Example of LRU Replacement Policy}
\label{fig:replacement:lru}
\end{figure}

Figure~\ref{fig:replacement:lru} is an example of LRU replacement
policy in action: in a cache of four lines, the memory elements $A$, $B$, $C$,
and $D$ were initially loaded, in that order. In the figure, the rightmost
number in each cache line indicates its index for the LRU replacement policy.
The figure shows what happens when a memory element $E$ is loaded, then the
memory element $C$ is accessed. Because $A$ was the least-recently used element,
it is the one that was evicted upon insertion of $E$. As it was just added, the
element $E$ is considered as having been accessed, and all other indices are
updated accordingly. The figure then shows what happens when $C$ is loaded: all
indices strictly below $C$'s are increased by one, then the index of $C$ is set
to $0$, marking it as the most recently used cache line.

\subsection{Placement Policies}
\begin{figure}[hbt!]
\begin{center}

\begin{tikzpicture}[->,>=stealth',shorten >=1pt,auto,  semithick]
\tikzstyle{address} = [state,rectangle,  node distance = 0.5cm and 3cm]
\tikzstyle{cacheline} = [state,rectangle, node distance = 0.5cm and 3cm]

\node[address] (ADDR0)        {Address 0};
\foreach \c in {1,...,7}
{
   \node[address] (ADDR\c)  [below = of ADDR\the\numexpr \c - 1\relax]    {Address \c};
}

\node[cacheline] (CL0) [right = of ADDR0] {Cache Line 0};
\foreach \c in {1,...,7}
{
   \node[cacheline] (CL\c)  [below = of CL\the\numexpr \c - 1\relax]    {Cache Line \c};
}

\node[address] (ADDR8) [right = of CL0] {Address 8};
\foreach \c in {9,...,15}
{
   \node[address] (ADDR\c)  [below = of ADDR\the\numexpr \c - 1\relax]    {Address \c};
}

\foreach \cc in {0,...,7}
{
   \foreach \cl in {0,...,7}
   {
      \path (ADDR\cc.east) edge (CL\cl.west);
   }
}
\foreach \cc in {8,...,15}
{
   \foreach \cl in {0,...,7}
   {
      \path (ADDR\cc.west) edge (CL\cl.east);
   }
}

\end{tikzpicture}

\end{center}
\caption{Example of Fully Associative Cache Organization}
\label{fig:placement:fully_associative}
\end{figure}

\begin{figure}[hbt!]
\begin{center}
\input{\chapterdirectory/figure/cache_placement_direct_map.tex}
\end{center}
\caption{Example of Directly Mapped Cache Organization}
\label{fig:placement:direct_map}
\end{figure}

\begin{figure}[hbt!]
\begin{center}
\input{\chapterdirectory/figure/cache_placement_set_associative.tex}
\end{center}
\caption{Example of Set Associative Cache Organization}
\label{fig:placement:set_associative}
\end{figure}

Placement policies determine where a memory element can be stored within the
cache, according to the memory element's address. There are three common
strategies: allow it to go anywhere; allow it to go only at one location; and
allow it to go anywhere within a group of cache lines;

A cache in which memory elements can be placed anywhere is called a
\textit{fully associative} cache. This makes the replacement policy able to
perform at its best, since it applies to the whole cache at once. However,
locating a cache line is costly, since it can effectively be anywhere in the
cache. An example of fully associative cache organization is shown in
Figure~\ref{fig:placement:fully_associative}.

A cache in which each address maps to a single cache line is called a
\textit{direct-mapped} cache. This effectively negates the need for a
replacement policy. Access to a cache line is very fast. Depending on
the mapping algorithm and the accesses made, this can end up being very
efficient. However, even a very frequently accessed memory element is at risk of
being evicted if another memory element that maps to same cache line is also
accessed. Furthermore, this may lead to cache lines being evicted even when
other cache lines remain unused. An example of directly mapped cache
organization is shown in Figure~\ref{fig:placement:direct_map}.

A cache in which a mix of both strategies is used is called a
\textit{set-associative} cache. This is equivalent to having the cache be split
into equally sized groups of cache lines, and applying the direct mapping to
groups instead of single cache lines: each address is mapped to a group, but the
memory element can end up anywhere within that group. The replacement policy is
then only applied within the relevant group, which is much cheaper than a
cache-wide implementation. Additionally, more frequently used memory elements
sharing a set with other memory elements are less likely to get evicted. The
issue of cache lines being evicted despite having potentially unused cache lines
is still there, but only if these unused cache lines are not part of the group
to which the newly allocated cache line belongs. An example of set associative
cache organization where each set has two lines is shown in
Figure~\ref{fig:placement:set_associative}.
