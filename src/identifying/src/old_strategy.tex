\section{Identification Strategy}
\label{sec:identification:strategy}
This section presents the strategy used to ensure a complete understanding of
the cache coherence mechanisms of an architecture. It assumes some capacity of
observation on the platform, and that an already non-ambiguous cache coherence
protocol has been hypothesized to be the one implemented by the architecture.
With this, the general idea is to ensure that all observations made are
coherent with those that this hypothetical cache coherence protocol would lead
to.

As previously indicated, the first step is to define a hypothetical cache
coherence protocol using the notation presented in
Chapter~\ref{cha:cache_coherence}. All further steps of the strategy perform
micro-benchmarks in order to attempt to obtain observations contradicting this
hypothetical cache coherence protocol. To avoid any confusion between the two
protocols, the use of notations from Chapter~\ref{cha:cache_coherence} is solely
for the description of the hypothetical cache coherence protocol, and new
definitions are provided for reasoning over the observations made of the
architecture's actual cache coherence protocol.

Two types of monitors are assumed to be available: a way to see the state of
cache and coherence manager lines (see
Definition~\ref{def:identifying:cache_controller_state}), and performance
monitors (see Definition~\ref{def:identifying:performance_monitor}). The state
of caches may contain information unrelated to cache coherence (e.g.~cache
replacement tags). The types of performance monitors available determine the
confidence one can have in the results of this strategy. Indeed, the final
result can only be considered to a match or a mismatch \textit{according to what
can be observed}.

As explained in Chapter~\ref{cha:cache_coherence}, cache coherence protocols are
defined around a single memory element. Thus, the analysis only considers a
single memory element in its process.

Step~\ref{step:reachability} performs a standard reachability analysis process
which catalogs the observable coherence states (see
Definition~\ref{def:identifying:cache_controller_state}), as well as the valid
system coherence states (see
Definition~\ref{def:identifying:observable_system_state}). Furthermore, both the
transition relation of these system coherence states (see
Definition~\ref{def:identifying:observable_system_transitions}) and the
observable associated behavior (see
Definition~\ref{def:identifying:performance_monitor}) are also recorded. This
first analysis is done with a single instruction per transition, and thus does
not cover the entirety of the protocol, but provides an easily observed
baseline.

\begin{definition}[Observable Coherence State]
\label{def:identifying:cache_controller_state}
   On an real architecture, each line of a cache or coherence manager has
   \archboolflagscount{} binary flags providing information on its state. It is
   likely not all combinations of valuations for these flags are valid states.
   We thus define $\validarchboolflags{} \subseteq \archboolflags{}$ the set of
   all valid states. We can then define $\cacheboolstatefun{}: \caches{} \to
   \memoryelements{} \to \validarchboolflags$, which indicates the valuation of
   these binary flags for a given memory element in a given cache, and
   $\coherencemanagerboolstatefun{}: \to \memoryelements{} \to
   \validarchboolflags$ the coherence manager equivalent.
\end{definition}

\begin{issue}[Cataloging the Valid Observable Coherence States]
\label{issue:catalog_observable_states}
   \validarchboolflags{} is not originally known.
\end{issue}

\begin{issue}[Matching Observable Stable States with
Hypothetical Stable States]
   \label{issue:define_state_decode}
   The relation between \validarchboolflags{} and $\cachestate{} \cup
   \coherencemanagerstate{}$ is not originally known.  We need to identify a
   relation $\decodefun{}: \validarchboolflags{} \times (\cachestate{} \cup
   \coherencemanagerstate{})$, which decodes the relevant part of
   \validarchboolflags{} and matches it to an element of $\cachestate{} \cup
   \coherencemanagerstate{}$, which in turns allows us to reason over
   \cachestatefun{} and \coherencemanagerstatefun{} when given
   \cacheboolstatefun{} and \coherencemanagerboolstatefun{}. To complicate
   matters, part of the flags in \validarchboolflags{} may be unrelated to the
   coherency state of the memory element. Thus, \decodefun{} may not be
   injective.
   However, in any successful match, \decodefun{} must be
   surjective:
$
   \forallin{f}{\coherencemanagerstate{} \cup \cachestate{}}{%
      \exists r \in \validarchboolflags{}, \;
         \langle r, f \rangle \in \decodefun{}
%         \decodefun{}(r) = f%
   }%
$.
   As a result, in a successful match, the \decodefun{} relation is a function.
\end{issue}

\begin{definition}[Observable System State]
\label{def:identifying:observable_system_state}
   As with \systemstate{} for the hypothetical cache coherence protocol, using a
   shorthand for the state of the studied memory element in each component
   greatly simplifies notations. \obssystemstate{} is thus introduced, being the
   set of all $\langle{}CC_1, \ldots, CC_n, CM\rangle{}$ such that, there exists
   a memory element $E$ verifying:
   $
      \coherencemanagerboolstatefun{}(E) = CM \land
      \forallin{c}{0..\cachecount}{%
         \cacheboolstatefun(c, E) = CC_c
      }
   $.
\end{definition}

\begin{definition}[Observable System Transitions]
\label{def:identifying:observable_system_transitions}
   \obsreachfun{} is the analog to \hypreachfun{} (see
   Definition~\ref{def:hypreachfun}), but for the architecture's cache
   coherence protocol instead of the hypothetical one. It is thus declared as
   $\obsreachfun{}: \obssystemstate{} \to \systemaction{} \to
   set(\obssystemstate{})$.
\end{definition}

\begin{issue}[Definition of \obsreachfun{}]
   \label{issue:define_reachability}
   \obsreachfun{} is not originally known.
\end{issue}

\begin{issue}[Matching Transition Graphs]
   \label{issue:match_decode}
   For the hypothetical protocol to match the observed one, \decodefun{} has to
   be a simulation relation such that:
   $
      \forallin{s}{\obssystemstate{}}{%
         \forallin{d}{\obssystemstate{}}{%
            \forallin{d'}{\systemstate{}}{%
               \forallin{i}{\systemaction{}}{%
                  (d \in \obsreachfun{}(s, i))
                  \land (\langle{} s, s' \rangle{} \in \decodefun{})
                  \land (%
                     \existsin{d'}{\systemstate{}}{%
                        (\langle{} d, d' \rangle{} \in \decodefun{})
                        \land (d' \in \hypreachfun{}(s', i))
                     }
                  )
               }
            }
         }
      }
   $
   Which, in the case of a valid match, can be simplified to:
   $\decodefun{}~\circ~\obsreachfun{} = \hypreachfun{}~\circ~\decodefun{}$
\end{issue}

\begin{definition}[Performance Monitors]
\label{def:identifying:performance_monitor}
   The observation of the architecture's behavior is done through performance
   monitors. A performance monitor holds a number that counts the occurrences of
   a certain event. Each core is assumed to have its own monitors (although
   these monitors do not have to be limited to those relating to events from
   that code). With \archmonitor{} the set of monitors available on every core,
   $\archmonitorval{}: \obssystemstate{} \to \systemaction{} \to \archmonitor{}
   \to (\mathbb{R}^{+})^\cachecount{}$ indicates, for each core, the number of
   occurrences of an event observed when performing a given transition from a
   given initial state. The event behind each \archmonitor{} is assumed to be
   understood by the user.
\end{definition}

\begin{definition}[Performance Monitor Oracles]
\label{def:identifying:performance_monitor_oracle}
   $\hyparchmonitorval{}: \systemstate{} \to \systemaction{} \to
   \archmonitor{} \to (\mathbb{R}^{+})^\cachecount{}$ is the analogue of
   \archmonitorval{} applied to the hypothetical cache coherence protocol.
   \hyparchmonitorval{} thus serves as an oracle, indicating what
   \archmonitorval{} is expected to yield for the cache coherence protocols to
   match.
\end{definition}

\begin{issue}[Definition of \archmonitorval{}]
   \label{issue:define_event_count}
   \archmonitorval{} is not originally known.
\end{issue}

\begin{step}[Reachability Analysis]
\label{step:reachability}
   To address Issue~\ref{issue:catalog_observable_states}, and a portion of both
   Issue~\ref{issue:define_reachability} and
   Issue~\ref{issue:define_event_count}, a reachability analysis is performed.
   Starting from the initial situation where all cache controllers consider the
   memory element to be invalid ($\emph{init}$), the reachable observable system
   states ($\obssystemstate{}$) and behaviors ($\archmonitorval{}$) are
   cataloged by observing the effect of a single core instruction and the
   associated transition relation $\obsreachfun{}$. The idea is to run a
   benchmark and observe the reached state and performance monitor results.  If
   the reached state has not been visited, it is added to the waiting list
   (\textit{Candidates}), otherwise it is not. This is a standard reachability
   algorithm:
   \lstset{%
      escapeinside={(*}{*)},%
      keywordstyle=\bfseries,%
      morekeywords={while,let,in,if,then,else,foreach},%
      numbers=none%
   }
\begin{lstlisting}
(*$\validarchboolflags{} \gets$*) tuple_to_set((*$\emph{init}$*))
(*$\obssystemstate{} \gets \{\emph{init}\}$*)
Candidates (*$ \gets \{\emph{init}\}$*)
while (Candidates (*$\neq \emptyset$*)):
   C(*$~\in~\!\!$*)Candidates;
   Candidates (*$\gets$*) Candidates/C;
   foreach k (*$\in$*) 0..(*$\cachecount{}$*)
         foreach instr(*$~\in\{load, store, evict\}$*)
               SysInstruction (*$\gets$*) single_instruction_on(instr, k)
               (*$\langle{}$*)ObservedState, PerformanceCounters(*$\rangle{} \gets$*) benchmark(C, SysInstruction)
               (*$\obsreachfun{}$*)(C, SysInstruction) (*$\gets$*) {ObservedState}
               (*\archmonitorval{}*)(C, SysInstruction) (*$\gets$*) PerformanceCounters
               if ObservedState (*$\not\in \obssystemstate{}$*)
                  (*$\validarchboolflags{} \gets \validarchboolflags{} \cup \{$*)tuple_to_set(ObservedState)(*$\}$*)
                  (*$\obssystemstate{} \gets \obssystemstate{} \cup \{$*)ObservedState(*$\}$*)
                  Candidates (*$\gets$*) Candidates (*$\cup~\{$*)ObservedState(*$\}$*)
\end{lstlisting}
This process can be shortened by taking into account cache states that do not
emit any message when a particular instruction is applied. Indeed, in these
cases, the state of the other states is inconsequential, as they are not made
aware of the instruction's existence.
\end{step}

The results of Step~\ref{step:reachability} are then compared with those
expected to be observed on an architecture implementing the hypothetical
protocol. Step~\ref{step:state_matching} starts by matching the observed
coherence states with the stable states from the hypothetical protocol, thus
creating \decodefun{} (see Issue~\ref{issue:define_state_decode}).
Step~\ref{step:state_matching} is unlikely to result in the detection of a
mismatch between the hypothetical and the platform's cache coherence protocol,
as multiple observed coherence states can be matched to the same hypothetical
one, in order to take non-coherence-related flags into account. Thus, unless
some observed coherence state is matched to multiple hypothetical one, or there
is a hypothetical coherence state with no observed coherence state assigned, the
identification process continues.

\begin{step}[State Matching]
\label{step:state_matching}
   We can now resolve Issues~\ref{issue:define_state_decode} and
   \ref{issue:match_decode} by constructing \decodefun{} so that the property
   in Issue~\ref{issue:match_decode} is verified. Not being able to do so would
   indicate a mismatch between the protocols.

   \lstset{%
      escapeinside={(*}{*)},%
      keywordstyle=\bfseries,%
      morekeywords={while,let,in,if,then,else,foreach},%
      numbers=none%
   }
\begin{lstlisting}
(*$\decodefun \gets \langle init, <I, ..., I> \rangle$*)
DecodedStates (*$\gets {init}$*)
WaitList (*$\gets {init}$*)

while (decoded_states != (*$\obssystemstate{}$*))
   CurrentState(*$~\in~\!\!$*)WaitList
   WaitList(*$\gets$*) WaitList/CurrentState
   (*$\langle$*) CurrentState, CurrentDecodedState (*$\rangle \in \decodefun$*)
   foreach k (*$\in$*) 0..(*$\cachecount{}$*)
         foreach instr(*$~\in\{load, store, evict\}$*)
               SysInstruction (*$\gets$*) single_instruction_on(instr, k)
               {NextState} (*$\gets$ $\obsreachfun{}$*)(CurrentState, SysInstruction)
               NextDecodedState (*$\gets$ $\hypreachfun{}$*)(CurrentDecodedState, SysInstruction)
               (*$\decodefun \gets \decodefun \cup \{\langle$*) NextState, NextDecodedState (*$\rangle\}$*)
               if (NextState (*$\not\in$*) DecodedStates):
                  WaitList (*$\gets$*) {NextState} (*$\cup$*) WaitList
                  DecodedStates (*$\gets$*) {NextState} (*$\cup$*) DecodedStates
\end{lstlisting}
\end{step}

The process continues with Step~\ref{step:behavior_matching}, in which the
matching of hypothetical and observed coherence states is challenged by the
observed behaviors. This challenge takes two forms: if the observed coherence
states associated to a same hypothetical coherence state do not all exhibit the
same behavior, as seen through events relevant to cache coherence, then they
cannot be considered equivalent in a cache coherence protocol and so, this would
invalidate the hypothetical cache coherence protocol by pointing out that
architecture's protocol has additional stable states; the other possibility is
that, while they do exhibit the same behavior, that behavior does not match what
is expected according to the hypothetical coherence protocol (see
Definition~\ref{def:identifying:performance_monitor_oracle}).

\begin{step}[Behavior Matching]
\label{step:behavior_matching}
   Once the observed states have been matched with their hypothetical
   counterpart, a comparison between expected and observed behaviors can take
   place. At this point, if the hypothetical cache coherence protocol does
   indeed match the one used by the architecture, the partially defined
   $\archmonitorval{}$ function verifies:
$
   \forallin{sys}{\obssystemstate{}}{
      \forallin{act}{\systemaction{}}{%
         \forallin{mon}{\archmonitor}{%
            defined(\archmonitorval{}) \implies
               \archmonitorval{}(sys, act, mon) =
               \hyparchmonitorval{}(sys, act, mon)
         }
      }
   }
$.
   For any monitor where this property does not hold, the user must either
   find the reason behind the mismatch, or consider the hypothetical cache
   coherence protocol as disproved.
\end{step}

By the end of Step~\ref{step:behavior_matching}, the user has a good
understanding of the cache coherence mechanisms. By going one step further,
special optimizations can be sought: Step~\ref{step:reachability} can be
performed once more, but this time multiple instructions are executed
simultaneously, to check for possible mismatches in yet unexplored transient
state behaviors. Because the general structure of the cache coherence protocol
has already been confirmed, the hypothetical cache coherence protocol can be
used to generate a list of sequences of events to analyze (see
Definition~\ref{def:stable_state_change_path}).

\begin{definition}[Stable State Change Path]
\label{def:stable_state_change_path}
A stable state change path is a path between two stable states of the
cache controller protocol definition table. It is formed of a stable state
followed by a non-empty and cycle-free sequence of transient states terminated
by a stable state, with an event (see Definition~\ref{def:event}) separating
every state. The two stables states in a path may in fact be the same one.
\end{definition}

\begin{step}[Simultaneous Events Behavior Matching]
\label{step:simultaneous_event_behavior_matching}
To complete the analysis, both \archmonitorval{} and \obsreachfun{} have to
be fulled defined. This requires observation of the behaviors of cache coherence
when multiple events occur simultaneously. The relevant tests to perform
correspond to stable state change paths where there is more than one
event corresponding to either a \instructions{} or \querymessages{}, since all
tests with a single one of these have already been performed.
\end{step}

Upon completion of Step~\ref{step:simultaneous_event_behavior_matching}, if not
mismatch have been found, the hypothetical cache coherence protocol is confirmed
to be a suitable description of the cache coherence protocol found on the
architecture.
