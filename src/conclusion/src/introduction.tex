The introduction of multi-core processors in critical avionic systems requires
the ability to sufficiently predict their behavior so that certification of the
system is achievable. Indeed, the parallel nature of multi-core processors
leads to numerous interactions that do not strictly pertain to the objective
of the applications being ran. This large amount of unprompted interactions
render execution times difficult to predict and subject to large variations.
Among the main causes of large time variations is cache coherence, the
mechanism ensuring that data modifications performed by programs running in
parallel are properly propagated. While cache coherence can be implemented with
predictability in mind (see
Section~\ref{sec:rel_work:handling_it:through_hardware}), this requires hardware
modifications, precluding such solutions in an aeronautical context.
Unfortunately, the available strategies to predict worst-case execution time of
applications running on multi-core \textit{Commercial Off-The-Shelf} processors
require cache coherence to be deactivated (see
Section~\ref{sec:rel_work:handling_it:accepting_it} and
Chapter~\ref{cha:analyzing_rel_work}).

This thesis proposes solutions to begin addressing the issue of cache coherence
predictability in an aeronautical context. To do so, it provides the applicant
with tools that can expose and explain the interference generated by cache
coherence. In this chapter is provided a summary, the limitations, and
suggestions of future improvements for each contribution made in this thesis.
