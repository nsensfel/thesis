\begin{otherlanguage}{french}
L'objectif de cette thèse est d'offrir des outils d'aide à la certification
aéronautique de processeurs COTS multi-cœurs. Ces architectures sont par nature
parallèles et peuvent de ce fait largement améliorer les performances de
calcul. Cependant elles souffrent d'un grand manque de prédictibilité, au sens
où calculer les pires d'exécution même pour des programmes simples est un
problème complexe, voire impossible dans le cas général. En effet, les cœurs
partagent l'accès à presque toutes les ressources ce qui provoque des conflits
(qualifiés d'interférences) entrainant des variations non maîtrisées des temps
d'exécutions.  Parmi les mécanismes complexes d'un processeur multi-coeur se
trouve la cohérence de caches. Celle-ci assure que tous les cœurs lisant ou
écrivant dans un même bloc mémoire ne peuvent pas aveuglement ignorer les
modifications appliquées par les autres.  Afin de maintenir la cohérence de
caches, le processeur suit un protocole pré-déterminé qui définit les messages
à envoyer en fonction des actions d'un cœur ainsi que les actions à effectuer
lors de la réception du message d'un autre cœur.

Cette thèse porte sur l'identification des interférences générées par les
mécanismes de cohérence de caches ainsi que sur les moyens de prédiction de
leurs effets sur les applications en vue de réduire les effets négatifs
temporels.  La première contribution adresse les ambiguïtés dans la
compréhension que les applicants ont de la cohérence de cache réellement
présente dans l'architecture.  En effet, la documentation des architectures ne
fournit généralement pas suffisamment de détails sur les protocoles.  Cette
thèse propose une formalisation des protocoles standards, ainsi qu'une
stratégie, reposant sur les micro-benchmarks, pour clarifier les choix
d'implémentation du protocole de cohérence présent sur l'architecture. Cette
stratégie a notamment été appliquée sur le NXP QorIQ T4240. Une fois le
protocole correctement identifié, la seconde contribution consiste à réaliser
une description bas-niveau de l'architecture en utilisant des automates
temporisés afin de représenter convenablement les micro-comportements et
comprendre clairement comment le protocole de cohérence de cache agit. Ainsi,
un framework de génération de modèles génériques a été développé, capable de
supporter plusieurs protocoles de cohérence de cache et de représenter
différents agencements d'architectures afin de mieux correspondre à
l'architecture choisie par le postulant. La troisième contribution explique
comment utiliser cette représentation de l'architecture pour exhiber les
interférences. Elle propose une stratégie pour détailler les causes et effets
de chaque interférence liée à la cohérence de caches sur les programmes.
Commençant par une simple analyse de temps d'exécution, les résultats
descendent jusqu'au niveau des instructions pour indiquer comment chaque
instruction génère et souffre des interférences. L'objectif étant alors de
fournir suffisamment d'information à l'appliquant à la fois pour la
certification, mais aussi pour définir une stratégie d'atténuation et de
maîtrise des effets temporels.

Ainsi, cette thèse fournit l'appliquant des outils pour comprendre les
mécanismes de cohérence de cache présent sur une architecture donnée et pour
exhiber les interférences associées.
\end{otherlanguage}
