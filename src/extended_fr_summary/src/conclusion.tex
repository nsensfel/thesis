\section{Conclusion}
La documentation des architectures ne fournit pas suffisamment de
détails sur la cohérence de cache pour répondre aux besoins de
certification et peut même parfois induire en erreur l'applicant.
Pour remédier à cela, nous avons proposé une approche d'identification
du protocole de cache réellement implémenté sur l'architecture.  Une
fois le protocole identifié, il faut analyser le système dans son
ensemble. L'approche choisie, pour répondre partiellement à cette
question, a consisté à modéliser le protocole de cohérence de cache
sous forme d'un réseau d'automates temporisés, ce qui permet de
modéliser les composants participant à la cohérence en incluant leurs
comportements bas niveau.  Dans un souci de généricité, une approche
modulaire a été proposée afin de facilement modifier certains
composants et un outil prenant en entrée plusieurs protocoles permet
d'instancier le modèle UPPAAL ad hoc.  Un fois le modèle UPPAAL
instancié grâce à des paramètres dont les valeurs sont à obtenir au
travers de méthodes existantes, le modèle UPPAAL peut être analysé, à
l'aide de techniques de vérification formelle, pour mieux cerner les
interférences liées à la cohérence de cache dans le système.  Il est
notamment possible d'explorer toutes les traces d'exécution du modèle
afin d'obtenir des informations sur les effets temporels des
interférences dues à la cohérence de cache.

Plusieurs limitations ont été identifiées sur l'approche et les
premiers axes d'extension de résultats seraient de réduire celles-ci.
Comment prendre en compte des programmes plus réalistes et non
déterministes ? comment intégrer les modèles UPPAAL uniquement
focalisés sur la cohérence de cache aux frameworks plus généralistes
proposant des modèles de c\oe urs détaillées?  Comment remonter ces
informations dans le calcul de WCET et les documentations demandées
par la certification?
