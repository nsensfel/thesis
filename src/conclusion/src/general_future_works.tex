\section{General Future Works}
Overall, the framework proposed in this thesis would greatly benefit from
further automation. Indeed, while the cache coherence protocol obtained through
the identification process can already be automatically inserted into the model
for analysis, there is still a need for numerous user actions in every
contribution proposed in this thesis. Ideally, the user should only have to
provide a description of the architecture through a dedicated description
language and the program binaries. The framework would then generate a template
for the coherence protocol identification in which only architecture specific
instructions would have to be filled in by the user. The model would be
automatically modified to match the architecture's description. It would also
use the programs' binaries to generate corresponding automata (such as is done
in the works presented in Chapter~\ref{cha:analyzing_rel_work}). The cache
coherence interference analyzes would then automatically be performed,
providing the user with the results.

Another point which should be the focus of future works is the range of
supported architectures. This would start by taking into account more
components in the model, such as the pipelines present in the approaches from
Chapter~\ref{cha:analyzing_rel_work}. If programs are analyzed using their
binary forms, support for different instruction sets would become a deciding
factor in the usability of the framework.

Lastly, future works could involve the integration of this framework within a
more general analysis of the multi-core architecture, exporting the results on
cache coherence interference in a way that can be exploited by existing tools.
For example, being able to interface with OTAWA
\cite{10.1007/978-3-642-16256-5_6} in order to compute more accurate WCETs.
