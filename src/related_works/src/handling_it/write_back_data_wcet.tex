\subsubsection{Write-Back Data Caches}
\cite{10.1145/3139258.3139269} tackles the issue of WCET computation in
multi-core processors that use data (or unified) shared caches with write-back
policies. Although not explicitly stated in the paper, this approach does not
consider separate caches of the same hierarchy: all caches are shared by all
cores that could access the data they contain. Ergo, no cache coherence, but a
cascade of non-inclusive shared caches.

In addition to the standard categorization of a cache line in a particular
cache level (always hits, never hits, and so on\ldots), the \textit{dirtiness}
is also modeled. The possible values are: dirty (i.e.~has been written to, but
changes were not yet propagated), clean, or possibly dirty. The main challenge
is then to estimate when the write back will occur.
\cite{10.1145/3139258.3139269} indicates how to take this new attribute into
account when performing the usual cache static analysis, and the constraints it
adds to the linear optimization problem representation of the WCET computation.


