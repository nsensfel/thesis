\section{Conclusion}
The use of timed automata to analyze the real-time properties of an architecture
has already been done in a few publications, both for single-core and for
multi-core processors. The two main advantages of these approaches is the
certainty that all scenarios are taken into account, and the modularity of the
models, which facilitates their adaptation to most architectures.

However, these approaches have a tendency to suffer from combinatorial
explosion, which limits the size of the models they can be used on. In addition,
none of the existing literature exploits timed automata to model the effects
of cache coherence on the system. Instead, as in the analyses of the
previous chapter, cache coherence is assumed to be disabled.

This last point is remediated in this thesis, by the proposition of a multi-core
architecture model with a focus on cache coherence. For the model to be
accurate however, the details of what is modeled need to be known. This
information is obtained through benchmarks (Chapter~\ref{cha:micro-benchs})
and a proper identification of the cache coherence protocol, which is the
subject of the next chapter and first contribution of this thesis.
