\section{Coherence Protocols}
When definiting the protocols, the behavior of the caches and the coherence
manager is given in the form of a classical automata (see
Chapter~\ref{cha:formal_methods}), and limited to a single memory component.

\subsection{Introduction to the MSI Protocol}
\label{sec:intro_to_msi}
The archetypal cache coherence protocols are those belonging to the MSI
protocol family. In their basic version, these protocols feature three stable
states from which the MSI name is derived:
\begin{itemize}
\item \textbf{Modified:}
When a cache assigns the Modified state to a memory element, it indicates that
this cache performed at least one write to that memory element. While in this
state, it can freely read and write to that memory element. This corresponds
to being the \textit{Single Writer} from Property~\ref{prop:swmr}.

\item \textbf{Shared:}
When a cache assigns the Shared state to a memory element, it indicates that
this cache is able to freely read that memory element but not write to it. This
corresponds to being one of the \textit{Multiple Readers} (and potentially the
only one) of Property~\ref{prop:swmr}.

\item \textbf{Invalid:}
When a cache assigns the Invalid state to a memory element, it is no longer
considered to be holding a copy of that memory element, thus absolving this
cache from verifying Property~\ref{prop:system_wide_value}. Not having a copy,
the cache can neither read nor write to the memory element.
\end{itemize}

\stopallthesefloats
\begin{figure}
   \begin{center}
   \begin{subfigure}[b]{0.7\linewidth}
      \begin{tikzpicture}[->,>=stealth',shorten >=1pt,auto,node distance=5cm,
                    semithick]
   \node[state] (I)              {\texttt{I}};
   \node[state] (S) [right of=I] {\texttt{S}};
   \node[state] (M) [below of=I] {\texttt{M}};

   \path[every node/.style={}]
      (I) edge [bend left] node {\textcolor{blue}{\loadinstr{}}?\textcolor{red}{\textcolor{red}{\getsquery{}}}!\textcolor{OliveGreen}{\simpledata{}}?} (S)
      (I) edge [bend right, sloped, anchor=center] node[yshift=-1em] {\textcolor{blue}{\storeinstr{}}?\textcolor{red}{\textcolor{red}{\getmquery{}}}!\textcolor{OliveGreen}{\simpledata{}}?} (M)

      (S) edge node {\textcolor{blue}{\evictinstr{}}?} (I)
      (S) edge [bend right=90, above] node {\textcolor{red}{\textcolor{red}{\getmquery{}}}?} (I)
      (S) edge [bend left, sloped, anchor=center] node[yshift=1.5em] {\textcolor{blue}{\storeinstr{}}?\textcolor{red}{\textcolor{red}{\getmquery{}}}!\textcolor{OliveGreen}{\simpledata{}}?} (M)
      (S) edge [out=430,in=400,looseness=8] node {\textcolor{blue}{\loadinstr{}}?} (S)

      (M) edge [bend left=90, above, sloped] node {\textcolor{red}{\textcolor{red}{\getmquery{}}}?\textcolor{OliveGreen}{\simpledata{}}!} (I)
      (M) edge [bend right=90, above, sloped, anchor=center] node[yshift=-1em] {\textcolor{red}{\textcolor{red}{\getsquery{}}}?\textcolor{OliveGreen}{\simpledata{}}!\textcolor{OliveGreen}{\simpledata{}}!} (S)
      (M) edge [sloped, anchor=center] node[yshift=-1em]
         {\textcolor{blue}{\evictinstr{}}?\textcolor{red}{\putmquery{}}!\textcolor{OliveGreen}{\simpledata{}}!} (I)
      (M) edge [out=230,in=200,looseness=8] node {\textcolor{blue}{\loadinstr{}}?} (M)
      (M) edge [out=280,in=250,looseness=8] node {\textcolor{blue}{\storeinstr{}}?} (M)
   ;
\end{tikzpicture}

      \caption{Cache}
      \label{fig:general_msi_cc}
   \end{subfigure}
   \\
   \begin{subfigure}[b]{0.5\linewidth}
      \begin{tikzpicture}[->,>=stealth',shorten >=1pt,auto,node distance=5cm,
                    semithick]
   \node[state] (U)              {\texttt{I}};
   \node[state] (M) [left of=U] {\texttt{M}};

   \path[every node/.style={sloped, anchor=center, yshift=1em}]
      (U) edge [bend left] node {\textcolor{red}{\textcolor{red}{\getmquery{}}}?\textcolor{OliveGreen}{\simpledata{}}!} (M)
      (U) edge [loop below] node [below,yshift=-1em] {\textcolor{red}{\textcolor{red}{\getsquery{}}}?\textcolor{OliveGreen}{\simpledata{}}!} (U)

      (M) edge [bend left] node {\textcolor{red}{\putmquery{}}?\textcolor{OliveGreen}{\simpledata{}}?} (U)
      (M) edge [bend left=90] node {\textcolor{red}{\textcolor{red}{\getsquery{}}}?\textcolor{OliveGreen}{\simpledata{}}?} (U)
   ;
\end{tikzpicture}

      \caption{Coherence Manager}
      \label{fig:general_msi_cmgr}
   \end{subfigure}
   \begin{subfigure}[b]{0.3\linewidth}
      \begin{tikzpicture}[->,>=stealth',shorten >=1pt,auto,node distance=5cm,
                    semithick]
   \node[state] (U)              {};

   \path[every node/.style={sloped, anchor=center, yshift=1em}]
      (U) edge [loop left] node [below,yshift=0.5em] {\textcolor{blue}{\loadinstr{}}!} (U)
      (U) edge [loop right] node [below,yshift=-1em] {\textcolor{blue}{\storeinstr{}}!} (U)
      (U) edge [loop below] node [below,yshift=-1em] {\textcolor{blue}{\evictinstr{}}!} (U)
   ;
\end{tikzpicture}

      \caption{Core}
      \label{fig:general_msi_cpu}
   \end{subfigure}
   \end{center}
   \caption{Overview of the MSI Protocol}
   \label{fig:general_msi}
\end{figure}

To keep it simple, the protocol presented in this section does not use any FIFO
queue. Instead, all exchanges are done synchronously. The only elements taken
into account are the state of the memory elements and the points of
synchronization (i.e. instruction and message exchanges). Thus, the notations
on the automata of this section are not the notations presented in the previous
section, but are instead based on CCS (Calculus of Communicating Systems,
\cite{10.5555/539036}): a \texttt{?} marks the reception of a signal, and
\texttt{!} marks its sending.  Queries are still sent to all components, and
data only to a single one.  Furthermore, only a single transaction (that is,
the resolution of an instruction) can occur at any
moment. This is reflected by transitions featuring multiple synchronizations,
as no other actions would be permitted to be interleaved.

Figure~\ref{fig:general_msi} shows the two automata corresponding to this
simplified protocol. Each of these automata pertains to a single memory element,
with states corresponding to what the component attributes to that memory
element.

In the case of the coherence manager (Figure~\ref{fig:general_msi_cmgr}), the
two states simply correspond to whether or not the coherence manager is in
charge of providing the current value for that memory element: \texttt{M}
indicates that one of the caches is currently in the \texttt{M} state, meaning
that the value stored in the system's main memory may be out of date; \texttt{I}
indicates that the main memory's value for that element is up-to-date, leaving
(in the MSI protocol) the coherence manager in charge of propagating copies
as the caches demand them.

\begin{example}[Simple MSI Transition]
\label{ex:general_msi_single_store}
Consider a system with two caches, $C_A$ and $C_B$, and a single memory
element $E$, such that, initially, $C_A$ holds no copy of $E$ (\texttt{I}
state) and $C_B$ holds one in the \texttt{S} state, the coherence manager
has the \texttt{I} state assigned to $E$. Were $C_A$ to receive a \storeinstr{}
request, it would broadcast a \getmquery{} to itself, $C_B$, and the coherence
manager. Upon receiving the \getmquery{}, $C_B$ would have to abandon the
\texttt{S} state for \texttt{I} in order to maintain Property~\ref{prop:swmr}.
The coherence manager would react to receiving the \getmquery{} by sending a
copy of $E$ from the system's main memory to $C_A$ (\simpledata!)
before moving to \texttt{M} so as to memorize that the most up-to-date value
for $E$ is now held by $C_A$. This \simpledata{} reply allows $C_A$ to move to
the \texttt{M} state, thus completing its transition.
\end{example}



\subsection{Properties to be Verified}
\label{sec:proper_cache_coherence_protocol}
A system maintaining cache coherence is a system in which the application of
each read and write instruction on the same memory elements across multiple
caches holds the same results as if these caches shared a permanent single copy
of each of those memory elements.

One possible strategy to achieve cache coherence is to enforce
Properties~\ref{prop:system_wide_value}, \ref{prop:swmr}, and
\ref{prop:forget_me_not}. Other approaches may also be possible, but this is
the one considered in this thesis.

\begin{property}[Caches Have the System-Wide Value]%
\label{prop:system_wide_value}
At any point, for each memory element, all copies of that memory element being
held in a cache have the value that was last written to that memory element,
regardless of which cache performed the writing.
\end{property}

\begin{property}[Single Writer or Multiple Readers]%
\label{prop:swmr}
At any point, for each memory element, there is either a single cache being
able to write to that memory element while the others can neither read nor
write to it, or there is any number of caches being able to read the memory
element and none able to write to it.
\end{property}

\begin{property}[Forget Me Not]%
\label{prop:forget_me_not}
If a memory element has no copy held in cache, then the system's main memory
has the value that was last written to that memory element.
\end{property}

Cache coherence protocols can be split into categories according to whether
they are directory-based or snooping-based, and write-back or write-through.

As this thesis only considers snooping-based protocols, the component
descriptions provided in Section~\ref{sec:cache_coherence_components} are
unlikely to fit a directory-based protocol. Indeed, in directory-based
protocols, cache queries are not broadcasted to every cache but instead
addressed to a component that, much like the coherence manager, keeps track of
which component may need to see this query and sends a copy of that query to
them. In snooping-based protocols, the caches and the coherence manager all
receive every query, and do so in the same order.

Write-through protocols ensure that any modification made to a memory element's
copy in a cache is also applied to the original in the system's main memory.
Conversely, write-back protocols delay the modification of the original until
the cache loses the write permissions on its own copy. As it happens, the
protocols studied during this thesis are all write-back ones.

\subsection{Protocol Definition}
Protocols are customarily defined by the behavior they require of caches (and
the coherence manager, if it is used) for a single memory element according to
its currently attributed state in the component and the type of incoming event.

\begin{definition}[Event]
\label{def:event}
Events for a single memory element are defined as $\events{} = \querymessages{}
\cup \instructions{} \cup \replymessages{}$
%\cup \{\ownquery{}\}$, with \ownquery{} indicating a query that was sent by this cache.
\end{definition}

\begin{definition}[Protocol]
\label{def:protocol}
A protocol is thus defined as two functions:
$\protocolccfun{} = \cachestate \to \events{} \to set(\actions{})$, which
defines the behavior of cache controllers, given the state of memory element
and an incoming event; and
$\protocolcmgrfun{} = \coherencemanagerstate \to \events{} \to set(\actions{})$,
the coherence manager equivalent. \actions{} corresponds to the actions
described in each component's sub-section.
\end{definition}

\begin{definition}[System State]
\label{def:systemstate}
   For the definition of a cache coherence protocol, only a single memory
   element needs to be considered. Indeed, the behavior is the same for any
   address, and thus defining it for one is defining it for all. This can be
   exploited to shorthand the notation of memory element states across the
   system: Given $E$ the memory element arbitrarily chosen for the protocol's
   definition, a system $\langle CC_1, \ldots, CC_n, CM\rangle$ denotes one in
   which
   $
      \coherencemanagerstatefun{}(E) = CM \land
      \forallin{c}{0..\cachecount}{%
         \cachestatefun(c, E) = CC_c
      }
   $. The set of all possible systems states is denoted \systemstate{}.
\end{definition}

\begin{definition}[Stable System Transitions]
   \label{def:hypreachfun}
   To allow reasoning more simply over state changes, another shorthand notation
   is introduced: $\hypreachfun{}: \systemstate{} \to
   \instructions{}^\cachecount{} \to set(\systemstate{})$ which, given a system
   state and instructions to apply on each cache, returns the set of all
   possible next system states composed solely of stable states that can be
   reached.
\end{definition}
