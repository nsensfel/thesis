\section{Identifier la coh\'erence de cache}
\label{fr:sec:identify}
Cette section présente la stratégie d'identification du protocole de cache
implémenté par l'architecture choisie par l'applicant. Ces travaux ont fait
l'objet d'une publication \cite{ecrts20}. Pour éviter les confusions, trois
protocoles sont définis ci-dessous:

\begin{definition}[Le protocole de l'architecture]
Le protocole réellement implémenté sur l'architecture. Celui que l'applicant
souhaite identifier. Il n'est probablement pas observable directement.
\end{definition}

\begin{definition}[Le protocole observé]
Le protocole observé est la vue partielle du protocole de l'architecture
obtenue à travers un ensemble de benchmarks. Puisqu'il n'est pas possible
d'assurer que les benchmarks soient exhaustifs, le protocole
observé est potentiellement incomplet.
\end{definition}

\begin{definition}[Le protocole hypothétique]
  Le protocole défini par l'applicant et qui correspond à ce qu'il
  s'attend à trouver d'après la documentation de l'architecture.
\end{definition}

Pour rappel, les protocoles de cache sont definis autour d'un seul
élément mémoire. En conséquent, la stratégie que nous proposons ne
considère qu'un seul élément mémoire. De plus, on suppose que
les états stables du protocole de l'architecture sont observables.

\subsection{Définir le protocole hypothétique}
La première étape de cette stratégie d'identification est la définition du
protocole hypothétique par l'applicant. Cette définition devrait utiliser
les notations présentées dans la Section~\ref{fr:sec:cache_coherence}. Un bon
point de départ est de se référer à la documentation de l'architecture.
Par exemple, pour le NXP QorIQ T4240, la documentation indique un protocole
MESI (dans \cite{e6500}), avec des optimisations pour le partage de données déjà
en cache (dans \cite{T4240}).

\subsection{Exploration naïve du protocole observable}
Pour définir le protocole observable, on effectue une exploration des états
accessibles. L'algorithme correspondant à cette exploration se trouve dans
la Figure~\ref{fr:fig:identification:state_exploration}. En commençant dans un
état où aucun cache ne contient la donnée ($\emph{init}$), on explore les états
accessibles par l'application d'une unique instruction à la fois, en notant
l'état de cohérence du système une fois l'instruction complétée, ainsi que les
valeurs des différents moniteurs de performance. Les étapes du protocole
correspondant à \lstinline!state_search!, \lstinline!decode! et
\lstinline!monitors! sont détaillées par la suite.

\begin{figure}[hbt!]
\lstset{%
   escapeinside={(*}{*)},%
   keywordstyle=\bfseries,%
   morekeywords={while,let,in,if,then,else,def,foreach},%
   numbers=none%
}
\begin{lstlisting}
init_state_search()
init_decode()

DstStates (*$ \gets \{\emph{init}\}$*)
WaitList (*$ \gets \{\emph{init}\}$*)
while (WaitList (*$\neq \emptyset$*)):
   SrcState(*$~\in~\!\!$*)WaitList;
   WaitList (*$\gets$*)WaitList \ SrcState;
   foreach k (*$\in$*) 1..(*$\cachecount{}$*)
         foreach instr(*$~\in\{load, store, evict\}$*)
               SysInstruction (*$\gets$*) single_instruction_on(instr, k)
               (*$\langle{}$*)DstState, PerformanceCounters(*$\rangle{} \gets \benchmark{}$*)(SrcState, SysInstruction)

               handle_state_search(SrcState, SysInstruction, DstState) // Step 1
               handle_decode(SrcState, SysInstruction, DstState) // Step 2
               handle_monitors(SrcState, SysInstruction, PerformanceCounters) // Step 3

               if DstState (*$\not\in$*) DstStates
                  DstStates (*$\gets$*) DstStates (*$\cup~\{$*)DstState(*$\}$*)
                  WaitList (*$\gets$*) WaitList (*$\cup~\{$*)DstState(*$\}$*)
\end{lstlisting}
\caption{General State Exploration Algorithm}
\label{fr:fig:identification:state_exploration}
\end{figure}

La fonction \lstinline!benchmark! retourne l'état de cohérence des différents
composants du système (caches et gestionnaire de cohérence). Dans le cas où le
gestionnaire de cohérence n'est pas observable, on considère qu'il suit les
règles du protocole hypothétique et on définit le protocole observé comme s'il
comportait le même gestionnaire de cohérence.

\subsection{Exploration d'état et atteignabilité}
L'étape \lstinline!state_search! catalogue les états stables de cohérence
observable dans chaque composant $\validarchboolflags{}$, ainsi que leurs
combinaisons à l'échelle du système (c'est-à-dire l'état de cohérence du
système) \obssystemstate{}.

\lstset{%
   escapeinside={(*}{*)},%
   keywordstyle=\bfseries,%
   morekeywords={while,let,in,if,then,else,def,foreach},%
   numbers=none%
}
\begin{lstlisting}
def init_state_search ()
   (*$\validarchboolflags{} \gets$*) tuple_to_set((*$\emph{init}$*))
   (*$\obssystemstate{} \gets \{\emph{init}\}$*)

def handle_state_search (SrcState, SysInstruction, DstState)
   (*$\obsreachfun{}$*)(SrcState, SysInstruction) (*$\gets$*) {DstState}
   if DstState (*$\not\in \obssystemstate{}$*)
      (*$\validarchboolflags{} \gets \validarchboolflags{} \cup \{$*)tuple_to_set(DstState)(*$\}$*)
      (*$\obssystemstate{} \gets \obssystemstate{} \cup \{$*)DstState(*$\}$*)
\end{lstlisting}

On récupère en fait le graphe de transition des états de cohérence du système.


\subsection{Correspondance entre état observé et hypothétique}
Les états et transitions observés lors de l'étape précédente sont
ensuite comparés avec ceux attendus avec le protocole hypothétique. On
associe alors les états de ces deux protocoles en définissant
$\decodefun$, la relation qui permet de passer d'état observé à état
hypothétique. Pour cela, on a besoin de $\hypreachfun{}$, qui
correspond au graphe de transition d'états stables pour le protocole
hypothétique.

\lstset{%
      escapeinside={(*}{*)},%
      keywordstyle=\bfseries,%
      morekeywords={while,let,in,if,then,else,def,foreach},%
      numbers=none%
   }
\begin{lstlisting}
def init_decode ()
   (*$\decodefun \gets \langle init, <I, \ldots, I> \rangle$*)

def handle_decode (SrcState, SysInstruction, DstState)
   (*$\langle$*)SrcState, DecodedSrcState(*$\rangle \in \decodefun$*)
   {DecodedDstState} (*$\gets$ $\hypreachfun{}$*)(DecodedSrcState, SysInstruction)
   (*$\decodefun \gets \decodefun \cup \{\langle$*)DstState, DecodedDstState(*$\rangle\}$*)
\end{lstlisting}

Il est possible que plusieurs états observés soient associés au même état
hypothétique. En effet, lorsque l'on observe l'état d'une ligne de cache, il est
possible qu'une partie de l'information ne soit pas liée à la cohérence de
cache. En conséquent, plusieurs états observés peuvent en fait correspondre au
même état de cohérence.

Cependant, si le protocole hypothétique correspond effectivement au
protocole de l'architecture, alors il ne doit pas y avoir d'état
observé associé à plusieurs états hypothétiques. En effet,
cela indiquerait que certains comportements du protocole hypothétique
sont absents de l'architecture. Il en est de même si l'un des états
hypothétiques n'est associé à aucun état observé. Si l'une de ces deux
situations non souhaitées parvient, alors le protocole hypothétique
est infirmé.


\subsection{Comparaison des activités}
Si le protocole hypothétique n'est pas infirmé lors de l'étape précédente, les
activités observées sur l'architecture grâce aux moniteurs de performance sont
comparées à celles attendues d'après le protocole hypothétique. Pour cela, l'applicant note le nombre d'événements observés lors de chaque benchmark dans
\archmonitorval{}.

\lstset{%
   escapeinside={(*}{*)},%
   keywordstyle=\bfseries,%
   morekeywords={while,let,in,if,then,else,def,foreach},%
   numbers=none%
}
\begin{lstlisting}
def handle_monitors(SrcState, SysInstruction, PerformanceCounters)
   (*\archmonitorval{}*)(SrcState, SysInstruction) (*$\gets$*) PerformanceCounters
\end{lstlisting}

Les résultats obtenus dans \archmonitorval{} doivent correspondre à ce que le
protocole hypothétique génère. Toute exception doit être étudiée, car
elle pourrait indiquer une différence significative entre les deux protocoles.
De plus, si certains états observés n'agissent pas de la même façon vis-à-vis
d'événements de cohérence alors qu'ils sont associés au même état théorique,
alors les deux protocoles ne se correspondent pas. En effet, le protocole
implémenté a dans ce cas plus d'états stables de cohérence que le protocole
hypothétique.

Si, à la fin de cette étape, le protocole hypothétique n'a pas été contredit
par les observations, alors le protocole hypothétique décrit tous les
comportements du protocole observé. Pour s'assurer que le protocole de
l'architecture implémente tous les comportements du protocole hypothétique,
l'étape suivante fait une exploration guidée par le protocole hypothétique.

\subsection{Exploration guidée par le protocole hypothétique}
\begin{definition}[Chemin de changement d'état stable]
\label{fr:def:stable_state_change_path}
Un chemin de changement d'état stable est un chemin entre deux états stables
de la partie ``contrôleur de cache'' du protocole. Il est formé d'un état stable
suivi d'une séquence d'états transitoires et termine par un état stable, avec
une transition du protocole entre chaque état. Les deux états stables
peuvent être les mêmes. Les transitions sans actions ne sont pas permises.
\end{definition}

Cette étape du processus d'identification s'assure que tous les comportements
du protocole hypothétique sont présents sur l'architecture. Elle repose sur
l'établissement de la liste exhaustive des chemins de changement d'états stables
(voir Définition~\ref{fr:def:stable_state_change_path}), qui peut être générée
automatiquement par un outil fourni dans le cadre de cette thèse.


L'idée générale est simple: reproduire la séquence de chacun de ces chemins sur
l'architecture afin de s'assurer que les activités observées sur l'architecture
correspondent à celles attendues. Cependant, l'implémentation de ces benchmarks
est beaucoup plus difficile que pour les étapes précédentes, puisque les
transitions doivent être faites dans un ordre précis. Comparé aux étapes
précédentes:
\begin{itemize}
    \setlength{\itemsep}{0pt}%
   \setlength{\parskip}{0pt}%
\item Plusieurs instructions peuvent être appliquées simultanément
  (sur différents cœurs) afin de générer la séquence désirée. De
  plus, les benchmarks peuvent comporter des séquences d'instructions,
  au lieu d'une seule par cache.
\item
   L'analyse est concentrée sur un seul cache et non l'ensemble du système. À la
   place, les autres caches sont utilisés pour causer les bonnes transitions.
 \item L'exploration n'est pas aveugle: l'espace d'états est
   maintenant connu. La difficulté repose sur l'obtention de
   la séquence souhaitée sur l'architecture.
\end{itemize}

Une fois cette exploration terminée, si tous les comportements du protocole
hypothétique ont correctement été observés, alors le protocole de
l'architecture est garanti de tous les implémenter. Combiné avec les résultats
montrant que le protocole hypothétique décrit correctement le protocole
observé, on peut conclure que cette stratégie d'identification a fourni une
bonne compréhension du protocole implémenté par l'architecture à l'applicant.

\subsection{Application au NXP QorIQ T4240}
L'application de cette stratégie sur le NXP QorIQ T4240 a révélé des résultats
intéressants. En effet, tenter de valider un protocole hypothétique MESI révèle
un seul état de cohérence observé pour chaque état de cohérence hypothétique,
sauf pour l'état hypothétique \texttt{S}, qui est associé à deux états observés
(\benchs{} et \benchx{}).

\begin{figure}[hbt!]
\begin{center}
\begin{tabular}{|c|c|c|}
\cline{2-3}
\multicolumn{1}{c}{}
& \multicolumn{2}{|c|}{$\langle$\texttt{load, -, -}$\rangle$}
\\

\hline
   \multirow{2}{*}{\textbf{Origin}}
 & \multicolumn{2}{c|}{\textbf{Behavior}}
\\
 & \textbf{Expected}
 & \textbf{Observed}
\\
\hline

%%%% I/\benchi{}/\benchi{} %%%%%%%%%%%%%%%%%%%%%%%%%%%%%%%%%%%%%%%%%%%%%%%%%%%%%%%%%%%%%%%%%%%%%
% Origin
\texttt{\textbf{$\langle$\benchi{}},\benchi{},\benchi{}$\rangle$}

% Behavior (Expected, Observed)
&
   \begin{tabular}{@{}l@{}}
      8000 L2D Accesses,\\
      8000 Reloads From CoreNet
   \end{tabular}
&
   \begin{tabular}{@{}l@{}}
      16000 L2D Accesses,\\
      8000 Reloads From CoreNet,\\
      1166700 CPU Cycles
   \end{tabular}
\\
\hline

%%%% I/\benchi{}/\benchs{} %%%%%%%%%%%%%%%%%%%%%%%%%%%%%%%%%%%%%%%%%%%%%%%%%%%%%%%%%%%%%%%%%%%%%
% Origin
\texttt{\textbf{$\langle$\benchi{}},\benchi{},\benchs{}$\rangle$}

% Behavior (Expected, Observed)
&
   \begin{tabular}{@{}l@{}}
      8000 L2D Accesses,\\
      8000 Reloads From CoreNet
   \end{tabular}
&
   \begin{tabular}{@{}l@{}}
      16000 L2D Accesses,\\
      8000 Reloads From CoreNet,\\
      850600 CPU Cycles
   \end{tabular}
\\
\hline

%%%% I/\benchi{}/X %%%%%%%%%%%%%%%%%%%%%%%%%%%%%%%%%%%%%%%%%%%%%%%%%%%%%%%%%%%%%%%%%%%%%
% Origin
\texttt{$\langle$\textbf{\benchi{}},\benchi{},\benchx{}$\rangle$}

% Behavior (Expected, Observed)
&
   \begin{tabular}{@{}l@{}}
      8000 L2D Accesses,\\
      8000 Reloads From CoreNet
   \end{tabular}
&
   \begin{tabular}{@{}l@{}}
      16000 L2D Accesses,\\
      8000 Reloads From CoreNet,\\
      1172600 CPU Cycles
   \end{tabular}
\\
\hline
\end{tabular}

\vspace{1em}

\begin{tabular}{|c|c|c|}
\cline{2-3}
\multicolumn{1}{c}{}
& \multicolumn{2}{|c|}{$\langle$\texttt{-, -, load}$\rangle$}
\\

\hline
   \multirow{2}{*}{\textbf{Origin}}
 & \multicolumn{2}{c|}{\textbf{Behavior}}
\\
 & \textbf{Expected}
 & \textbf{Observed}
\\
\hline

%%%% I/\benchi{}/\benchi{} %%%%%%%%%%%%%%%%%%%%%%%%%%%%%%%%%%%%%%%%%%%%%%%%%%%%%%%%%%%%%%%%%%%%%
% Origin
\texttt{$\langle$\textbf{\benchi{}},\benchi{},\benchi{}$\rangle$}

% Behavior (Expected, Observed)
&
   \begin{tabular}{@{}l@{}}
      8000 External Snoop Requests
   \end{tabular}
&
   \begin{tabular}{@{}l@{}}
      8000 External Snoop Requests
   \end{tabular}
\\
\hline

%%%% F/\benchi{}/\benchi{} %%%%%%%%%%%%%%%%%%%%%%%%%%%%%%%%%%%%%%%%%%%%%%%%%%%%%%%%%%%%%%%%%%%%%
% Origin
\texttt{$\langle$\textbf{\benchs{}},\benchi{},\benchi{}$\rangle$}

% Behavior (Expected, Observed)
&
   \begin{tabular}{@{}l@{}}
      8000 L2 Snoop Hits,\\
      8000 External Snoop Requests\\
   \end{tabular}
&
   \begin{tabular}{@{}l@{}}
      8000 L2 Snoop Hits,\\
      \colorbox{pink}{\textbf{8000 L2 Snoop Pushes}},\\
      8000 External Snoop Requests,\\
      \colorbox{pink}{\textbf{8000 SINTs}}
   \end{tabular}
\\
\hline

%%%% S/\benchi{}/I %%%%%%%%%%%%%%%%%%%%%%%%%%%%%%%%%%%%%%%%%%%%%%%%%%%%%%%%%%%%%%%%%%%%%
% Origin
\texttt{$\langle$\textbf{\benchx{}},\benchi{},\benchi{}$\rangle$}

% Behavior (Expected, Observed)
&
   \begin{tabular}{@{}l@{}}
      8000 L2 Snoop Hits,\\
      8000 External Snoop Requests
   \end{tabular}
&
   \begin{tabular}{@{}l@{}}
      8000 L2 Snoop Hits,\\
      8000 External Snoop Requests
   \end{tabular}
\\
\hline
\end{tabular}

\end{center}
\caption{Anomalies pour les états \benchs{} et \benchx{}}
\label{fr:fig:unexpected_behaviors}
\end{figure}

Bien qu'avoir deux états observés pour un même état hypothétique ne soit pas
suffisant pour contredire le protocole hypothétique, la suite du processus
d'identification a révélé que ces deux états observés sont différents du point
de vue de la cohérence de cache. En effet, comme montré dans la
Figure~\ref{fr:fig:unexpected_behaviors}, les états observés \benchs{} et
\benchx{} réagissent de manière différente à une demande provenant d'un autre
cache: l'état \benchs{} entraine une réponse à la demande de la part du cache,
alors que l'état \benchx{} ne répond pas. Le comportement attendu pour un état
hypothétique \texttt{S} est de ne pas répondre. Ainsi, les observations montrent
que l'architecture a un état stable supplémentaire et n'utilise donc pas un
protocole MESI. Cette état supplémentaire semble correspondre au \texttt{F} d'un
protocole MESIF et, après une seconde application de la stratégie
d'identification du protocole, cela s'est confirmé.
