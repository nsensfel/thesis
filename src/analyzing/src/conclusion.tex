\section{Conclusion}
This chapter has shown how UPPAAL's model checking capabilities can be exploited
to analyze the interference caused by cache coherence on the model from
Chapter~\ref{cha:modeling_cache_coherence}.

This analysis starts by a computation of the WCET for each program. Useful in
itself, this analysis is extended by that of the WCET for these programs with
the architecture in different configurations in order to extract more
information about how much of the execution time is caused by interference.

In order to more precisely understand what determine the WCET and to provide
the user with information about elements of the program that can directly be
manipulated, the analysis proceeds by an categorization of the accuracy of each
instruction. This indicates which instructions are unaffected by the
interference, which instructions are always time-consuming, and which
instructions take a varying amount of time depending on the execution. By
looking at the accuracy of all accesses made on each memory element, patterns
for these instructions of varying execution time can sometimes be found, which
results in a more predictable system.

The determining factor for the accuracy of instructions is then properly
defined. This corresponds to a categorization of all external queries depending
on their effects on the permissions held by a cache, and whether a loss of
permission led to an instruction taking additional time. Thus, three categories
of interference are defined: minor (no change of permission, but loss of time
due to query processing), demoting (loss of writing permission), and expelling
(loss of all permissions).

Finally, analyses are performed in order to determine how each instruction
interfere with the other instructions. This results in a graph showing, for
each instruction, which instruction can generate interference that will
directly impact it, the category of this interference, and whether this
interference occurs on all possible executions or not.

This provides the user with a clear understanding of the causes and effects of
cache coherence interference on the programs' instructions, opening the way to
finding means of mitigation for this interference.
