The first step toward the analysis and the control of interference caused by
the use of cache coherence is to ensure that the mechanisms of cache coherence
used on the targeted architecture are well understood.

Chapter~\ref{cha:micro-benchs} presented papers on benchmark-based strategies to
expand the information available on the architecture beyond what the
documentation provides. That chapter was limited to performance (namely,
execution time and bandwidth) analysis. In this chapter, benchmarks are used to
clarify and expand the information available to the user about the architecture'
cache coherence protocol.

This chapter includes the results of applying this strategy to the NXP QorIQ
T4240 architecture. This architecture is documented as implementing a MESI cache
coherence protocol, and as able to perform \textit{cache intervention}. The
application of the cache coherence protocol identification strategy on this
architecture reveals a MESIF protocol. The work presented in this chapter has
been published in \cite{ecrts20}.

The identification strategy uses benchmarks to identify a cache coherence
protocol by comparing them with the protocol the user believes it to be. In
order to so, it is important to distinguish the following three notions:
\begin{definition}[The Architecture's Coherence Protocol]
The actual coherence protocol implemented by the architecture, which is the one
the applicant needs to identify. It may not be directly observable.
\end{definition}

\begin{definition}[The Observed Coherence Protocol]
The \textbf{observed} cache coherence protocol is the partial view of the
architecture's coherence protocol that is observed by performing a given set of
benchmarks. As these benchmarks cannot be exhaustive, the observed coherence
protocol is potentially incomplete.
\end{definition}

\begin{definition}[The Hypothetical Coherence Protocol]
The user believes the architecture to be implementing a certain protocol. This
corresponds to the \textbf{hypothetical} cache coherence protocol. It originally
comes from the user's understanding of the architecture's documentation.
\end{definition}

The purpose of the identification strategy is to show that the hypothetical
protocol is indeed the architecture's cache coherence protocol. This
verification is done by resolving the observed cache coherence protocol and
analyzing it through comparison with the hypothetical protocol.

The chapter starts by a presentation of the strategy itself, in an
architecture-agnostic manner (Section~\ref{sec:identification:strategy}), then
follows up with its application to the NXP QorIQ T4240 architecture:
Section~\ref{sec:identification:implementation} presents how the benchmarks
were implemented, Section~\ref{sec:identification:mesi} presents a hypothetical
MESI protocol implementation, Section~\ref{sec:identification:application}
shows the result of applying the strategy until the point where the protocol is
found to be a mismatch, Section~\ref{sec:identification:mesif} presents a
hypothetical MESIF protocol implementation, and
Section~\ref{sec:identification:second_application} shows the results pointing
out the slight discrepancies in implementation choices.
