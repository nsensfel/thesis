\stopallthesefloats
\section{Related Works Overview}
The related works part of this thesis is split into three parts. The first part
covers works relating to the general issue of using the caches of a multi-core
processors in a critical environment. The general consensus ending up being:
the use of caches in a way that does not make WCET computations yield unusable
results requires either hardware modifications, or severe restrictions on their
use. Nevertheless, solutions do exist. The very few that do allow cache
coherence require hardware modifications. This thesis considers a context in
which such modifications are not possible, and thus explores an approach relying
solely on the analyzing the effects of cache coherence in order to control the
interference it generates.

Even without cache coherence, the analysis of caches by themselves is not
simple: determining the state of the cache when an access is made requires
perfect knowledge of the placement and replacement policies, as well as of all
access made until that point. Cache coherence complicates matters, by adding
the notion of access rights (\textit{read-only}, \textit{read-and-write}) and
allowing other caches to change the content of the analyzed cache. Thus, there
are even more cases where a small simplification of the analysis may lead to
completely incorrect results. In this thesis, the use of formal methods (see
Chapter~\ref{cha:timed_automata}) ensure all cases are covered. The next
chapter of the related works section covers similar approaches for the analysis
of execution time using timed automata.

These approaches transform the problem into the issue of having an accurate
representation of the architecture and the applications running on it, and the
issue of model checking performance. The latter is not addressed in this
thesis. The former, however, is. Indeed, among the contributions is a strategy
to ensure understanding of a platform's cache coherence mechanisms. It relies
on micro-benchmarks. Micro-benchmarks are widely used to measure capabilities
of architectures, and so examples of such micro-benchmarks are given in the
last chapter of the related works part of this thesis.
