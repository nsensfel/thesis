\section{Analyzing Impact on Hit \& Miss}
\label{sec:analysis:hit_and_miss}
The most impactful effect interference can have on instructions is to prevent
them from being able to retrieve memory elements that would otherwise be
present in the cache. Thus, one way to look at the causes behind the WCET of a
program is to see, for each instruction, whether retrieving the memory element
they access from the cache requires the cache to fetch it or not.

The analyses in this section provide the user with an understanding of which
instructions cause caches to have to fetch memory elements (a time consumming
operation), whether this fetching always has to happen, never does, or
sometimes does (see Section~\ref{sec:analysis:instr_characterization}). It also
shows how the model can be used for a similar purpose, but focused on memory
elements themselves instead of each instruction (see
Section~\ref{sec:analysis:mem_elem_accuracy}). In effect, these analyses point
out what part of the programs should be the focus of attempts at execution time
improvements.

The same idea is employed in
Section~\ref{sec:rel_work:handling_it:accepting_it}, which presented papers
that use abstract interpretation in a way that makes all accesses be
categorized as either \textit{always-hit}, \textit{always-miss},
\textit{first-miss}, or \textit{uncategorized}. However, these papers did not
account for cache coherence. Using our model, this categorization can be
performed even on systems featuring cache coherence.

To avoid any ambiguity, the definition of both cache hit and cache miss used in
this chapter is provided below. The reason for \stallact{} actions being
mentioned in those definitions is that as long as an event has not led to any
transition other than ones which have a \stallact{} action, the component is not
considered to have acknowledged the event's existence.

\begin{definition}[Cache Hit]
\label{def:cache_hit}
A request is considered to have resulted in a cache hit if, according to the
cache coherence protocol, the first triggered transition not featuring a
\stallact{} action features a \hitact{} action.
\end{definition}

\begin{definition}[Cache Miss]
\label{def:cache_miss}
A request is considered to have resulted in a cache miss, according to the
cache coherence protocol, the first triggered transition not featuring a
\stallact{} action does not features a \hitact{} action.
\end{definition}
\begin{example}[Cache Hit and Miss]
In the MESI protocol from Figure~\ref{fig:mesi_cc_table}
(Section~\ref{sec:identification:mesi}), a \storeinstr{} instruction results in
a cache hit for a memory element currently held in the \texttt{E}, \texttt{M}
or even \texttt{MI\textsuperscript{B}} state. If the memory
element is held in the \texttt{S} or the \texttt{I} state, the \storeinstr{}
would result in a cache miss. Lastly, if the memory element is in the
\texttt{IM\textsuperscript{D}} state, the categorization of the \storeinstr{}
instruction would not be determined until some other state is reached because
of the \stallact{} action.
\end{example}

The model, as presented in Chapter~\ref{cha:modeling_cache_coherence}, does not
have any notion associated with either a cache miss or a cache hit.  This makes
it impossible to simply use UPPAAL's model checking in order to detect them.
The model has thus been modified to allow these analyses to be performed. These
additions are detailed in the next sub-section.

\subsection{Hit and Miss in the Model}
\label{sec:analyzing:hit_miss_mod}
Hitherto unmentioned are some aspects of the model ensuring that each
cache hit and cache miss is kept track of. Indeed, the cache automaton from
Section~\ref{sec:model:cache} was made so that:
\paragraph{Cache Automata:}
\begin{itemize}
\item
   The structure corresponding to a core request in the cache automata gets two
   extra fields, \lstinline!is_cache_hit! and \lstinline!instruction_addr!,
   described below.
\item
   The \lstinline!is_cache_hit! field defaults to \lstinline!true!, and will,
   once the request is completed, indicate whether than request was a cache hit
   or a cache miss. Thus, a request representation can be modified to be a
   cache miss, but not the reverse.
\item
   The \lstinline!instruction_addr! field keeps track of which program line this
   request stems from. Indeed, requests may be completed out of order, so this
   index is necessary to trace the resulting categorization back to the program
   itself.
\item
   There is a \textit{cache miss} action, which marks the current request as
   being a cache miss.
\item
   An array, \lstinline!cache_local_address_infos!, is part of the local
   variables, and has one cell per memory element. This array keeps track of the
   number of both hit and miss observed by this cache for each memory element.
\item
   When acknowledging the completion of a request,
   \lstinline!cache_local_address_infos! is updated according to whether the
   request for that memory element was a cache hit or a cache miss.
\end{itemize}

The added \textit{cache miss} action is not explicitly part of the cache
coherence protocol. Instead, the protocol switching tool presented in
Section~\ref{sec:protocol_switching} adds it automatically.
Upon generating the actions for the cache automata, the \textit{cache miss}
action is added to any action list that contains neither a \stallact{} nor
a \hitact{}.

With these variables in the model, UPPAAL's model checking can be used to
analyze cache hits and cache misses.

\subsection{Instruction Characterization}
\label{sec:analysis:instr_characterization}
The instruction characterization analysis assigns to each instruction a
category among \textit{always-hit}, \textit{always-miss}, and
\textit{uncategorized}. The \textit{first-miss} category, which was present in
Chapter~\ref{sec:rel_work:handling_it:accepting_it} cannot currently be
detected by the model, as programs do not feature loops, meaning that each
instruction is only performed once. As a result, instructions that would
otherwise be characterized as \textit{first-miss} are indistinguishable from
\textit{always-miss} ones.

This categorization allows the user to determine which instructions are
mostly unaffected by interference (all \textit{always-hit}), which ones are
strongly affected (which are among the \textit{always-miss}), and which ones are
likely to cause variation in execution time (all \textit{uncategorized}).

This analysis requires a minimum of one model checking query per instruction,
and a maximum of two per instruction (if the first one is
inconclusive). The UPPAAL queries corresponding to the verification of an
\textit{always-hit} or an \textit{always-miss} can be written for the
10th instruction of Core1 as follows:
\begin{itemize}
\item \textbf{Always Hits:}\\
$\agop{}($\lstinline!Cache1.completed_requests[0].instruction_addr!
$= 9~\texttt{imply} $
\lstinline!Cache1.completed_requests[0].is_cache_hit!$)$
\item \textbf{Always Misses:}\\
$\agop{}($\lstinline!Cache1.completed_requests[0].instruction_addr!
$= 9~\texttt{imply}~(\texttt{not} $
\lstinline!Cache1.completed_requests[0].is_cache_hit!$))$
\end{itemize}
The idea being that all requests completed by a cache are guaranteed to be in
\lstinline!completed_requests[0]! (i.e.~the top of their cache's completed
request queue) at one point or another, making it an ideal variable to analyze.
\lstinline!Cache1! corresponds to the cache being analyzed, as the
\lstinline!completed_requests! variable is local to each cache automaton.
\lstinline!9! is the index of the 10th instruction. Thus, these queries check
whether the completion of the 10th instruction in \lstinline!Cache1! always (or
never, in the case of the \textit{always-miss}) results in a cache hit.

This analysis does not separate the cache misses caused by cache coherence and
those that are innate to the program itself (e.g.~the first access to a memory
element is sure to be a cache miss). However, the analysis proposed in
Section~\ref{sec:analysis:missing_link} provides a way to determine what
categorization is the direct result of interference.

Attempting to compare the results of this analysis with the system in different
configurations (namely, in isolation compared to with shared variables) in
order to figure out which cache misses are caused by cache coherence is in no
way assured to work. Indeed, the interference can cause cache evictions, which
in turn change the effects of the cache's replacement policy: if a cache line
is freed because of interference, the cache will have one extra available slot
at this point in the program compared to the run in isolation, causing a
divergence that is likely to void the relation between categorization in
isolation and categorization with shared variables. Indeed, this may even cause
instructions to be more accurate with shared variables than in isolation if an
interference ends up evicting a memory element never to be accessed later and
causes a memory element that is accessed later to not be removed by the cache
replacement policy because of the freed cache line. Nevertheless, even without
distinction of which cache misses are caused by interference, the categorizing
the accuracy of instructions provides useful information.

\begin{example}[Application to the Model of Section~\ref{sec:analysis:demo_model}]
\label{ex:analysis:instr_chara}
Applying the instruction characterization to the model of
Section~\ref{sec:analysis:demo_model} yields the results shown in
Figure~\ref{fig:analysis:demo_chara_progs}. \textit{AM} denotes
\textit{always-miss}, \textit{AH} stands for \textit{always-hit}, and
\textit{UN} is for all other \textit{uncategorized} cases.
\begin{figure}[hbt!]
\begin{center}
\begin{subfigure}[t]{0.45\textwidth}
\centering
\begin{lstlisting}
{INSTR_STORE,  1, 0, 0} is AM (AM)
{INSTR_STORE,  2, 0, 0} is AM (AM)
{INSTR_LOAD,   1, 0, 0} is UN (AH)
{INSTR_STORE,  1, 0, 0} is UN (AH)
{INSTR_LOAD,   3, 0, 0} is AM (AM)
{INSTR_STORE,  2, 0, 0} is AM (AH)
{INSTR_LOAD,   1, 0, 0} is AH (AH)
{INSTR_STORE,  1, 0, 0} is AM (AH)
{INSTR_LOAD,   2, 0, 0} is UN (AH)
{INSTR_STORE,  2, 0, 0} is UN (AH)
{INSTR_END,    0, 0, 0}
\end{lstlisting}
\caption{Program Characterization for Core1}
\label{fig:analysis:demo_chara_prog1}
\end{subfigure}
\begin{subfigure}[t]{0.45\textwidth}
\centering
\begin{lstlisting}
{INSTR_STORE,  1, 0, 0} is AM (AM)
{INSTR_STORE,  3, 0, 0} is AM (AM)
{INSTR_LOAD,   3, 0, 0} is AH (AH)
{INSTR_STORE,  2, 0, 0} is AM (AM)
{INSTR_LOAD,   1, 0, 0} is AM (AH)
{INSTR_STORE,  2, 0, 0} is UN (AH)
{INSTR_LOAD,   3, 0, 0} is AH (AH)
{INSTR_STORE,  1, 0, 0} is AM (AH)
{INSTR_LOAD,   2, 0, 0} is UN (AH)
{INSTR_STORE,  3, 0, 0} is AM (AH)
{INSTR_END,    0, 0, 0}
\end{lstlisting}
\caption{Program Characterization for Core2}
\label{fig:analysis:demo_chara_prog2}
\end{subfigure}
\end{center}
\caption{Example of Program Characterizations}
\label{fig:analysis:demo_chara_progs}
\end{figure}
In this very small example, the cache replacement policy does not activate,
which means that the performing the analysis in isolation is able to provide a
reference that will indicate which instructions are cache misses because of
interference and which are not. Indeed, in the figures, the first indicated
characterization on each line corresponds to that obtained in the configuration
with the other program running in parallel. The second one, between
parentheses, is the result of the analysis in isolation. The following results
can be observed:
\begin{itemize}
\item
   In the program on Core1, only the instruction at line 7 is
   sure to not be negatively affected by cache coherence (\textit{always-hit}).
\item
   The execution time variability of the program on Core1 is caused by the
   instructions at lines 3, 4, 9, and 10 (\textit{uncategorized}).
\item
   For the program on Core2, the instructions at lines 3 and 7 are never
   negatively affected by cache coherence.
\item
   The execution time variability of the program on Core2 is only caused by the
   instructions at lines 6 and 9.
\item
   The result of the analysis in isolation shows that the first access for each
   memory element is always a cache miss, and all subsequent accesses are always
   cache hits.
\end{itemize}
The result of the analysis in isolation are unsurprising, because those first
accesses obtain all the permissions for the memory elements required by the
subsequent accesses. A different result would be observed if the first access
what a \loadinstr{} and a subsequent \storeinstr{} was present: both would be
classified as \textit{always-miss}. Note that this particular example does not
feature any \evictinstr{} instruction nor does it trigger the caches'
replacement policy. As indicated, the latter would impact the results of the
analysis in isolation. Considering the amount of interference in this example,
the program running on Core2 is surprisingly predictable.
\end{example}

At this point, the user has a better understanding of how each instructions
affect the program's execution time.
Section~\ref{sec:analysis:mem_elem_accuracy} provides an auxiliary analysis,
which focuses on the accuracy of memory elements instead.

\subsection{Memory Element Accuracy Analysis}
\label{sec:analysis:mem_elem_accuracy}
In addition of characterizing each instruction, the overall accuracy of accesses
made to each memory elements can also be quantified in each cache. As with the
instructions, this analysis reveals which memory elements are the cause of time
variation, which are not affected by interference, and which are frequent
causes of cache misses. Furthermore, some information on \textit{uncategorized}
instructions can sometimes be obtained by looking at the accuracy of memory
elements.

Instead of categorizing the memory elements, this analysis looks at the minimum
and maximum of cache hits and cache misses for each memory element in each
cache. The UPPAAL queries to determine either the maximum or minimum of cache
hits for the memory element at address \lstinline!3! on Cache1 are as follow:
\begin{itemize}
\item \textbf{Maximum Number of Cache Hits:}\\
   \lstinline!sup: Cache1.cache_local_address_infos[3].hit!
\item \textbf{Minimum Number of Cache Hits:}\\
   \lstinline!inf{Core1.Terminated and Core2.Terminated}: Cache1.cache_local_address_infos[3].hit!
\end{itemize}
The \lstinline!Core1.Terminated and Core2.Terminated! formula parameter used
when calculating the minimum number of cache hits ensures that only values from
after both programs have successfully completed are considered. Otherwise, the
result would always be 0, as it is the initial and lowest value on all traces.

Combined with the categorization of instructions, this analysis can provide
information on the \textit{uncategorized} instructions accessing
a memory element. Indeed, by removing the accesses pertaining to instructions
classified as \textit{always-hit} or \textit{always-miss} from the results, the
remaining numbers indicate the range of cache hits and cache misses distributed
among all \textit{uncategorized} instructions for this memory element.

\begin{example}[Application to the Model of Section~\ref{sec:analysis:demo_model}]
\label{ex:analysis:hit_miss}
Using UPPAAL to query the extrema of both cache hits and cache misses on each
cache for the model of Section~\ref{sec:analysis:demo_model} yields the results
shown in Figures~\ref{fig:analyzing:maximum_hit_miss} and
\ref{fig:analyzing:minimum_hit_miss}. The first number in each cell corresponds
to the result of the analysis for all accesses. The number between parenthesis
subtracts the accesses which are not from \textit{uncategorized} instructions
according to Example~\ref{ex:analysis:instr_chara}.

\begin{figure}[hbt!]
\begin{center}
\begin{subfigure}[t]{0.45\textwidth}
\centering
\begin{tabular}{|c|c|c|c|c|}
\cline{2-5}
\multicolumn{1}{c|}{} &
\multicolumn{2}{c|}{Cache1} &
\multicolumn{2}{c|}{Cache2} \\
\multicolumn{1}{c|}{} & Hit & Miss & Hit & Miss\\
\hline
Address 1 & 3 (2) & 4 (2) & 0 (0) & 3 (0)\\
\hline
Address 2 & 2 (2) & 4 (2) & 1 (1) & 2 (1)\\
\hline
Address 3 & 0 (0) & 1 (0) & 2 (0) & 2 (0)\\
\hline
\end{tabular}
\caption{Maximum}
\label{fig:analyzing:maximum_hit_miss}
\end{subfigure}
\begin{subfigure}[t]{0.45\textwidth}
\centering
\begin{tabular}{|c|c|c|c|c|}
\cline{2-5}
\multicolumn{1}{c|}{} &
\multicolumn{2}{c|}{Cache1} &
\multicolumn{2}{c|}{Cache2} \\
\multicolumn{1}{c|}{} & Hit & Miss & Hit & Miss\\
\hline
Address 1 & 1 (0) & 2 (0) & 0 (0) & 3 (0)\\
\hline
Address 2 & 0 (0) & 2 (0) & 1 (1) & 2 (1)\\
\hline
Address 3 & 0 (0) & 1 (0) & 2 (0) & 2 (0)\\
\hline
\end{tabular}
\caption{Minimum}
\label{fig:analyzing:minimum_hit_miss}
\end{subfigure}
\caption{Cache Hit/Miss Extrema for each Memory Element}
\end{center}
\end{figure}

The results of this analysis can be read as follows:
\begin{itemize}
\item
   On Cache1, the address 1 has 1 \textit{always-hit} and 2
   \textit{always-miss} instructions. Thus, the number corresponding to the
   accesses from \textit{uncategorized} instructions are obtained by removing
   1 from the \textit{Hit} columns and 2 from the \textit{Miss} columns.
\item
   For address 2, the program that uses Cache1 has 2 \textit{always-miss}
   instructions, which are thus removed from the \textit{Miss} columns.
\item
   Address 3 has no \textit{uncategorized} accesses, so there is nothing
   remaining.
\item
   Address 1 on Cache2 is only accessed by 3 \textit{always-miss}, which are
   thus removed. This leaves 0 hit or miss for \textit{uncategorized} accesses,
   as expected since there are none in  Example~\ref{ex:analysis:instr_chara}.
\item
   Address 2 on Cache2 only removes a single \textit{always-miss}.
\end{itemize}

From these results, the following observations can be made:
\begin{itemize}
\item The memory element at address 1 is the main cause of cache misses.
\item The memory element at address 3 is the least problematic one.
\item
   The results of the analyses for Cache2 show the same values for both minimum
   and maximum cache hits and misses. This adds information to the
   categorization done in Example~\ref{ex:analysis:instr_chara}. Indeed, this
   means that despite being neither \textit{always-hit} nor
   \textit{always-miss}, the \textit{uncategorized} instructions of Cache2
   never both hit (minimum and maximum of 1 hit on any execution for
   \textit{uncategorized} instructions) nor both miss (minimum and maximum of 1
   miss on any execution for \textit{uncategorized} instructions) on the same
   execution.
\item
   On Cache1, the \textit{uncategorized} accesses to address 1 can both be miss
   or hit on the same execution. Further analysis would be required in order to
   determine if every execution has them both hit or both miss.
\item
   This also holds true from the \textit{uncategorized} accesses to address 2
   on Cache1.
\end{itemize}
\end{example}

This auxiliary analysis provided some additional information on which memory
elements should be the focus of the user when attempting to mitigate the
interference generated by cache coherence. It also provided a way to obtain a
bit more information on \textit{uncategorized} instructions.

In order to understand what causes instructions to result in cache misses when
cache coherence is active, the next section studies the external queries
received by caches.
