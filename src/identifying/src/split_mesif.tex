\section{Hypothetical Split-Transaction MESIF Protocol}
\label{sec:identification:mesif}
As the hypothetical protocol has been invalidated, a new one matching the
observations must be defined. The difference of behavior between \benchs{}
and \benchx{} points to a MESIF protocol.

\begin{figure}[htb!]
\begin{center}
\resizebox{\textwidth}{!}{%
\begin{tabular}{|l||c|c|c||c||c|c||c|c|c|}
 \hline
 \multicolumn{10}{|c|}{\textbf{Cache Controller}}\\

 \hline

 \multirow{2}{*}{\textbf{State}}
 & \multicolumn{3}{c||}{\textbf{Core Request}}
 & \begin{tabular}{@{}c@{}}
      \textbf{Interconnect}\\
      \textbf{Access}
   \end{tabular}
 & \multicolumn{2}{c||}{\textbf{Data Reply}}
 & \multicolumn{3}{c|}{\textbf{Received Queries}}
 \\

%%%%%%%%%%%%%%%%%%%%%%%%%%%%%%%%%%%%%%%%%%%%%%%%%%%%%%%%%%%%%%%%%%%%%%%%%%%%%%%%
 & \texttt{load} & \texttt{store} & \texttt{evict}
 &
 & \texttt{data} & \texttt{data-e}
 & \texttt{GetS} & \texttt{GetM} & \texttt{PutM}
 \\
 \hline

%%%%%%%%%%%%%%%%%%%%%%%%%%%%%%%%%%%%%%%%%%%%%%%%%%%%%%%%%%%%%%%%%%%%%%%%%%%%%%%%
 \texttt{I}

 % Load/Store/Evict
 & \sendqueryact{\texttt{GetS}}, \setstateact{\texttt{IF\textsuperscript{BD}}}
 & \sendqueryact{\texttt{GetM}}, \setstateact{\texttt{IM\textsuperscript{BD}}}
 & \texttt{hit}

 % Seeing Own Query
 & \disablecell{}

 % Data Reply
 & \disablecell{}
 & \disablecell{}

 % GetS/GetM/PutM
 & -
 & -
 & -
 \\
 \hline

%%%%%%%%%%%%%%%%%%%%%%%%%%%%%%%%%%%%%%%%%%%%%%%%%%%%%%%%%%%%%%%%%%%%%%%%%%%%%%%%
 \texttt{IF\textsuperscript{BD}}

 % Load/Store/Evict
 & \texttt{stall}
 & \texttt{stall}
 & \texttt{stall}

 % Seeing Own Query
 & \setstateact{\texttt{IEoF\textsuperscript{D}}}

 % Data Reply
 & \setstateact{\texttt{IF\textsuperscript{B}}}
 & \setstateact{\texttt{IE\textsuperscript{B}}}

 % GetS/GetM/PutM
 & -
 & -
 & -
 \\
 \hline

%%%%%%%%%%%%%%%%%%%%%%%%%%%%%%%%%%%%%%%%%%%%%%%%%%%%%%%%%%%%%%%%%%%%%%%%%%%%%%%%
 \texttt{IF\textsuperscript{B}}

 % Load/Store/Evict
 & \texttt{stall}
 & \texttt{stall}
 & \texttt{stall}

 % Seeing Own Query
 & \setstateact{\texttt{F}}

 % Data Reply
 & \disablecell{}
 & \disablecell{}

 % GetS/GetM/PutM
 & -
 & -
 & \disablecell{}
 \\
 \hline

%%%%%%%%%%%%%%%%%%%%%%%%%%%%%%%%%%%%%%%%%%%%%%%%%%%%%%%%%%%%%%%%%%%%%%%%%%%%%%%%
 \texttt{IEoF\textsuperscript{D}}

 % Load/Store/Evict
 & \texttt{stall}
 & \texttt{stall}
 & \texttt{stall}

 % Seeing Own Query
 & \disablecell{}

 % Data Reply
 & \setstateact{\texttt{F}}
 & \setstateact{\texttt{E}}

 % GetS/GetM/PutM
 & \storereplytoact{},
   \setstateact{\texttt{IS\textsuperscript{D}}}
 &
   \storereplytoact{},
   \setstateact{\texttt{IS\textsuperscript{D}I}}
 & \disablecell{}
 \\
 \hline

%%%%%%%%%%%%%%%%%%%%%%%%%%%%%%%%%%%%%%%%%%%%%%%%%%%%%%%%%%%%%%%%%%%%%%%%%%%%%%%%
 \texttt{IS\textsuperscript{D}}

 % Load/Store/Evict
 & \texttt{stall}
 & \texttt{stall}
 & \texttt{stall}

 % Seeing Own Query
 & \disablecell{}

 % Data Reply
 & \begin{tabular}{@{}c@{}}
      \sendqueryact{\texttt{r}}{\texttt{data}},\\
      \resetreplytoact{},\\
      \setstateact{\texttt{S}}
   \end{tabular}

 & \begin{tabular}{@{}c@{}}
      \sendqueryact{\texttt{r}}{\texttt{data}},\\
      \sendqueryact{\texttt{m}}{\texttt{no-data}},\\
      \resetreplytoact{},
      \setstateact{\texttt{S}}
   \end{tabular}

 % GetS/GetM/PutM
 & -
 & \setstateact{\texttt{IS\textsuperscript{D}I}}
 & \disablecell{}
 \\
 \hline


%%%%%%%%%%%%%%%%%%%%%%%%%%%%%%%%%%%%%%%%%%%%%%%%%%%%%%%%%%%%%%%%%%%%%%%%%%%%%%%%
 \texttt{IS\textsuperscript{D}I}

 % Load/Store/Evict
 & \texttt{stall}
 & \texttt{stall}
 & \texttt{stall}

 % Seeing Own Query
 & \disablecell{}

 % Data Reply
 &
   \begin{tabular}{@{}c@{}}
      load hit,\\
      \sendqueryact{\texttt{r}}{\texttt{data}},\\
      \resetreplytoact{},\\
      \setstateact{\texttt{I}}
   \end{tabular}
 &
   \begin{tabular}{@{}c@{}}
      load hit,\\
      \sendqueryact{\texttt{r}}{\texttt{data}},\\
      \resetreplytoact{},\\
      \sendqueryact{\texttt{m}}{\texttt{no-data}},\\
      \setstateact{\texttt{I}}
   \end{tabular}

 % GetS/GetM/PutM
 & -
 & -
 & \disablecell{}
 \\
 \hline

%%%%%%%%%%%%%%%%%%%%%%%%%%%%%%%%%%%%%%%%%%%%%%%%%%%%%%%%%%%%%%%%%%%%%%%%%%%%%%%%
 \texttt{IM\textsuperscript{BD}}

 % Load/Store/Evict
 & \texttt{stall}
 & \texttt{stall}
 & \texttt{stall}

 % Seeing Own Query
 & \setstateact{\texttt{IM\textsuperscript{D}}}

 % Data Reply
 & \setstateact{\texttt{IM\textsuperscript{B}}}
 & \disablecell{}

 % GetS/GetM/PutM
 & -
 & -
 & -
 \\
 \hline

%%%%%%%%%%%%%%%%%%%%%%%%%%%%%%%%%%%%%%%%%%%%%%%%%%%%%%%%%%%%%%%%%%%%%%%%%%%%%%%%
 \texttt{IM\textsuperscript{B}}

 % Load/Store/Evict
 & \texttt{stall}
 & \texttt{stall}
 & \texttt{stall}

 % Seeing Own Query
 & \setstateact{\texttt{M}}

 % Data Reply
 & \disablecell{}
 & \disablecell{}

 % GetS/GetM/PutM
 & -
 & -
 & -
 \\
 \hline

%%%%%%%%%%%%%%%%%%%%%%%%%%%%%%%%%%%%%%%%%%%%%%%%%%%%%%%%%%%%%%%%%%%%%%%%%%%%%%%%
 \texttt{IM\textsuperscript{D}}

 % Load/Store/Evict
 & \texttt{stall}
 & \texttt{stall}
 & \texttt{stall}

 % Seeing Own Query
 & \disablecell{}

 % Data Reply
 & \setstateact{\texttt{M}}
 & \disablecell{}

 % GetS/GetM/PutM
 & \storereplytoact{}, \setstateact{\texttt{IM\textsuperscript{D}S}}
 & \storereplytoact{}, \setstateact{\texttt{IM\textsuperscript{D}I}}
 & \disablecell{}
 \\
 \hline

%%%%%%%%%%%%%%%%%%%%%%%%%%%%%%%%%%%%%%%%%%%%%%%%%%%%%%%%%%%%%%%%%%%%%%%%%%%%%%%%
 \texttt{IM\textsuperscript{D}I}

 % Load/Store/Evict
 & \texttt{stall}
 & \texttt{stall}
 & \texttt{stall}

 % Seeing Own Query
 & \disablecell{}

 % Data Reply
 &
   \begin{tabular}{@{}c@{}}
      store hit,\\
      \sendqueryact{\texttt{r}}{\texttt{data}},\\
      \resetreplytoact{}, \setstateact{\texttt{I}}
   \end{tabular}
 & \disablecell{}

 % GetS/GetM/PutM
 & -
 & -
 & \disablecell{}
 \\
 \hline

%%%%%%%%%%%%%%%%%%%%%%%%%%%%%%%%%%%%%%%%%%%%%%%%%%%%%%%%%%%%%%%%%%%%%%%%%%%%%%%%
 \texttt{IM\textsuperscript{D}S}

 % Load/Store/Evict
 & \texttt{stall}
 & \texttt{stall}
 & \texttt{stall}

 % Seeing Own Query
 & \disablecell{}

 % Data Reply
 &
   \begin{tabular}{@{}c@{}}
      store hit,\\
      \sendqueryact{\texttt{r}}{\texttt{data}},\\
      \sendqueryact{\texttt{m}}{\texttt{data}},\\
      \resetreplytoact{}, \setstateact{\texttt{S}}
   \end{tabular}
 & \disablecell{}

 % GetS/GetM/PutM
 & -
 & \setstateact{\texttt{IM\textsuperscript{D}SI}}
 & \disablecell{}
 \\
 \hline

%%%%%%%%%%%%%%%%%%%%%%%%%%%%%%%%%%%%%%%%%%%%%%%%%%%%%%%%%%%%%%%%%%%%%%%%%%%%%%%%
 \texttt{IM\textsuperscript{D}SI}

 % Load/Store/Evict
 & \texttt{stall}
 & \texttt{stall}
 & \texttt{stall}

 % Seeing Own Query
 & \disablecell{}

 % Data Reply
 & \begin{tabular}{@{}c@{}}
      store hit,\\
      \sendqueryact{\texttt{r}}{\texttt{data}},\\
      \sendqueryact{\texttt{m}}{\texttt{data}},\\
      \resetreplytoact{}, \setstateact{\texttt{I}}
   \end{tabular}
 & \disablecell{}

 % GetS/GetM/PutM
 & -
 & -
 & \disablecell{}
 \\
 \hline

%%%%%%%%%%%%%%%%%%%%%%%%%%%%%%%%%%%%%%%%%%%%%%%%%%%%%%%%%%%%%%%%%%%%%%%%%%%%%%%%
 \texttt{S}

 % Load/Store/Evict
 & \texttt{hit}
 & \sendqueryact{\texttt{GetM}}, \setstateact{\texttt{SM\textsuperscript{BD}}}
 & \texttt{hit}, \setstateact{\texttt{I}}

 % Seeing Own Query
 & \disablecell{}

 % Data Reply
 & \disablecell{}
 & \disablecell{}

 % GetS/GetM/PutM
 & -
 & \setstateact{\texttt{I}}
 & \disablecell{}
 \\
 \hline

%%%%%%%%%%%%%%%%%%%%%%%%%%%%%%%%%%%%%%%%%%%%%%%%%%%%%%%%%%%%%%%%%%%%%%%%%%%%%%%%
 \texttt{F}

 % Load/Store/Evict
 & \texttt{hit}
 & \sendqueryact{\texttt{GetM}}, \setstateact{\texttt{FM\textsuperscript{B}}}
 & \sendqueryact{\texttt{PutM}}, \setstateact{\texttt{FI\textsuperscript{B}}}

 % Seeing Own Query
 & \disablecell{}

 % Data Reply
 & \disablecell{}
 & \disablecell{}

 % GetS/GetM/PutM
 &
   \begin{tabular}{@{}c@{}}
      \sendqueryact{\texttt{s}}{\texttt{data}}, \setstateact{\texttt{S}}
   \end{tabular}
 &
   \begin{tabular}{@{}c@{}}
      \sendqueryact{\texttt{s}}{\texttt{data}}, \setstateact{\texttt{I}}
   \end{tabular}
 & \disablecell{}
 \\
 \hline


%%%%%%%%%%%%%%%%%%%%%%%%%%%%%%%%%%%%%%%%%%%%%%%%%%%%%%%%%%%%%%%%%%%%%%%%%%%%%%%%
 \texttt{SM\textsuperscript{BD}}

 % Load/Store/Evict
 & \texttt{hit}
 & \texttt{stall}
 & \texttt{stall}

 % Seeing Own Query
 & \setstateact{\texttt{SM\textsuperscript{D}}}

 % Data Reply
 & \setstateact{\texttt{SM\textsuperscript{B}}}
 & \disablecell{}

 % GetS/GetM/PutM
 & -
 & \setstateact{\texttt{IM\textsuperscript{BD}}}
 & \disablecell{}
 \\
 \hline

%%%%%%%%%%%%%%%%%%%%%%%%%%%%%%%%%%%%%%%%%%%%%%%%%%%%%%%%%%%%%%%%%%%%%%%%%%%%%%%%
 \texttt{FM\textsuperscript{B}}

 % Load/Store/Evict
 & \texttt{hit}
 & \texttt{stall}
 & \texttt{stall}

 % Seeing Own Query
 & \setstateact{\texttt{M}}

 % Data Reply
 & \disablecell{}
 & \disablecell{}

 % GetS/GetM/PutM
 & \begin{tabular}{@{}c@{}}
      \sendqueryact{\texttt{s}}{\texttt{data}},\\
      \setstateact{\texttt{SM\textsuperscript{BD}}}
   \end{tabular}
 & \begin{tabular}{@{}c@{}}
      \sendqueryact{\texttt{s}}{\texttt{data}},\\
      \setstateact{\texttt{IM\textsuperscript{B}}}
   \end{tabular}
 & \disablecell{}
 \\
 \hline

%%%%%%%%%%%%%%%%%%%%%%%%%%%%%%%%%%%%%%%%%%%%%%%%%%%%%%%%%%%%%%%%%%%%%%%%%%%%%%%%
 \texttt{SM\textsuperscript{B}}

 % Load/Store/Evict
 & \texttt{hit}
 & \texttt{stall}
 & \texttt{stall}

 % Seeing Own Query
 & \setstateact{\texttt{M}}

 % Data Reply
 & \disablecell{}
 & \disablecell{}

 % GetS/GetM/PutM
 & -
 & \setstateact{\texttt{IM\textsuperscript{B}}}
 & \disablecell{}
 \\
 \hline


%%%%%%%%%%%%%%%%%%%%%%%%%%%%%%%%%%%%%%%%%%%%%%%%%%%%%%%%%%%%%%%%%%%%%%%%%%%%%%%%
 \texttt{SM\textsuperscript{D}}

 % Load/Store/Evict
 & \texttt{hit}
 & \texttt{stall}
 & \texttt{stall}

 % Seeing Own Query
 & \disablecell{}

 % Data Reply
 & store hit, \setstateact{\texttt{M}}
 & \disablecell{}

 % GetS/GetM/PutM
 & \storereplytoact{}, \setstateact{\texttt{SM\textsuperscript{D}S}}
 & \storereplytoact{}, \setstateact{\texttt{SM\textsuperscript{D}I}}
 & \disablecell{}
 \\
 \hline

%%%%%%%%%%%%%%%%%%%%%%%%%%%%%%%%%%%%%%%%%%%%%%%%%%%%%%%%%%%%%%%%%%%%%%%%%%%%%%%%
 \texttt{SM\textsuperscript{D}I}

 % Load/Store/Evict
 & \texttt{hit}
 & \texttt{stall}
 & \texttt{stall}

 % Seeing Own Query
 & \disablecell{}

 % Data Reply
 & \begin{tabular}{@{}c@{}}
      store hit,\\
      \sendqueryact{\texttt{r}}{\texttt{data}},\\
      \resetreplytoact{}, \setstateact{\texttt{I}}
   \end{tabular}
 & \disablecell{}

 % GetS/GetM/PutM
 & -
 & -
 & \disablecell{}
 \\
 \hline

%%%%%%%%%%%%%%%%%%%%%%%%%%%%%%%%%%%%%%%%%%%%%%%%%%%%%%%%%%%%%%%%%%%%%%%%%%%%%%%%
 \texttt{SM\textsuperscript{D}S}

 % Load/Store/Evict
 & \texttt{hit}
 & \texttt{stall}
 & \texttt{stall}

 % Seeing Own Query
 & \disablecell{}

 % Data Reply
 & \begin{tabular}{@{}c@{}}
      store hit,\\
      \sendqueryact{\texttt{r}}{\texttt{data}},\\
      \sendqueryact{\texttt{m}}{\texttt{data}},\\
      \resetreplytoact{}, \setstateact{\texttt{S}}
   \end{tabular}
 & \disablecell{}

 % GetS/GetM/PutM
 & -
 & \setstateact{\texttt{SM\textsuperscript{D}SI}}
 & \disablecell{}
 \\
 \hline

%%%%%%%%%%%%%%%%%%%%%%%%%%%%%%%%%%%%%%%%%%%%%%%%%%%%%%%%%%%%%%%%%%%%%%%%%%%%%%%%
 \texttt{SM\textsuperscript{D}SI}

 % Load/Store/Evict
 & \texttt{hit}
 & \texttt{stall}
 & \texttt{stall}

 % Seeing Own Query
 & \disablecell{}

 % Data Reply
 & \begin{tabular}{@{}c@{}}
      store hit,\\
      \sendqueryact{\texttt{r}}{\texttt{data}},\\
      \sendqueryact{\texttt{m}}{\texttt{data}},\\
      \resetreplytoact{}, \setstateact{\texttt{I}}
   \end{tabular}
 & \disablecell{}

 % GetS/GetM/PutM
 & -
 & -
 & \disablecell{}
 \\
 \hline

%%%%%%%%%%%%%%%%%%%%%%%%%%%%%%%%%%%%%%%%%%%%%%%%%%%%%%%%%%%%%%%%%%%%%%%%%%%%%%%%
 \texttt{M}

 % Load/Store/Evict
 & \texttt{hit}
 & \texttt{hit}
 & \sendqueryact{\texttt{PutM}}, \setstateact{\texttt{MI\textsuperscript{B}}}

 % Seeing Own Query
 & \disablecell{}

 % Data Reply
 & \disablecell{}
 & \disablecell{}

 % GetS/GetM/PutM
 &
   \begin{tabular}{@{}c@{}}
      \sendqueryact{\texttt{m}}{\texttt{data}},\\
      \sendqueryact{\texttt{s}}{\texttt{data}}, \setstateact{\texttt{S}}
   \end{tabular}
 & \sendqueryact{\texttt{s}}{\texttt{data}}, \setstateact{\texttt{I}}
 & \disablecell{}
 \\
 \hline

%%%%%%%%%%%%%%%%%%%%%%%%%%%%%%%%%%%%%%%%%%%%%%%%%%%%%%%%%%%%%%%%%%%%%%%%%%%%%%%%
 \texttt{MI\textsuperscript{B}}

 % Load/Store/Evict
 & \texttt{hit}
 & \texttt{hit}
 & \texttt{stall}

 % Seeing Own Query
 & \sendqueryact{\texttt{m}}{\texttt{data}}, \setstateact{\texttt{I}}

 % Data Reply
 & \disablecell{}
 & \disablecell{}

 % GetS/GetM/PutM
 &
   \begin{tabular}{@{}c@{}}
      \sendqueryact{\texttt{m}}{\texttt{data}},\\
      \sendqueryact{\texttt{s}}{\texttt{data}},
      \setstateact{\texttt{II\textsuperscript{B}}}
   \end{tabular}
 & \sendqueryact{\texttt{s}}{\texttt{data}},
   \setstateact{\texttt{II\textsuperscript{B}}}
 & \disablecell{}
 \\
 \hline

%%%%%%%%%%%%%%%%%%%%%%%%%%%%%%%%%%%%%%%%%%%%%%%%%%%%%%%%%%%%%%%%%%%%%%%%%%%%%%%%
 \texttt{II\textsuperscript{B}}

 % Load/Store/Evict
 & \texttt{stall}
 & \texttt{stall}
 & \texttt{stall}

 % Seeing Own Query
 & \setstateact{\texttt{I}}

 % Data Reply
 & \disablecell{}
 & \disablecell{}

 % GetS/GetM/PutM
 & -
 & -
 & -
 \\
 \hline

%%%%%%%%%%%%%%%%%%%%%%%%%%%%%%%%%%%%%%%%%%%%%%%%%%%%%%%%%%%%%%%%%%%%%%%%%%%%%%%%
 \texttt{E}

 % Load/Store/Evict
 & \texttt{hit}
 & \texttt{hit}, \setstateact{\texttt{M}}
 & \sendqueryact{\texttt{PutM}}, \setstateact{\texttt{EI\textsuperscript{B}}}

 % Seeing Own Query
 & \disablecell{}

 % Data Reply
 & \disablecell{}
 & \disablecell{}

 % GetS/GetM/PutM
 &
   \begin{tabular}{@{}c@{}}
      \sendqueryact{\texttt{m}}{\texttt{no-data}},\\
      \sendqueryact{\texttt{s}}{\texttt{data}}, \setstateact{\texttt{S}}
   \end{tabular}
 & \sendqueryact{\texttt{s}}{\texttt{data}}, \setstateact{\texttt{I}}
 & \disablecell{}
 \\
 \hline

%%%%%%%%%%%%%%%%%%%%%%%%%%%%%%%%%%%%%%%%%%%%%%%%%%%%%%%%%%%%%%%%%%%%%%%%%%%%%%%%
 \texttt{IE\textsuperscript{B}}

 % Load/Store/Evict
 & \texttt{stall}
 & \texttt{stall}
 & \texttt{stall}

 % Seeing Own Query
 & \setstateact{\texttt{E}}

 % Data Reply
 & \disablecell{}
 & \disablecell{}

 % GetS/GetM/PutM
 & -
 & -
 & -
 \\
 \hline

%%%%%%%%%%%%%%%%%%%%%%%%%%%%%%%%%%%%%%%%%%%%%%%%%%%%%%%%%%%%%%%%%%%%%%%%%%%%%%%%
 \texttt{EI\textsuperscript{B}}

 % Load/Store/Evict
 & \texttt{hit}
 & \hitact{}, \setstateact{\texttt{MI\textsuperscript{B}}}
 & \texttt{stall}

 % Seeing Own Query
 & \sendqueryact{\texttt{m}}{\texttt{no-data}}, \setstateact{\texttt{I}}

 % Data Reply
 & \disablecell{}
 & \disablecell{}

 % GetS/GetM/PutM
 &
   \begin{tabular}{@{}c@{}}
      \sendqueryact{\texttt{m}}{\texttt{no-data}},\\
      \sendqueryact{\texttt{s}}{\texttt{data}},
      \setstateact{\texttt{II\textsuperscript{B}}}
   \end{tabular}
 & \sendqueryact{\texttt{s}}{\texttt{data}},
   \setstateact{\texttt{II\textsuperscript{B}}}
 & \disablecell{}
 \\
 \hline

%%%%%%%%%%%%%%%%%%%%%%%%%%%%%%%%%%%%%%%%%%%%%%%%%%%%%%%%%%%%%%%%%%%%%%%%%%%%%%%%
 \texttt{FI\textsuperscript{B}}

 % Load/Store/Evict
 & \texttt{hit}
 & \texttt{stall}
 & \texttt{stall}

 % Seeing Own Query
 & \setstateact{\texttt{I}}

 % Data Reply
 & \disablecell{}
 & \disablecell{}

 % GetS/GetM/PutM
 &
   \begin{tabular}{@{}c@{}}
      \sendqueryact{\texttt{s}}{\texttt{data}},
      \setstateact{\texttt{II\textsuperscript{B}}}
   \end{tabular}
 & \sendqueryact{\texttt{s}}{\texttt{data}},
   \setstateact{\texttt{II\textsuperscript{B}}}
 & \disablecell{}
 \\
 \hline

% \multicolumn{1}{l|}{}
% & \multicolumn{6}{c||}{\textbf{Handling Requests}}
% & \multicolumn{3}{c|}{\textbf{Handling Queries}}
% \\
% \cline{2-10}
\end{tabular}

}
\end{center}
\caption{Description of the cache controller for the MESIF protocol}
\label{fig:mesif_cc_table}
\end{figure}

\begin{figure}[htb!]
\begin{center}
\resizebox{\textwidth}{!}{%
\begin{tabular}{|l||c|c|c|c||c|c|}
 \hline

 \multicolumn{7}{|c|}{\textbf{Coherence Manager}}\\

 \hline

 \multirow{2}{*}{\textbf{State}}
 & \multicolumn{4}{c||}{\textbf{Received Queries}}
 & \multicolumn{2}{c|}{\textbf{Data Reply}}
 \\

%%%%%%%%%%%%%%%%%%%%%%%%%%%%%%%%%%%%%%%%%%%%%%%%%%%%%%%%%%%%%%%%%%%%%%%%%%%%%%%%
 & \texttt{GetS}
 & \texttt{GetM}
 &
   \begin{tabular}{@{}c@{}}
      \texttt{PutM}\\
      (Owner)
   \end{tabular}
 &
   \begin{tabular}{@{}c@{}}
      \texttt{PutM}\\
      (Other)
   \end{tabular}
 & \texttt{data}
 & \texttt{no-data}
 \\
 \hline

%%%%%%%%%%%%%%%%%%%%%%%%%%%%%%%%%%%%%%%%%%%%%%%%%%%%%%%%%%%%%%%%%%%%%%%%%%%%%%%%
 \texttt{I}

 % GetS
 & \texttt{read},
   \sendqueryact{\texttt{s}}{\texttt{data-e}},
   \storeowneract{},
   \setstateact{\texttt{M}}

 % GetM
 & \sendqueryact{\texttt{s}}{\texttt{data}},
   \storeowneract{},
   \setstateact{\texttt{M}}

 % PutM (owner)
 & \disablecell{}

 % PutM (other)
 & -

 % Data & Data-Exclusive
 & \disablecell{}
 & \disablecell{}
 \\
 \hline

%%%%%%%%%%%%%%%%%%%%%%%%%%%%%%%%%%%%%%%%%%%%%%%%%%%%%%%%%%%%%%%%%%%%%%%%%%%%%%%%
 \texttt{M}

 % GetS
 & \storeowneract{}, \setstateact{\texttt{F\textsuperscript{D}}}

 % GetM
 & \storeowneract{}

 % PutM (owner)
 & \resetowneract{}, \setstateact{\texttt{I\textsuperscript{D}}}

 % PutM (other)
 & -

 % Data & Data-Exclusive
 & write, \setstateact{\texttt{IoF\textsuperscript{B}}}
 & \setstateact{\texttt{IoF\textsuperscript{B}}}
 \\
 \hline

%%%%%%%%%%%%%%%%%%%%%%%%%%%%%%%%%%%%%%%%%%%%%%%%%%%%%%%%%%%%%%%%%%%%%%%%%%%%%%%%
 \texttt{I\textsuperscript{D}}

 % GetS
 & \texttt{stall}

 % GetM
 & \texttt{stall}

 % PutM (owner)
 & \texttt{stall}

 % PutM (other)
 & -

 % Data & Data-Exclusive
 & write, resume, \setstateact{\texttt{I}}
 & resume, \setstateact{\texttt{I}}
 \\
 \hline

%%%%%%%%%%%%%%%%%%%%%%%%%%%%%%%%%%%%%%%%%%%%%%%%%%%%%%%%%%%%%%%%%%%%%%%%%%%%%%%%
 \texttt{F\textsuperscript{D}}

 % GetS
 & \texttt{stall}

 % GetM
 & \texttt{stall}

 % PutM (owner)
 & \texttt{stall}

 % PutM (other)
 & -

 % Data & Data-Exclusive
 & write, resume, \setstateact{\texttt{F}}
 & resume, \setstateact{\texttt{F}}
 \\
 \hline

%%%%%%%%%%%%%%%%%%%%%%%%%%%%%%%%%%%%%%%%%%%%%%%%%%%%%%%%%%%%%%%%%%%%%%%%%%%%%%%%
 \texttt{IoF\textsuperscript{B}}

 % GetS
 & \storeowneract{}, \setstateact{\texttt{F}}

 % GetM
 & \storeowneract{}

 % PutM (owner)
 & \resetowneract{}, \setstateact{\texttt{I}}

 % PutM (other)
 & -

 % Data & Data-Exclusive
 & write
 & -
 \\
 \hline

%%%%%%%%%%%%%%%%%%%%%%%%%%%%%%%%%%%%%%%%%%%%%%%%%%%%%%%%%%%%%%%%%%%%%%%%%%%%%%%%
 \texttt{S}

 % GetS
 &
   \texttt{read},
   \sendqueryact{\texttt{s}}{\texttt{data}},
   \setstateact{\texttt{F}}

 % GetM
 & \sendqueryact{\texttt{s}}{\texttt{data}}, \storeowneract{},
   \setstateact{\texttt{M}}

 % PutM (owner)
 & \disablecell{}

 % PutM (other)
 & -

 % Data & Data-Exclusive
 & \disablecell{}
 & \disablecell{}
 \\
 \hline

%%%%%%%%%%%%%%%%%%%%%%%%%%%%%%%%%%%%%%%%%%%%%%%%%%%%%%%%%%%%%%%%%%%%%%%%%%%%%%%%
 \texttt{F}

 % GetS
 & \storeowneract{}

 % GetM
 & \storeowneract{}, \setstateact{\texttt{M}}

 % PutM (owner)
 & \resetowneract{}, \setstateact{\texttt{S}}

 % PutM (other)
 & -

 % Data & Data-Exclusive
 & write, \setstateact{\texttt{IoF\textsuperscript{B}}}
 & \setstateact{\texttt{IoF\textsuperscript{B}}}
 \\
 \hline
\end{tabular}

}
\end{center}
\caption{Description of the coherence manager for the MESIF protocol}
\label{fig:mesif_cmgr_table}
\end{figure}

Figures~\ref{fig:mesif_cc_table} and \ref{fig:mesif_cmgr_table} show a formal
definition of the MESIF protocol.  Introduced in \cite{mesif}, this protocols
adds a \textit{Forward} stable state, which is equivalent to a \textit{Shared}
state with the added constraint of being responsible for the propagation of the
memory element's current value.  This makes it possible to avoid reading from
the system's main memory even when multiple caches hold the same memory element.
However, unlike the \textit{Exclusive} state, it does not allow the cache to
upgrade to a \textit{Modified} state by itself, since the other caches still do
have to be informed that their copies are out-of-date.

As with any stable state that gives a cache the responsibility of propagating
the memory element's current value, the challenge lies in determining when a
cache can enter that state, and making sure that the responsibility is properly
transferred when the cache leaves it. The coherence manager keeps track of which
cache holds memory elements in the \textit{Forward} state. As this cache cannot
actually make modifications while in this state, informing the coherence manager
that it was left does not require sending any kind of \texttt{data} message: a
simple \texttt{PutM} query broadcast is sufficient.

A cache moving from \textit{Forward} to \textit{Modified} still has to broadcast
a \texttt{GetM} query and process all the queries that preceded before
proceeding. This is unclear in the definitions of the protocol I have seen so
far, but this version of the protocol assumes that if the cache still is
responsible for the propagation of the memory element when it sees its own
\texttt{GetM} query (meaning that it stayed in the
\texttt{FM\textsuperscript{B}} state), then it should be able to simply move to
the \textit{Modified} state without receiving any \texttt{data} reply. However,
if the responsibility was lost (because of either an external \texttt{GetS} or
\texttt{GetM} query), then it will need to re-acquire the current value of the
memory element as a \texttt{data} reply before entering the \textit{Modified}
state.
