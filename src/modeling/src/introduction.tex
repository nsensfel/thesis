This chapter presents a UPPAAL model\footnote{Available at
\url{https://github.com/nsensfel/phylog-cache-coherence}}
for the analysis of the effects of cache
coherence in multi-core processors. The goal is to create a formal model in
order to perform automatic analyses (which are described in
Chapter~\ref{cha:exposing_interference}), while ensuring that:
\begin{itemize}
\item The model is as generic as possible in how it models the coherence
protocol, making it easy to switch protocol.
\item The protocols are modeled in detail, taking into account all transient
states and being defined for split-transaction buses.
\end{itemize}

The approach chosen is similar to the papers presented in
Chapter~\ref{cha:analyzing_rel_work}: use a network of small timed
automaton, each representing a component, so that the model of the system ends
up easily readable, modular, and re-usable.

The chapter starts by an overview of the modeling strategy
(Section~\ref{sec:modeling:strategy}). The model is then seen through its
communication channels in Section~\ref{sec:modeling:channels}, which provides
an understanding of how the components interact. The sections that follow each
provide a precise description of each component's automaton. Once all
components have been covered, Section~\ref{sec:protocol_switching} presents a
tool to allow the coherence protocol being used by the model to be
automatically changed to another.
