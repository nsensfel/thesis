\section{UPPAAL and Networks of Timed Automata}
This section summarizes the differences brought by timed automata and networks
of timed automata to the classical automata described previously. The features
presented here are those found within UPPAAL, a modeling tool for networks of
timed automata introduced in \cite{Bengtsson:1996:UTS:239587.239611}. The
addition of time leads to states in which a new type of variables, clocks,
evolve even when no transition is activated. To account for this,
\textit{states} are now referred to as \textit{locations} instead. Transitions
are all instantaneous. Readers looking for further information on timed
automata are encouraged to check out the papers in which they were first
described (\cite{143902} and \cite{tcs126(2)-AD}), or a more in-depth
introductory course on the subject (\cite{bouyer-cours} and
\cite{cours-raskin}).

\subsection{System Definition}
\begin{definition}[Clocks]
Timed automata feature a special type of variable, called \textit{clocks},
which model the passing of time. Transitions are instantaneous, can reset
clocks (but not set them to a specific value), and have guards referring to the
clock's current value. Within a location, however, time passes at the same rate
for all clocks in the system.
\end{definition}

\begin{definition}[Syntax of Constraints and Actions]
\label{def:formal_methods:transition_grammar2}
Given a set of variables \automatavariables{}, and a set of clocks
\automataclocks{}, the grammar used when writing
constraints and actions in transitions is as follows, with \textbf{ident}
standing for a variable in \automatavariables{}, and \textbf{clk} standing
for a clock in \automataclocks{}:\\
$\textbf{lop} ::= \land~|~\lor $\\
$\textbf{cop} ::=\!\!\! ~< | \le | = | \ge | > $\\
$\textbf{mop} ::= +~| - | * |~/$\\
$
\textbf{val} ::=
   \textbf{ident}
   ~|~ \mathbb{Z}
$\\
$\textbf{mexpr} ::=
   \textbf{mexpr}~\textbf{mop}~\textbf{mexpr}
   ~|~ \textbf{val}
$\\
$
\textbf{bexpr} ::=
   \neg \textbf{bexpr}
   ~|~ \textbf{bexpr}~\textbf{lop}~\textbf{bexpr}
   ~|~ \textbf{mexpr}~\textbf{cop}~\textbf{mexpr}
   ~|~ \textbf{clk}~\textbf{cop}~\textbf{val}
   ~|~ \textbf{clk} - \textbf{clk}~\textbf{cop}~\textbf{val}
   ~|~ \texttt{true}
   ~|~ \texttt{false}
$\\
$\textbf{iexpr} ::=
   \textbf{iexpr} \land \textbf{iexpr}
   ~|~ \textbf{clk}~\textbf{cop}~\textbf{val}
   ~|~ \textbf{clk} - \textbf{clk}~\textbf{cop}~\textbf{val}
   ~|~ \texttt{true}
$\\
$
\textbf{assign} ::=
   \textbf{assign};~\textbf{assign}
   ~|~ \textbf{ident}~:=~\textbf{mexpr}
   ~|~ if~(\textbf{bexpr})~\{\textbf{assign}\}
   ~|~ \textbf{clk} := 0
   ~|~ \texttt{nop}
$
\end{definition}

\begin{definition}[Locations]
Unlike in classic automata, states are referred to as \textit{locations}, since a state is now defined by a location \textit{and} a value for each clock (and a value for each integer variable in our framework).
% This can also be used to force a location to be left
% before a certain amount of time passes.
% \iffalse
% Indeed, without additional constraints
% such as an \texttt{urgent} channel synchronization, there is no obligation for
% activeable transitions to be taken immediately.
%\fi
Time related attributes can be applied to locations:
\begin{description}
\item[\texttt{urgent}:] This location must be left before any time passes.
\item[\texttt{committed}:] This location must be left before any time passes,
and only transition leaving a \texttt{committed} location are enabled.
\item[Invariant in \textbf{iexpr}:] Locations can feature an invariant on
clocks, which must be verified in order for the location to exist.
\end{description}
\end{definition}

The difference between an \texttt{urgent} and a \texttt{committed} location is
only meaningful if there are multiple automata.
Example~\ref{ex:urgent_vs_committed_locations} uses a network of automata to
illustrate the difference between these two attributes.

\begin{definition}[Timed Automata System]
A timed automata system \automatasystem{} is a
$\langle \automatastates{}, \allowbreak{}
\automatainvariants{}, \allowbreak{}
\automatainit{}, \allowbreak{}
\automatacondlabels{}, \allowbreak{}
\automatachanlabels{}, \allowbreak{}
\automatachanpriorities{}, \allowbreak{}
\automatavariables{}, \allowbreak{}
\automataclocks{}, \allowbreak{}
\automataactionlabels{}, \allowbreak{}
\automatarelations{}\rangle$ tuple, where:
\begin{itemize}
\item \automatastates{} is a finite set of locations.
   $\automataurgentlocations \subseteq \automatastates{}$
   denotes the \texttt{urgent} locations, and
   $\automatacommittedlocations \subseteq \automatastates{}$
   the \texttt{committed} ones.
   $\automataurgentlocations \cap \automatacommittedlocations = \emptyset$.
\item
   $\automatainvariants{} : \automatastates{} \to \textbf{iexpr}$ indicates
   the invariant of each location.
\item \automatainit{} is the initial location ($\automatainit{} \in
\automatastates{}$).
\item \automatavariables{} is a finite set of variables.
\item \automataclocks{} is a finite set of clocks.
\item
   $\automatachanlabels{} = \automatachanlabels{}^{\alpha} \cup
   \automatachanlabels{}^{\texttt{sync}}$ is a finite set of labels, with
   $\automatachanlabels{}^{\texttt{sync}}$ corresponding to labels meant for
   synchronization and $\automatachanlabels{}^{\alpha}$ being regular labels,
   with
   $\automatachanlabels{}^{\texttt{sync}} \cap \automatachanlabels{}^{\alpha}
   = \emptyset$.
   The labels in $\automatachanlabels{}^{\texttt{sync}}$ affixed by either `?'
   or `!', with `?' denoting a reception on a ``channel'', and `!' an emission.
   In addition, synchronization labels can be further categorized
   into $\automataurgentchans{} \subseteq \automatachanlabels{}^{\texttt{sync}}$
   corresponding to the \texttt{urgent} channels, and
   $\automatabroadcastchans{} \subseteq \automatachanlabels{}^{\texttt{sync}}$
   corresponding to the \texttt{broadcast} ones.
\item
   $\automatachanpriorities{}:
      \automatachanlabels{}^{\texttt{sync}}
      \to set(\automatachanlabels{}^{\texttt{sync}})$
   indicates, for any label, which labels have a strictly lower priority. It satisfies the following properties:
   $\forallin{l_1,l_2}{\automatachanlabels{}^{\texttt{sync}}}{%
      (l_1 \notin \automatachanpriorities{}(l_1)) \land (l_2 \in \automatachanpriorities{}(l_1))
      \implies
      (
         l_1 \not\in \automatachanpriorities{}(l_2)
         \land
         \automatachanpriorities{}(l_2) \subset \automatachanpriorities{}(l_1)
      )
   }
   $.
\item
   \automatacondlabels{} = \textbf{bexpr}(\automatavariables{},
   \automataclocks{}), as defined in
   Definition~\ref{def:formal_methods:transition_grammar2}.
\item
   \automataactionlabels{} = \textbf{assign}(\automatavariables{},
   \automataclocks{}), as defined in
   Definition~\ref{def:formal_methods:transition_grammar2}.
\item
   $\automatarelations{} \subseteq
      \automatastates{}
      \times \automatacondlabels{}
      \times \automatachanlabels{}
      \times \automataactionlabels{}
      \times \automatastates{}
   $ is the transition relation.
\end{itemize}
The semantics of \automatasystem{} is given via its valid transitions
(see Definition~\ref{sec:system-definition}) and its execution traces
(see Definition~\ref{def:formal_methods:trace2}).
\end{definition}

\begin{definition}[Clock Valuation]
Clocks are kept separate from standard variables, including in the definition of
the valuation. $\automataclockvals{}: \automataclocks \to \mathbb{R}^+$ is the
function mapping each clock to its valuation. As a shorthand, the increment
of the value of all clocks in \automataclockvals{} by $t$ units of time is
written $(\automataclockvals{} + t)$.
\end{definition}

\begin{definition}[Transition]
\label{sec:system-definition}Let $\automatasystem{} = \allowbreak{}
\langle \automatastates{}, \allowbreak{}
\automatainvariants{}, \allowbreak{}
\automatainit{}, \allowbreak{}
\automatacondlabels{}, \allowbreak{}
\automatachanlabels{}, \allowbreak{}
\automatachanpriorities{}, \allowbreak{}
\automatavariables{}, \allowbreak{}
\automataclocks{}, \allowbreak{}
\automataactionlabels{}, \allowbreak{}
\automatarelations{}\rangle$ be an automaton. Given a location $\automatastate{} \in \automatastates{}$, a valuation $\automataenvironment{}$, a clock valuation $\automataclockvals{}$, a duration $t$ and a transition $\langle \automatastate{}, c, l, a, \automatastate{}' \rangle \in \automatarelations{}$, we define the set of the reachable states (without considering priorities) $\automatareach{}(\langle \automatastate{}, \automataenvironment{}, \automataclockvals{}\rangle, \langle \automatastate{}, c, l, a, \automatastate{}' \rangle, t) \triangleq \{\langle \automatastate{}', \automataenvironment{}', \automataclockvals{}' \rangle |
%      (o = \automatastate{})
%      \land (d = \automatastate{}')
      \langle \automataenvironment{}, (\automataclockvals{} + t) \rangle \models_{PL} c
      \land
      \automataenvironment{}' = \automataenvironment{}[a]
      \land
      \automataclockvals{}' = (\automataclockvals{} + t)[a]
      \land
      \langle \automataenvironment{}, \automataclockvals{} \rangle \models_{PL} \automatainvariants{}(\automatastate{})
      \land
      \langle \automataenvironment{}', \automataclockvals{}' \rangle \models_{PL} \automatainvariants{}(\automatastate{}')\}$.

We now define \automatanext{}, which
indicates all valid transitions that can be performed, and takes into account that no transition with higher priority is doable.

$\automatanext{}(\langle \automatastate{}, \automataenvironment{}, \automataclockvals{}\rangle, t)
\triangleq
\{\langle \automatastate{}', \automataenvironment{}', \automataclockvals{}' \rangle |
  \existsin{\langle \automatastate{}, c, l, a, \automatastate{}' \rangle \rangle}{\automatarelations}{\langle \automatastate{}', \automataenvironment{}', \automataclockvals{}' \rangle \in \automatareach{}(\langle \{ \automatastate{}, \automataenvironment{}, \automataclockvals{}\rangle, t) \land \\
      \forallin{\langle \automatastate{b}, c_b, l_b, a_b, \automatastate{b}' \rangle}{\automatarelations{}}{\text{ if } l \in \automatachanpriorities{}(l_b) \text{ then }
\automatareach{}(\langle \{ \automatastate{}, \automataenvironment{}, \automataclockvals{}\rangle, \langle \automatastate{b}, c_b, l_b, a_b, \automatastate{b}' \rangle, t) \text{ is empty.}}
% \forallin{\langle \automatastate{}'', \automataenvironment{}'', \automataclockvals{}'' \rangle}{}{}
%          (\langle \automataenvironment{}, (\automataclockvals{} + t) \rangle \models_{PL} c_b)
%          \land
%          \automataenvironment{}'' = \automataenvironment{}[a_b]
%          \land
%          \automataclockvals{}'' = (\automataclockvals{} + t)[a_b]
%          \land
%          \langle \automataenvironment{b}'', \automataclockvals{b}'' \rangle \models_{PL} \automatainvariants{}(\automatastate{b}')
%          \land
%          \langle \automataenvironment{b}'', \automataclockvals{b}'' \rangle \models_{PL} \automatainvariants{}(\automatastate{b}')
%          \land
%          l \in \automatachanpriorities{}(l_b) \allowbreak{}
%       }
   }\}
$
\end{definition}

\begin{definition}[Path \& Trace]
\label{def:formal_methods:trace2}
We consider a path to be a \emph{maximal} sequence of states/transitions
$\langle \automatastate{1}, \automataenvironment{1}, \automataclockvals{1}\rangle
\automatatransition{}^{\!\!\!t_1} \langle \automatastate{2},
\automataenvironment{2}, \automataclockvals{2}\rangle \automatatransition{}^{\!\!\!t_2} \cdots$ such that $
   \forall i%
      \langle \automatastate{i+1}, \automataenvironment{i+1}, \automataclockvals{i+1}\rangle \in \automatanext{}(\langle \automatastate{i}, \automataenvironment{i}, \automataclockvals{i}\rangle, t)
$. The sequence is maximal in the sense that it is either infinite or of length $N$ and such that $\automatanext{}()$ is empty for any $t\in \mathbb{R}$.
A path starting from $ \langle
\automatainit{}, \automataenvironment{0},  \automataclockvals{0}\rangle$ (where
\automataenvironment{0} is the initial valuation and \automataclockvals{0}
associates each clock with 0) is called a trace.
\end{definition}

\begin{figure}[hbt!]
   \centering
   \begin{tabular}{cc}
   \begin{tikzpicture}[->,>=stealth',shorten >=1pt,auto,node distance=3cm,
                    semithick]
   \node[initial,state] (S0)              {$S_0$};
   \node[state] (S1) [right of=S0] {$S_1$};
   \node[state] (S2) [below of=S0] {$S_2$};
   \node[state] (S3) [below of=S1] {$S_3$};

   \path[every node/.style={sloped, anchor=center, yshift=1em}]
      (S0) edge [bend left] node [above=-1em]{
      \begin{tabular}{l}
         $\textbf{chan0}!$\\
      \end{tabular}
      } (S1)

      (S1) edge [bend left] node [below=1em]{
      \begin{tabular}{l}
         $\textbf{chan1}?$\\
      \end{tabular}
      } (S0)

      (S0) edge [bend right] node [below=1em]{
      \begin{tabular}{l}
         $\textbf{chan2}!$\\
      \end{tabular}
      } (S2)

      (S1) edge [bend left] node [above=-1em]{
      \begin{tabular}{l}
         $\textbf{chan3}?$\\
      \end{tabular}
      } (S3)
   ;
\end{tikzpicture}
 &
   \begin{tikzpicture}[->,>=stealth',shorten >=1pt,auto,node distance=3cm,
                    semithick]
   \node[initial,state] (S0)              {$S_4$};
   \node[state] (S1) [right of=S0] {$S_5$};
   \node[state] (S2) [below of=S0] {$S_6$};
   \node[state] (S3) [below of=S1] {$S_7$};

   \path[every node/.style={sloped, anchor=center, yshift=1em}]
      (S0) edge [bend left] node [above=-1em]{
      \begin{tabular}{l}
         $\textbf{chan0}?$\\
      \end{tabular}
      } (S1)

      (S1) edge [bend left] node [below=1em]{
      \begin{tabular}{l}
         $\textbf{chan1}!$\\
      \end{tabular}
      } (S0)

      (S0) edge [bend right] node [below=1em]{
      \begin{tabular}{l}
         $\textbf{chan2}!$\\
      \end{tabular}
      } (S2)

      (S1) edge [bend left] node [above=-1em]{
      \begin{tabular}{l}
         $\textbf{chan3}?$\\
      \end{tabular}
      } (S3)
   ;
\end{tikzpicture}

   \end{tabular}\\
   \begin{tikzpicture}[->,>=stealth',shorten >=1pt,auto,node distance=3cm,
                    semithick]
   \node[initial,state] (S0)              {$S_8$};

   \path[every node/.style={sloped, anchor=center, yshift=1em}]
      (S0) edge [loop above] node [above=-1em]{
      \begin{tabular}{l}
         $\textbf{chan3}!$\\
      \end{tabular}
      } (S0)

      (S0) edge [loop below] node [below=1em]{
      \begin{tabular}{l}
         $\textbf{chan2}?$\\
      \end{tabular}
      } (S0)
   ;
\end{tikzpicture}

   \caption{Example of Network of Timed Automata}
   \label{fig:timed_automata_urgent_comitted}
\end{figure}

\begin{example}[Urgent and Committed Locations]
\label{ex:urgent_vs_committed_locations}
Consider the network of timed automata shown in Figure~\ref{fig:timed_automata_urgent_comitted}.
Without any attributes on locations, the following path is legal:
$
   \langle \langle S_0, S_4, S_8 \rangle, \{\},  \{\langle C_0, 0\rangle\}\rangle \allowbreak
   \automatatransitiontrace{\langle \textbf{chan0!}, \textbf{chan0?}, -\rangle}{}^{75}
   \allowbreak
   \langle \langle S_1, S_5, S_8 \rangle, \{\}, \{\langle C_0, 75\rangle\}\rangle \allowbreak
   \automatatransitiontrace{\langle \textbf{chan3?}, -, \textbf{chan3!}\rangle}{}^{39}
   \allowbreak
   \langle \langle S_3, S_5, S_8 \rangle, \{\}, \{\langle C_0, 114\rangle\}\rangle \allowbreak
$
Let us now consider $S_1$ as \texttt{urgent}. The previous
$\langle \langle S_1, S_5, S_8 \rangle, \{\}, \{\langle C_0, 75\rangle\}\rangle \allowbreak
\automatatransitiontrace{\langle \textbf{chan3?}, -, \textbf{chan3!}\rangle}{}^{39}$
transition is no longer legal, as a maximum of $0$ units of time is now allowed
to be stayed at $S_1$.
If we instead considered $S_5$ to be \texttt{committed}, that transition would
still not be allowed to occur after more than $0$ units of time, but in
addition, it would also be illegal because it does not involve a transition
from a \texttt{committed} location (neither $S_1$ nor $S_8$, the two locations
involved in the transition, are \texttt{committed}). Performing
$\langle \langle S_1, S_5, S_8 \rangle, \{\}, \{\langle C_0, 75\rangle\}\rangle \allowbreak
\automatatransitiontrace{\langle -, \textbf{chan3?}, \textbf{chan3!}\rangle}{}^{0}$
instead would be legal (although the resulting state is different).
\end{example}

\begin{definition}[Channel Attributes]
In UPPAAL, all synchronizations have a single emitter transition, but the
number of receivers depends on the type of channel: standard channels have
exactly one receiver, and \texttt{broadcast} channels activate any automaton
able to perform a reception on that channel (even allowing the emitter to
transition alone if no receiver is available). It should be noted that available
receivers for the channel are forced to transition.

Communication channels can have attributes related to time. Indeed, the
\texttt{urgent} attribute indicates that, if a transition featuring this channel
is able to occur, it must do so before any time passes.

In a network of automata, the \automatachanpriorities{} function is shared by
all automata. Thus, the priority of channels is the same accross all automata.
\end{definition}

\begin{example}[Urgent and Broadcast Channels]
\label{ex:urgent_vs_committed_locations}
Consider the network of timed automata shown in Figure~\ref{fig:timed_automata_urgent_comitted}.
Without any attributes on channels, the following paths are legal:
\begin{itemize}
\item
$
   \langle \langle S_0, S_4, S_8 \rangle, \{\}, \{\}\rangle \allowbreak
   \automatatransitiontrace{\langle \textbf{chan0!}, \textbf{chan0?}, -\rangle}{}^{75} \allowbreak
   \langle \langle S_1, S_5, S_8 \rangle, \{\}, \{\}\rangle \allowbreak
   \automatatransitiontrace{\langle \textbf{chan3?}, -, \textbf{chan3!}\rangle}{}^{39} \allowbreak
   \langle \langle S_3, S_5, S_8 \rangle, \{\}, \{\}\rangle \allowbreak
   \automatatransitiontrace{\langle -, \textbf{chan3?}, \textbf{chan3!}\rangle}{}^{42} \allowbreak
   \langle \langle S_3, S_7, S_8 \rangle, \{\}, \{\}\rangle
$
\item
$
   \langle \langle S_0, S_4, S_8 \rangle, \{\}, \{\}\rangle \allowbreak
   \automatatransitiontrace{\langle \textbf{chan2!}, -, \textbf{chan2?}\rangle}{}^{23}
   \allowbreak
   \langle \langle S_2, S_4, S_8 \rangle, \{\}, \{\}\rangle \allowbreak
$
\item
$
   \langle \langle S_0, S_4, S_8 \rangle, \{\}, \{\}\rangle \allowbreak
   \automatatransitiontrace{\langle -, \textbf{chan2!}, \textbf{chan2?}\rangle}{}^{0}
   \allowbreak
   \langle \langle S_0, S_6, S_8 \rangle, \{\}, \{\}\rangle \allowbreak
   \automatatransitiontrace{\langle \textbf{chan2!}, -, \textbf{chan2?}\rangle}{}^{78}
   \allowbreak
   \langle \langle S_2, S_6, S_8 \rangle, \{\}, \{\}\rangle \allowbreak
$
\end{itemize}
Let us now consider \textbf{chan0} as \texttt{urgent}.
$
   \langle \langle S_0, S_4, S_8 \rangle, \{\}, \{\}\rangle \allowbreak
   \automatatransitiontrace{\langle \textbf{chan0!}, \textbf{chan0?}, -\rangle}{}^{75}
   \allowbreak
$
is no longer a legal transition: as the synchronization on \textbf{chan0} is
able to occur, it must do so before any time passes. The transition would be
legal if $75$ was $0$ instead. Furthermore, the transition
$
   \langle \langle S_0, S_4, S_8 \rangle, \{\}, \{\}\rangle \allowbreak
   \automatatransitiontrace{\langle \textbf{chan2!}, -, \textbf{chan2?}\rangle}{}^{23}
$ is also not legal, for the same reason: despite \textbf{chan0} not being
synchronized on, it was still an available synchronization and so no time must
pass while the \textbf{chan0} synchronization is available. Replacing $23$ by
$0$ makes this a legal transition. The third proposed path is still allowed
when \textbf{chan0} is marked as urgent: on the first transition, \textbf{chan0}
can be synchronized on, but no time passes prior to the transition occuring;
on the second transition, \textbf{chan0} can no longer be synchronized on, and
so its \texttt{urgent} attribute is not taken into account.

Let us now consider \textbf{chan3} as a broadcast channel.
$
\langle \langle S_1, S_5, S_8 \rangle, \{\}, \{\}\rangle \allowbreak
   \automatatransitiontrace{\langle \textbf{chan3?}, -, \textbf{chan3!}\rangle}{}^{39} \allowbreak$
is not longer a legal transition: $S_5$ is able to synchronize on \textbf{chan3}
as well, and thus must do so. On the other hand, a previously illegal transition
is now possible:$
\langle \langle S_1, S_5, S_8 \rangle, \{\}, \{\}\rangle \allowbreak
   \automatatransitiontrace{\langle \textbf{chan3?}, \textbf{chan3?}, \textbf{chan3!}\rangle}{}^{39} \allowbreak
$.
Perhaps more surprisingly, the following path is legal:
$
\langle \langle S_0, S_4, S_8 \rangle, \{\}, \{\}\rangle \allowbreak
   \automatatransitiontrace{\langle -, -, \textbf{chan3!}\rangle}{}^{3} \allowbreak
\langle \langle S_0, S_4, S_8 \rangle, \{\}, \{\}\rangle \allowbreak
   \automatatransitiontrace{\langle -, -, \textbf{chan3!}\rangle}{}^{89} \allowbreak
$.
Indeed, all automata able to synchronize (none) do so. Note that only the
receiving label can have any number of occurrences in the transition, there is
always a single emitting label. Thus, setting \textbf{chan2} as a broadcast
channel instead would not cause any change in legal paths.
\end{example}

\begin{figure}[hbt!]
   \centering
   \begin{tabular}{cc}
   \begin{tikzpicture}[->,>=stealth',shorten >=1pt,auto,node distance=3cm,
                    semithick]
   \node[initial,state] (S0)              {$S_0$};
   \node[state] (S1) [right of=S0] {$S_1$};

   \path[every node/.style={sloped, anchor=center, yshift=1em}]
      (S0) edge [bend left] node [above=-1em]{
      \begin{tabular}{l}
         $\textbf{request\_files}!$\\
         $\textit{first}$
         $\textit{fetched} := 0; C_0 := 0$
      \end{tabular}
      } (S1)

      (S1) edge [bend left] node [below=1em] {
      \begin{tabular}{l}
         $\textbf{done}?$\\
         $\textit{first} := \bot$
      \end{tabular}
      } (S0)

      (S1) edge [loop right] node [below=1em] {
      \begin{tabular}{l}
         $\textbf{new\_file}?$\\
         $\textit{fetched} := \textit{fetched} + 1$\\
      \end{tabular}
      } (S1)
   ;
\end{tikzpicture}
 &
   \begin{tikzpicture}[->,>=stealth',shorten >=1pt,auto,node distance=3cm,
                    semithick]
   \node[initial,state] (S0)              {$S_0$};
   \node[state,label=below:{$C_1 \le 64$}] (S1) [right of=S0] {$S_1$};

   \path[every node/.style={sloped, anchor=center, yshift=1em}]
      (S0) edge [bend left] node [above=-1em]{
      \begin{tabular}{l}
         $\textbf{request\_files}?$\\
         $\textit{sent} := 0; C_1 := 0$\\
      \end{tabular}
      } (S1)

      (S1) edge [bend left] node [below=1em] {
      \begin{tabular}{l}
         $\textbf{done}!$\\
         $\textit{sent} = 386$
      \end{tabular}
      } (S0)

      (S1) edge [loop right] node [below=1em] {
      \begin{tabular}{l}
         $\textbf{new\_file}!$\\
         $\textit{sent} < 386 \land C_1 \ge 32$\\
         $\textit{sent} := \textit{sent} + 1; C_1 := 0$
      \end{tabular}
      } (S1)
   ;
\end{tikzpicture}

   \end{tabular}
   \caption{Another example of timed automata}
   \label{fig:timed_automata}
\end{figure}

\begin{example}
\label{ex:timed_automata}
In an evolution from Example~\ref{ex:classical_automata},
Figure~\ref{fig:timed_automata} shows a network of automata modeling the exact
same scenario, but with the addition temporal behaviors. The server now
takes between 32 and 64 time units before providing each file. Transfer times
are thus able to be modeled. To avoid having both automata wait forever in
their $S_0$ location, we consider the \textbf{request\_file} channel as being
\texttt{urgent}. Similarly, the \textbf{done} channel needs to be made urgent
for the $S_1$ locations to be left as soon as all files were transfered. Instead
of having the \textbf{request\_file} channel be \texttt{urgent}, another
solution would be to mark the client's $S_0$ location as \texttt{urgent}. This
would also lead to a deadlock after all transfers have been done.
\end{example}

\iffalse
\begin{definition}[Synchronizations]
In regards to actions performed during a transition, the emitter's actions
are performed prior to that of the receivers. This makes it possible for the
emitter to store a value in a variable, and have the receiver read that new
value within a single step, much like a message being passed.
\end{definition}
\fi


\subsection{Query Logic Operators and Semantics}
\label{sec:uppaal_queries}
UPPAAL uses the temporal operators described in
Section~\ref{sec:automata_query_logic}. In this case however, $\phi$ is
a
$
\textbf{bexpr} ::= \allowbreak
   \neg \textbf{bexpr} \allowbreak
   ~|~ \textbf{bexpr}~\textbf{lop}~\textbf{bexpr} \allowbreak
   ~|~ \textbf{mexpr}~\textbf{cop}~\textbf{mexpr} \allowbreak
   ~|~ \textbf{clk}~\textbf{cop}~\textbf{val} \allowbreak
   ~|~ \textbf{clk} - \textbf{clk}~\textbf{cop}~\textbf{val} \allowbreak
   ~|~ \texttt{true} \allowbreak
   ~|~ \texttt{false} \allowbreak
   ~|~ \texttt{deadlock} \allowbreak
   ~|~ \textbf{sub-automaton}.\textbf{location} \allowbreak
$,
with \texttt{deadlock} being true if and only if no transition is able to occur
and time is not allowed to pass, and \textbf{sub-automaton}.\textbf{location}
being true if and only if said sub-automaton is currently in the given
location.

In addition to these temporal operators, UPPAAL features operators that seek
the extremum of either a clock or a variable, with the possibility of an
invariant delimiting the system locations in which the value is considered. Given
either a variable or a clock $t$,
$\texttt{sup}\{\phi\}: t \triangleq$
maximum value for $t$ across all traces, such that this value has been reached
in a state satisfying $\phi$.\\
The $\texttt{inf}$ operator is the equivalent for finding the minimum value.

\begin{example}
In Example~\ref{ex:timed_automata}, computation of the WCET for file fetching
can now be achieved through model checking:
$\texttt{sup}\{\textit{client}.S_1\}: C_0$.  This query looks for the maximum
value reached by $C_0$ when the \textit{client} automaton is in the $S_1$
location.
Considering $max$ to be the returned value, a trace leading to that value could
be obtained by using UPPAAL to query
for $\agop{}~\neg((C_0 = max) \land client.S1)$
\end{example}
