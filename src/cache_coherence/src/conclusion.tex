\section{Conclusion}
This chapter ensured readers have sufficient knowledge of cache coherence
mechanisms to understand the concepts shown in this thesis. Indeed, it provided
a formalization and examples in order to allow the readers to familiarize
themselves with the MSI protocol.

It should be noted that the MSI protocol presented here is a very basic cache
coherence protocol, even in its split-transaction bus adaptation. Despite its
primitive nature, the complex behaviors it allows to result from seemingly
simple sequences of instructions make the analysis of its impact on software
execution difficult to estimate. Especially when considering that, outside of
the content of these sequences of instructions, all the behaviors are done
automatically and outside of the user's control.

While its predictability is an issue, the benefits of using cache coherence on
an architecture to exchange data between concurrent software are too important
to simply go without. The next part presents existing solutions proposed to
tackle this problem.
